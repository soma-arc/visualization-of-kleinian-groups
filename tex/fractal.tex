\section{Fractal}
フラクタルは数学者は数学者ブノワ・マンデルブロが提唱した幾何学の概念である.フラクタルとは,自己相似性という特徴を持つ.どこを拡大しても自分と同じ形を見ることができる.自然界にも多くの形を見ることができる.
コンピュータの普及によって,アートとしても大きく発展した.
Fractal Forumには大きなコミュニティが形成されており,多くのフラクタルファンが日夜議論を重ねている.

\subsection{About Rendering}
これまではCPUによって計算されることが多かったが,グラフィクス性能の向上により,GPUを用いた計算を容易に行うことができるようになった.GPUは浮動小数点演算と並列演算が優れている・
ピクセルごとに計算することができるので,容易に並列計算を行うことができる.
最近ではGPUを用いた高速な並列計算を行うことができる.OpenGL Shading Language(GLSL)やHigh Level Shading Language(HLSL)のフラグメントシェーダを用いる方法である.フラグメントシェーダは本来,ポリゴンに陰影をつけるために使用されるが,スクリーンスペースに矩形のポリゴンをレンダリングすることで各ピクセルごとの処理を記述し計算を行うことができる.
その性質上,何か図を描くときには陰関数表示された関数から図を描く.ただし,より扱いやすくするために,最短距離を返す距離関数(Distance Function)を用いることが多い.例えば,円の距離関数はこの関数は以下のようになる.この関数は演習からの距離を返す.微分と陰関数から導出することができる.この手法はDistance Estimationと呼ばれる.詳しい導出はInigo Quilez氏の記述が詳しい.
距離関数は二次元平面だけでなく,三次元空間でも定義することができる.
シェーダ用いて三次元物体を描画する場合にはレイトレーシングが用いたれる.視点からスクリーンへレイ(光線)を飛ばし,その挙動をシミュレーションすることで描画する手法である.レイの方向と位置から物体との交差点を代数的に計算する.
また,レイマーチングという手法を用いることで,距離関数を用いて任意の三次元形状との交差点を近似的に解くことができる.この方法は,物体の形状が代数的に計算することが難しいフラクタル形状のレンダリングに適している.GLSLによるレンダリングや,レイマーチングに関しては,doxas氏のwgldが詳しい.
デモシーンというコミュニティで用いられる.
\subsection{Escape-time Fractals}
Escape-time Fractalsは複素数平面上の点に対して,特定の漸化式を計算し,その点の軌道の収束,発散を見ることで描画されるフラクタルである.
マンデルブロ集合は最も有名ともいえるフラクタル図形であるが,その漸化式は以下のようになる.
\begin{eqnarray*}
 \begin{cases}
  z_{n+1} = z^2_{n} + c \\ z_0 = 0
 \end{cases}
\end{eqnarray*}
$z_n$が無限大に発散しないものをマンデルブロ集合と呼ぶ.図では黒い部分がマンデルブロ集合である.先端部分が拡大しても見ることができることがわかる.マンデルブロ集合はその単純な式から驚くほど豊富なバリエーションの図を見ることができる.
カラーリングには様々な方法がある.図では発散と判定されるまでの計算の回数によって色をつけた.その他のEscape-time fractalには,ジュリア集合やファタウ集合等がある.
各点における漸化式の計算はお互いに干渉しないので,そのままシェーダでの実装が可能であるが,先に述べたDistance Estimationを用いることで,エイリアシングノイズ等を避けてより精細に描画することができる.Green Functionと呼ばれる陰関数を用いる.導出はiq氏の記事が詳しい.Green Functionに関しては,マンデルブロ集合に関する詳細な議論となってしまうので,こちらの論文を参考にされたい.Doaudy Hubard Potensialを用いることでカラーリングやテセレーションに応用することができる.

\subsection{Distance Estimated 3D Fractals}
メンガーのスポンジ等の形状は見られたが,マンデルブロ集合のような複雑な3次元形状を持つフラクタルが求められていた.マンデルブロ集合を3次元に拡張する試みはさまざまに行われていたものの,あまり大きな成果はあげられていなかった.しかし,年に発表された阿原・荒木による4次元クライン群の一種であるQuasi Fuchsian 3D Fractalsはある程度の複雑性を持った3次元フラクタルであり,コミュニティに大きな影響を与えたと言われている.このフラクタルに関しては後の章で触れる.その後,年にマンデルブロ集合をうまく拡張したmandelbulbという式が発見された.その後mandelboxなどといった三次元フラクタルをレンダリングする式が次々と作られていった.
漸化式を定義し,Distance Estimationを使ってレイマーチングでレンダリングされる.
これら一連のフラクタルに関してはSyntopiaの一連のポストにまとめられている.
注意が必要なのは,これらのフラクタルは数学の研究対象になりえないことである.
\subsection{Coloring by Orbit trap}

ここでOrbit trapと呼ばれるカラーリングのアルゴリズムを紹介しておく.Escape-timeのような点に軌道が存在

\subsection{Other Fractals}

Iterated Function systemは点に様々な関数を適用して,その集積点をレンダリングするアルゴリズムである.
Fractals EverywhereはIFSによるフラクタルの研究を広めた書籍である.
アルゴリズムやカラーリングに関する事項はThe Fractal Frame Algorithmによくまとまっている.並列計算による高速化に関して以下の論文にまとめられている.

他にはL-Systemなどのアルゴリズムが存在するが,後に記述するクライン群のレンダリングには関連してこないのでここでは触れない.
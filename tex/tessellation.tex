
\section{Tessellation}

テセレーション(Tessellation)はタイリング(Tiling)とも呼ばれ, 平面や空間を平面図形や空間図形で埋め合わせることをいう. 
図\ref{fig:rightTriangular}は直角二等辺三角形を各辺における鏡映変換で移すことで平面を敷き詰めている. 
こうした敷き詰め模様は装飾として太古から親しまれており, アートとしても非常に人気のある分野である.

クライン群の図でも様々なところに敷き詰め模様を見ることができる. 
例えば図\ref{fig:schottky}は黒の基本領域を周囲の円盤の反転で移した敷き詰め模様と捉えることができる.

この章では,シェーダを用いてタイリングを描く方法と双曲タイリングについて簡単にまとめる.

\begin{figure}[h!tbp]
   \begin{center}
    \includegraphics[width=3in, height=3in, keepaspectratio]{../img/tessellation/rightTriangular.pdf}
    \caption{Right triangular tiling}
    \label{fig:rightTriangular}
   \end{center}
\end{figure}

\subsection{About Rendering}

簡単に図\ref{fig:rectTile}のような正方形のタイリングを考える.
この図では図\ref{fig:tile}のように,基本となる正方形を縦横全4種類の変換の組み合わせで動かすことで, 平面全体を正方形で埋め尽くすことができる.
これは前章でみた,変換の語の木を幅優先探索で探索し, 得られた合成変換で正方形を移動させることで描画することができる.

しかし, シェーダを用いる場合はその逆を行う.
すなわち,図\ref{fig:tileMove}のように各ピクセルを敷き詰めを構成する4種の変換で基本タイルへと点を動かす.
そうして,基本タイルへ動かした後で操作の回数や基本タイル状の色を参照し描画する.
このアルゴリズムをまとめると以下のような手順になる
\begin{enumerate}
 \item 基本タイルを見つける
 \item 基本タイルを敷き詰めるための変換を見つける
 \item 各ピクセルを基本タイルに入るまで変換し続ける
 \item 変換の回数や基本タイル上の色を使って色を付ける
\end{enumerate}

\begin{figure}[h!tbp]
   \begin{subfigure}{0.3\textwidth}
   \begin{center}
    \includegraphics[width=2in, height=2in, keepaspectratio]{../img/tessellation/rectTile.pdf}
    \caption{rectangular tiling}
    \label{fig:rectTile}
   \end{center}
  \end{subfigure}
 \hspace*{\fill}
   \begin{subfigure}{0.3\textwidth}
   \begin{center}
    \includegraphics[width=2in, height=2in, keepaspectratio]{../img/tessellation/tile.pdf}
    \caption{Tiling}
    \label{fig:tile}
   \end{center}
  \end{subfigure}
 \hspace*{\fill}
 \begin{subfigure}{0.3\textwidth}
  \begin{center}
   \includegraphics[width=2in, height=2in, keepaspectratio]{../img/tessellation/tileMove.pdf}
   \caption{Move}
   \label{fig:tileMove}
  \end{center}
 \end{subfigure}
 \hspace*{\fill}
\caption{Tile}
\end{figure}

ユークリッド平面状での敷き詰め模様は,数学的に分類がされている.
\cite{hyperbolization}はユークリッド平面上での敷き詰めを双曲平面での敷き詰めに変換する方法を提案する論文であるが, 敷き詰めの分類に関する事項もまとめられている.
また,著者が書いた\emph{morenaments}\footnote{morenaments: \url{http://www.morenaments.de/}}というソフトウェアを用いることで, 各敷き詰めパターンの挙動をみることができる.
『装飾パターンの法則』\cite{tessellationDesign}は装飾に用いられる敷き詰め模様の法則, 分類を解説すると共にデザインへの応用についてもまとめられている.

\subsection{Hyperbolic Tessellation}

双曲タイリングは双曲平面上におけるタイルの敷き詰めである. 
双曲幾何学では,ポアンカレの円盤モデルと呼ばれる.円盤の端がユークリッド平面における無限遠点となり,円弧がこの平面における直線だと考えられる.また,三角形の和が180度以下になるなどの興味深い現象もみられる.
この平面における敷き詰め模様が\emph{双曲タイリング}{\it (Hyperbolic Tessellation)}と呼ばれ,画家のエッシャーはこの図に影響されて有名な『天使と悪魔』や『Circle Limit』といった作品を生み出した.

図\ref{fig:poincare}の半球の内側にある円盤がポアンカレの円盤モデルの円盤である.直線を半径が無限の円盤の円周だとみなすと,あらゆるタイルが円弧で構成されていることがわかる.
円盤の内側と外側は円に関する反転で移りあう.
幾何学的には内側も外側も同じものとして扱う.

円盤は原点中心,半径1の円とする.中央の最も大きなタイルはx軸,y軸,そして2つの円弧で囲まれた領域である.
これらの弧に関する反転を用いて,先ほどのアルゴリズムを用いると図\ref{fig:outer}のような敷き詰め模様を描くことができる.
円盤の内側に属する点は内側の基本タイルに,円盤の外側に属する点は外側の基本タイルに移るため,最終的な点の位置で処理を切り分けることにより,両者の色を区別することができる.
双曲タイリングは,木構造の探索によって計算しようとすると,計算量が多くなるため,円盤の端まで描画することは難しいが,前述のアルゴリズムを用いると端までリアルタイムで描画することができる.

図\ref{fig:outer}では基本タイルとして,3つの角が$\frac{\pi}{2}$,残りの角が$\frac{\pi}{3}$である4辺形を用いたが,基本タイルでも敷き詰めることができるわけではない.
敷き詰め条件は\emph{ポアンカレの張り合わせ定理}として知られている.双曲三角形のタイリング条件は以下のようになる.
\begin{eqnarray*}
\text{3内角を}\frac{\pi}{p},\frac{\pi}{q},\frac{\pi}{r}\text{とする時,}
 \frac{1}{p} + \frac{1}{q} + \frac{1}{r} < 1 \text{を満たす.}
\end{eqnarray*}
双曲三角形を2つ張り合わせることで双曲4辺形を得ることができる.

Vladimir Bulatov氏は円盤を変形させる方法を考案した\cite{bending}.Bulatov氏は双曲平面や双曲空間におけるタイリングや,その変形についてまとめている\footnote{Bulatov Abstract Creations: \url{http://bulatov.org/math/index.html}}.
筆者は図\ref{fig:deformed}のような双曲タイリングとその変形を行うWebアプリケーションを制作している.
\footnote{Hyperbolic Tessellator: \url{https://soma-arc.net/HyperbolicTessellator/}}.


\begin{figure}[h!tbp]
 \begin{minipage}{0.49\hsize}
   \begin{center}
    \includegraphics[width=3in, height=3in, keepaspectratio]{../img/tessellation/poincare.pdf}
    \caption{Poincare disk}
    \label{fig:poincare}
   \end{center}
 \end{minipage}
 \hspace*{\fill}
 \begin{minipage}{0.49\hsize}
   \begin{center}
    \includegraphics[width=3in, height=3in, keepaspectratio]{../img/tessellation/outer.pdf}
    \caption{Hyperbolic Tessellation}
    \label{fig:outer}
   \end{center}
 \end{minipage}
\end{figure}


\begin{figure}[htbp]
 \begin{center}
      \includegraphics[width=3in, height=3in, keepaspectratio]{../img/tessellation/deformedTessellation.pdf}
    \caption{deformed}
    \label{fig:deformed}
 \end{center}
\end{figure}

%#!uplatex main.tex

\section{Fractal}

\emph{フラクタル}(\textit{fractal})は数学者ブノワ・マンデルブロ(Benoit
B. Mandelbrot)が提唱した幾何学の概念である.フラクタル図形とよばれる図形
は自己相似性を持ち,図形のどこを拡大しても全体と同じ形を見ることができる.

図\ref{fig:gasket}に\emph{シェルピンスキーのギャスケッ
ト}(\textit{Sierpinski gasket})と呼ばれるフラクタル図形を示した.
三角形を一定の法則に基づいて分割していくことで得られる図形である.
図\ref{fig:menger}に示される物体は\emph{メンガーのスポンジ}(\textit{Menger
sponge})とよばれ,立方体に一定の法則で穴を空けることで得られる.
これらの形はどこを拡大しても同様の法則性を持った形をみることができる.

海岸線やシダ植物,体内の肺など自然界の多くの場所でもフラクタル構造をみる
ことができる.
フラクタルを考えることは自然科学研究の新たなアプローチとしても発展した.

フラクタルの特異な形状は多くの人々の興味を引き,アートとしても大きく発展
した.これにはコンピュータの普及も大きく関連しており,コンピュータ文化と
は切り離せない.
インターネット上のフラクタルコミュニティとして最大のものに,\textit{Fractal
Forum}\footnote{Fractal Forum:~\url{http://www.fractalforums.com/}}
という電子掲示板がある.
このコミュニティから生まれたフラクタル図形や描画手法はとても多い.

また,\emph{デモシーン}(\textit{Demoscene})という文化もフラクタルレンダリン
グと関連が深い.
デモシーンは\emph{デモ}(\textit{Demo})と呼ばれるリアルタイ
ムにコンピュータグラフィクスや音楽を生成するプログラムを見せ合うコンピュータサブ
カルチャーの一つである.美しいコンピュータグラフィクスをリアルタイムに,
さらに小さなサイズのプログラムで描画することを追求している.
フラクタルは形として面白いことに加え,数式で複雑な形状を描画できるの
でデモと相性がよい.

デモシーンに関しては,\textit{Demoscene - The
Art of the Algorithm}\footnote{Demoscene - The Art of the Algorithms:
~\url{https://www.youtube.com/watch?v=iRkZcTg1JWU}}が詳しい.
また,著名なデモシーナーのInigo Quilez (iq)は自身のウェブページ\footnote{Inigo
Quilez's webpage:~\url{http://iquilezles.org/index.html}}にコンピュータグラフィ
クスに関する様々な記事をまとめている.
ここで紹介する手法もまとめられているので,併せて参考にされたい.

この章では,後のクライン群の可視化に関わりのあるフラクタルとその描画手
法についてまとめる.
ここで記述があるもの以外にも様々なフラクタルが存在している.
マンデルブロ自身の著作である『フラクタル幾何学』
\cite{mandelbrot-ja-201102-1}
\cite{mandelbrot-ja-201102-2}
(原著:~\cite{mandelbrot1983fractal})では種々のフラクタルが網羅的にまとめ
られている.

\begin{figure}[htbp]
 \begin{minipage}{0.5\hsize}
   \center
  \includegraphics[width=3in, height=3in, keepaspectratio]{../img/fractal/gasket.pdf}
  \caption{Sierpinski gasket}
  \label{fig:gasket}
 \end{minipage}
 \begin{minipage}{0.5\hsize}
  \center
  \includegraphics[width=3in, height=3in, keepaspectratio]{../img/fractal/menger.pdf}
  \caption{Menger sponge rendered with Fractal Lab}
  \label{fig:menger}
 \end{minipage}

\end{figure}

\subsection{Shader Based Rendering}

最近のデモ作品でよく用いられる作品制作手法の一つに\textit{OpenGL Shading
Language (GLSL)}や\textit{High Level Shading Language (HLSL)}の\emph{フラ
グメントシェーダ}(\textit{Fragment Shader})を用いるものがある.
フラグメントシェーダは本来,ポリゴンに陰影をつけるために使用され,その処
理はGPUで行なわれる.
GPUは浮動小数演算性能に優れていることに加え,各ピクセルの演算は並列に行
われるため高速である.

この手法はスクリーンスペースに矩形のポリゴンをレンダリングすることで
\emph{プロシージャル}(\textit{Procedural})に図を描画する.
プロシージャルとは,数式やアルゴリズムを用いて形状を生成すること等を
指す.
複雑な形状をあらかじめモデリングされたポリゴンで表現するよりもプログラム
の実行サイズを下げたり,計算速度を向上させたりすることができる.
このことはデモにおける小さなプログラムサイズでリアルタイムにレンダリングする
という目的に適している.
フラクタルはしばしば簡潔なアルゴリズムで複雑な形状を生成するので,デモに
おいてもよく使われる.

シェーダはピクセルごとにその座標とCPU側から渡される変数を受け取る.
これらを用いてピクセルの色を決定する.
アルゴリズム\ref{shaderSample}に座標から色を決定して塗るシェーダの
擬似コードを示した.
図\ref{fig:simpleShader}にそのレンダリング結果を示した.
\begin{algorithm}
 \begin{algorithmic}
  \begin{minipage}{0.5\hsize}
   \caption{Sample shader}
   \label{shaderSample}
   \REQUIRE COORDINATES, resolution
   \STATE uv $\leftarrow$ COORDINATES / resolution
   \STATE COLOR $\leftarrow$ Vector3(uv.x$,~$uv.y$,~1$)
  \end{minipage}
 \end{algorithmic}
\end{algorithm}

幾何図形を描く際には\emph{距離関数}(\textit{Distance Function})がよく用いられる.
距離関数は任意の点が与えられたときに,点と図形の最短距離を返す関数である.
例えば,平面上の点$z$が与えられたとき,原点を中心とする半径$r$の円との距
離を返す距離関数$f(z)$は次のようになる.
\begin{align*}
 f(z) =  length(z) - r
\end{align*}
この関数は円周からの距離を返す.
そのため,平面上の各点に距離関数を作用させ,得られた距離が負であるときに
色をつける処理をシェーダで書くと円を描くことができる.
複数の図形を描きたい時は,複数の距離関数を評価し,最も小さな距離を求めれ
ばよい.
距離関数による描画は,得られた距離によって色の濃さを変えるといった工夫で
図形をより滑らかに描くことができる.
アルゴリズム\ref{renderCircle}に円を描く疑似コードを示した.図
\ref{fig:circleShader}にそのレンダリング結果を示した.

\begin{algorithm}
 \begin{algorithmic}
  \caption{Render circle}
  \label{renderCircle}
  \REQUIRE COORDINATES, resolution, radius
  \STATE p $\leftarrow$ COORDINATES $-$ (resolution$ / 2$)
  \STATE distance $\leftarrow$ length(p) $-$ radius
  \IF{distance $\leq$ 0}
  \STATE COLOR $\leftarrow$ Vector3($1,~1,~1$)
  \ELSE
  \STATE COLOR $\leftarrow$ Vector3($0,~0,~0$)
  \ENDIF
 \end{algorithmic}
\end{algorithm}

 \begin{figure}[htbp]
  \begin{minipage}{0.5\hsize}
   \center
   \includegraphics[ height=1.5in, keepaspectratio]{../img/fractal/uv.pdf}
   \caption{Simple shader}
   \label{fig:simpleShader}
  \end{minipage}
  \begin{minipage}{0.5\hsize}
   \center
   \includegraphics[ height=1.5in, keepaspectratio]{../img/fractal/circle.pdf}
   \caption{Circle}
   \label{fig:circleShader}
  \end{minipage}
 \end{figure}

シェーダを用いたレンダリングはglslsandbox\footnote{glslsandbox:
~\url{http://glslsandbox.com/}}やShadertoy\footnote{Shadertoy:
~\url{https://www.shadertoy.com/}}といったウェブサービスで手軽に試すこと
ができる.h\_doxasによるwgld\footnote{wgld:
~\url{https://wgld.org/d/glsl/}}にはシェーダによるレンダリングの入門事項
がよくまとまっている.

また,GPUの計算性能をより汎用的な計算に用いるためのプラットフォームとしてCUDA
やOpenCLが登場している.
これらのプラットフォームを用いることでシェーダではできない複雑な処理を
GPUで行うことができるようになり,GPUによる並列演算は機械学習などの分野に
も活躍の場を広げた.
このようにGPUによる演算を画像処理以外の汎用的な用途に用いる技術は
\textit{GPGPU (General-purpose computing on graphics processing units)}
とよばれている.

\subsubsection{Distance Estimation}

簡単な幾何図形の距離関数は簡単に導出できるが,そうでない図形やフラクタル
の場合は難しい.
しかし,\textit{Distance Estimation}という手法を用いることで,
距離関数を陰関数から近似的に導出することができる.
その式の一つをiqによる導出\footnote{distance estimation:
~\url{http://iquilezles.org/www/articles/distance/distance.htm}}
をベースにして紹介する.

図\ref{fig:distance-estimate}における赤色の曲線を表す陰関数$f(x)$を考え
る.
平面上のある点を$x$とし,陰関数が表す零点集合上で$x$から最も近い点を
$x + e$とすると,$e$は$x$から最も近い点へ向かうベクトルである.
そして求めたい距離は$|e|$となる.
 \begin{figure}[htbp]
  \center
  \includegraphics[width=3in, keepaspectratio]{../img/fractal/distance-estimate.pdf}
  \caption{Distance Estimation}
  \label{fig:distance-estimate}
 \end{figure}

$f(x + e) = 0$をテイラー展開すると次のようになる.
\begin{align*}
f(x + e) = f(x) + f'(x) \cdot e  + \frac{1}{2}f''(x) \cdot e^2 ...
\end{align*}
ここで$e$が十分に小さいと仮定し,2次以上の項を無視する線型近似を行うこと
で次の近似式を得る.
\begin{align*}
0=|f(x + e)| \approx |f(x) + f'(x) \cdot e|
\end{align*}
ここで,$x,~e$はベクトルであり$\cdot$は内積を表すことに注意する.
三角不等式によって,
\begin{align*}
| f(x) + f'(x) \cdot e| \geq |f(x)| - |f'(x) \cdot e|
\end{align*}
であるから,
\begin{align*}
0 \geq |f(x)| - |f'(x) \cdot e|
\end{align*}
となる.
次に,コーシーシュワルツの不等式より,
\begin{align*}
 |f'(x)| \cdot |e| \geq | f'(x)\cdot e|
\end{align*}
であるから,
\begin{align*}
 0 \geq |f(x)| - |f'(x) \cdot e| \geq |f(x)| - |f'(x)| \cdot |e|
\end{align*}
となる.
最後に
\begin{align*}
0 \geq |f(x)| - |f'(x)| \cdot |e|
\end{align*}
を変形すると
\begin{align*}
|e| \geq \frac{|f(x)|}{|f'(x)|}
\end{align*}
が得られる.
このことから,ある点$x$に対して陰関数$f(x)$が表す零点集合への最短距離を返す近
似関数は次のように表すことができる.
\begin{align*}
 DistanceFunction(x) \approx \frac{|f(x)|}{|\bigtriangledown f(x)|}
\end{align*}
ただし,$e$が十分小さいことを仮定しているため,零点集合から遠い点につい
ては誤差が大きくなってしまうことに注意する.
導関数から得られる微分値には,ヤコビアンや中心差分法等
で導出された微分の近似値が用いられることもある.

\subsubsection{Ray tracing}

シェーダのみで三次元形状を描画する際には\emph{レイトレーシング}({\it Ray
tracing})が用いられる.
レイトレーシングはあらかじめ設定された視点(カメラ)から
\emph{レイ}({\it Ray},光線)を飛ばし,
その挙動を計算することで物体を描画する手法である.
図\ref{fig:raytrace}にレイトレーシングの模式図を示した.
レイと物体の交差点はレイの原点と方向から代数的に計算することができる.

また,\emph{レイマーチング}({\it Ray marching,Sphere tracing}とも呼ばれ
る)~\cite{hart1996sphere}という手法で,物体の距離関数からレイとの交差点を
近似的に求めることができる.
図\ref{fig:raymarch}にレイマーチングの模式図を示した.
レイマーチングではまず,レイの原点$P_0$からこの点に最も近い物体との距
離を距離関数で求める.そしてその距離分だけレイを進めた位置を$P_1$とする.
次に$P_1$に対しても同様の操作を行なう.
この作業を繰り返し,得られた点列の極限点が交差点となる.

レイマーチングでは距離関数をうまく組み合わせることで,三次元形状
の和,差,積をとるような演算も高速に行なえる.
この手法はメッシュによらないプロシージャルな形状表現に適しており,デモシー
ンでも広く使われている.
交差点を代数的に計算することが難しいフラクタル形状をレンダリングする際に
は,Distance Estimationによって形状との距離を求めてレイマーチングを行な
うことが一般的である.

 \begin{figure}[htbp]
  \begin{minipage}{0.5\hsize}
   \center
   \includegraphics[width=3in, keepaspectratio]{../img/fractal/raytrace.pdf}
   \caption{Ray tracing}
   \label{fig:raytrace}
  \end{minipage}
  \begin{minipage}{0.5\hsize}
   \center
   \includegraphics[width=3in, keepaspectratio]{../img/fractal/raymarching.pdf}
   \caption{Ray marching}
   \label{fig:raymarch}
  \end{minipage}
 \end{figure}

\subsection{Escape-time Fractals}

\textit{Escape-time Fractals}は複素平面上の点に対して,特定の漸化式を計算し,
その点の軌道の収束,発散を見ることで描画されるフラクタルである.
Escape-time Fractalsで代表的なものが\emph{マンデルブロ集
合}(\textit{Mandelbrot set})である.
複素平面上の点$c$に対して,以下に示す漸化式を計算し,$\displaystyle \lim_{n
\to \infty} z_n$が無限大に発散しないものをマンデルブロ集合とよぶ.
\begin{align*}
 z,~c \in \mathbb{C} \quad
 \begin{cases}
  z_{n+1} = z^2_{n} + c \\ z_0 = 0
 \end{cases}
\end{align*}

図\ref{fig:mandelbrot}において,赤で縁取られた黒い部分がマンデルブロ集合
である.その他の部分については,発散と判定されるまでの計算の回数によって
色をつけた.
マンデルブロ集合はその単純な式から,描画する位置や拡大率によって驚くほど
豊富なバリエーションの形を見ることができる.

各点における漸化式の計算はお互いに干渉しないので,そのままシェーダでの実
装が可能であるが,先に述べたDistance Estimationを用いることで,エイリア
シングノイズ等を避けてより精細に描画することができる.
これには\textit{Green Function}と呼ばれるマンデルブロ集合の陰関数を用いる.
導出はiqによる記事\footnote{distance rendering for fractals:
~\url{http://iquilezles.org/www/articles/distancefractals/distancefractals.htm}}
が詳しい.
Green Functionについてはマンデルブロ集合に関する詳細な議論となってしま
うので\cite{douady1984exploring}を参考にされたい.

その他のEscape-time fractalには図\ref{fig:julia}に示した
\emph{ジュリア集合}(\textit{Julia set})や\emph{ファトゥー集
合}(\textit{Fatou set})等がある.

\begin{figure}[htbp]
 \begin{minipage}{0.49\hsize}
  \center
   \includegraphics[width=3in, height=3in, keepaspectratio]{../img/fractal/mandelbrot.pdf}
   \caption{Mandelbrot set}
   \label{fig:mandelbrot}
 \end{minipage}
 \begin{minipage}{0.49\hsize}
  \center
  \includegraphics[width=3in, height=3in, keepaspectratio]{../img/fractal/julia.pdf}
  \caption{Julia set}
  \label{fig:julia}
 \end{minipage}
\end{figure}

\subsection{Distance Estimated 3D Fractals}

マンデルブロ集合は数あるフラクタルの中でも特に複雑な形状を見せるものであった
ので,その三次元への拡張も古くから興味が持たれてきた.
例えば,ジュリア集合を四元数を用いて計算した
\textit{Quaternion Julia 3D fractal}~\cite{hart1989ray}が1989年に発表され
ている.
しかし,この形状は図\ref{fig:qjulia}のように四元数による回転対称性をもつ
ため,形状が単純でフラクタルコミュニティの満足のいくものではなかった.

その他にもマンデルブロ集合のような複雑な三次元フラクタルを描画しよう試み
が様々にあった.
それらについてはDaniel White (twinbee)がまとめている
\footnote{Skytopia -The Mystery of the REAL, 3D Mandelbrot Fractal:~
\url{http://www.skytopia.com/project/fractal/mandelbrot.html}}.

そのような中で,2003年に発表された阿原・荒木による四次元クライン群の一種
である\textit{Quasi Fuchsian 3D Fractals}~\cite{ahara2003sphairahedral}
は,ある程度の複雑性を持った三次元形状をもつフラクタルで
あった.そのため,コミュニティに大きな影響を与えたと言われている.
その形状のひとつを図\ref{fig:quasi-fuchsian-3d}に示した.
このフラクタルに関しては2章で触れる.

\begin{figure}[htbp]
  \begin{minipage}[t]{0.49\hsize}
   \center
   \includegraphics[width=3in, height=3in, keepaspectratio]{../img/fractal/qjulia.pdf}
   \caption{Quaternion Julia rendered with Fragmentarium}
   \label{fig:qjulia}
  \end{minipage}
 \begin{minipage}[t]{0.49\hsize}
  \center
  \includegraphics[width=3in, height=3in, keepaspectratio]{../img/fractal/quasi-fuchsian.pdf}
  \caption{Quasi Fuchsian 3D Fractal rendered with Fragmentarium
  (Knighty's script)}
  \label{fig:quasi-fuchsian-3d}
 \end{minipage}
\end{figure}

また,この時期から,コンピュータのグラフィクス性能が向上し,三次元のフラクタ
ル図形をリアルタイムに描画することができるようになった.
例えば,2004年にKeenan CraneはQuaternion JuliaをGPUによってリアルタイムに
描画している\footnote{Ray Tracing Quaternion Julia Sets on
the GPU:
~\url{https://www.cs.cmu.edu/~kmcrane/Projects/QuaternionJulia/}}.
iqは2007年にQuaternion Juliaをテーマにした4k intro~(実行ファイルのサイズ
を4kb以内に抑えたデモ作品),
\textit{kindernoiser}\footnote{kindernoiser:
~\url{http://www.pouet.net/prod.php?which=32549}}
を発表している.

その後,2009年に遂にマンデルブロ集合をうまく三次元に拡張した
\textit{Mandelbulb}が開発された.
これを皮切りに,次々と複雑な三次元フラクタルの式が開発された.
それらの多くはEscape-time fractalの一種であり,
Distance Estimationとレイマーチングを用いることで効率よくレンダリングす
ることができる.
これらのフラクタルの歴史と実装についてはMikael Hvidtfeldt Christensen
による一連の記事\footnote{Syntopia, Distance Estimated 3D Fractals:\\
\url{http://blog.hvidtfeldts.net/index.php/2011/06/distance-estimated-3d-fractals-part-i/}}
にまとめられている.
こちらも併せて参考にされたい.

これらのフラクタルをレンダリングするためのソフトウェアには\textit{Mandelbulb
3D}\footnote{Mandelbulb 3D:
~\url{http://mandelbulb.com/2014/mandelbulb-3d-mb3d-fractal-rendering-software/}}
や\textit{Mandelbulber}\footnote{Mandelbulber:
~\url{http://www.mandelbulber.com/}}が有名である.
シェーダベースのグラフィクスを開発するための環境である
\textit{Fragmentarium}\footnote{Fragmentarium:
~\url{http://syntopia.github.io/Fragmentarium/}}は
フラクタルのサンプルコードが豊富でアルゴリズムの学習に役立つ.
\textit{Fractal Lab}\footnote{Fractal Lab:
~\url{http://hirnsohle.de/test/fractalLab/}}はブラウザ上でGLSLを用いてフ
ラクタルをレンダリングすることができるウェブアプリケーションである.
この章で使われている図のいくつかはFractal LabとFragmentariumを用いてレン
ダリングした.

\subsubsection{Mandelbulb}

MandelbulbはDaniel WhiteとPaul Nylanderが2009年に開発した.
マンデルブロ集合の三次元拡張として広く知られている.

マンデルブロ集合の三次元の拡張において問題になるのは,三次元空間において
用いる代数である.
マンデルブロ集合の漸化式は複素数の乗算と加算で構成されている.
$z^2$は点$z$に原点中心の回転と拡縮を作用させ,加算$z + c$は平行移動を作用さ
せる.これによって点の軌道を考えることができる.
そのため,三次元空間の点において,軌道を考えることができるような$z^2$の
意味付けが必要であった.

Whiteは球面座標上においてマンデルブロ集合の漸化式を計算するアプローチを提案
した
\footnote{True 3D mandelbrot type fractal: \\
\url{http://www.fractalforums.com/3d-fractal-generation/true-3d-mandlebrot-type-fractal/}}
.
図\ref{fig:mandelbulb2}は$z_{n+1} = z_n^2 + c$を用いてレンダリングした結
果であるが,あまり複雑な形状は現れなかった.
しかし,その後Nylanderが式を高次の積を扱えるように拡張した.
図\ref{fig:mandelbulb8}は$z_{n+1} = z_n^8 + c $という式でレンダリングさ
れた形状であり,これがMandelbulbとして知られている.Whiteのウェブページ
\footnote{Skytopia:
~\url{http://www.skytopia.com/project/fractal/mandelbulb.html}}には開発の
詳しい経緯がまとめられている.

また,Mandelbulbも通常のマンデルブロ集合と同様にDistance Estimationによっ
て形状への最短距離を求めることができるため,レイマーチングで描画す
ることができる.

\begin{figure}[h!tbp]
 \begin{subfigure}{0.49\hsize}
  \center
  \includegraphics[width=3in, height=3in, keepaspectratio]{../img/fractal/mandelbulb2.pdf}
  \caption{$z_{n + 1} = z_n^2 + c$}
  \label{fig:mandelbulb2}
 \end{subfigure}
 \hspace*{\fill}
 \begin{subfigure}{0.49\hsize}
  \center
  \includegraphics[width=3in, height=3in, keepaspectratio]{../img/fractal/mandelbulb8.pdf}
  \caption{$z_{n+1} = z_n^8 + c$}
  \label{fig:mandelbulb8}
 \end{subfigure}
 \caption{Mandelbulb rendered with Fractal Lab}
\end{figure}

\subsubsection{Mandelbox}

その後,2010年にTom Loweによって\textit{Mandelbox}\footnote{Mandelbox:
~\url{https://sites.google.com/site/mandelbox/what-is-a-mandelbox}}が開発
された.
Mandelboxも三次元空間におけるEscape-time fractalである.
漸化式は次のようになる.
\begin{align*}
 z_{n+1} = scale * spherefold(minR,~boxfold(z_n)) + c
\end{align*}
ここで\textit{boxfold}は$x=\pm1,~y=\pm1,~z=\pm1$を頂点にもつ立方体の各面
に対して,点がその外側にある場合にそれらの面に関する反転を行なう操作である.
\textit{spherefold}は点が原点中心の球の内側にある時にその球に関す
る反転を行う操作である.
ただし,球の反転は中心付近を無限遠点付近へと移してしまうため,漸化式の計算途中
に中心付近を通る点が発散と判定されてしまう.
そこでspherefoldでは中心付近の点の反転をあらかじめ決めておいた最小の半
径における反転で置きかえることで軌道が発散しないようにしている.
球の半径を$R$,最小の半径を$r$としたときに,$z$に対するspherefoldは以下
の式になる.
\begin{align*}
 spherefold(R,~r,~z) = \begin{cases}
                  z~\frac{R^2}{r^2} & 0 \le |z| < r \\
                  z~\frac{R^2}{|z|^2} & r \le |z| < R
                 \end{cases}
\end{align*}

MandelboxのDistance Estimationでは漸化式の微分としてヤコビアンの積を用いる.
式の各操作のたびにヤコビアンを累積していき,
最終的な点の原点からの距離をヤコビアンの積で割ることで距離を近似的に求
めることができる.

\begin{figure}[h!tbp]
 \begin{subfigure}{0.49\hsize}
  \center
  \includegraphics[width=3in, height=3in, keepaspectratio]{../img/fractal/mandelbox.pdf}
  \caption{}
  \label{fig:mandelbox1}
 \end{subfigure}
 \hspace*{\fill}
 \begin{subfigure}{0.49\hsize}
  \center
  \includegraphics[width=3in, height=3in, keepaspectratio]{../img/fractal/mandelbox2.pdf}
  \caption{}
  \label{fig:mandelbox2}
 \end{subfigure}
 \caption{Mandelbox rendered with Fractal Lab}
 \label{fig:mandelbox}
\end{figure}


\subsubsection{Pseudo-Kleinian}

Mandelboxの登場以降は各$n$ごとに異なる式を組み合わせて漸化式を計算する
\textit{Hybrid System}によって,様々なフラクタル作品が作られている.

興味深いHybrid Systemの一つに\textit{pseudo-kleinian}がある.
これは球の反転に近い操作であるspherefoldと平行移動と同じような働きをする
boxfoldをうまく組み合わせることによって,可視化されたクライン群のような
形状が得られる式である.
これを図\ref{fig:pseudoKleinian}に示した.

\begin{figure}[htbp]
 \begin{minipage}{0.5\hsize}
  \center
  \includegraphics[width=3in, height=3in, keepaspectratio]{../img/fractal/pseudoKleinian.pdf}
  \subcaption{}
 \end{minipage}
 \begin{minipage}{0.5\hsize}
  \center
  \includegraphics[width=3in, height=3in,
  keepaspectratio]{../img/fractal/pseudo-kleinian2.pdf}
  \subcaption{}
 \end{minipage}
  \caption{Pseudo-kleinian rendered with Fragmentarium}
  \label{fig:pseudoKleinian}
\end{figure}


\subsection{Coloring by Orbit Trap}

フラクタルのレンダリングにおいてカラーリングは重要な要素である.
Escape-timeアルゴリズムを用いたフラクタルのカラーリングには,主に\textit{
Orbit Trap}とよばれる手法が使われる.
図\ref{fig:mandelbox}のMandelboxもOrbit Trapを用いて色がつけられている.

Orbit trapはtrapと呼ばれる点や直線などの図形を空間上に置いておき,
trapと漸化式の計算で得られる軌道の点との位置関係から色を付ける方法である.
図\ref{fig:orbitTrap}はマンデルブロ集合とジュリア集合の漸化式において
$z_n$が桃色の領域に入った点を白く塗った.
これは桃色の領域に到達するまでの点の軌道を描いたと考えることができる.

また,この方法を応用すると画像をtrapに用いて,画像を変換で移した軌道を描
くことも可能である.
この手法は特に\textit{Bitmap Orbit Trap}とも呼ばれる.

\begin{figure}[htbp]
 \begin{minipage}{0.49\hsize}
  \center
  \includegraphics[width=3in, height=3in, keepaspectratio]{../img/fractal/mandelbrot-orbit.pdf}
  \subcaption{Mandelbrot set}
  \label{}
 \end{minipage}
 \begin{minipage}{0.49\hsize}
  \center
  \includegraphics[width=3in, height=3in,
   keepaspectratio]{../img/fractal/juliaOrbit.pdf}
  \subcaption{Julia set}
  \label{}
 \end{minipage}
 \caption{Orbit trap}
 \label{fig:orbitTrap}
\end{figure}

\subsection{Iterated Function System}

もう一つ有名なアルゴリズムとして,\textit{Iterated Function System (IFS)}が
ある.IFSは点に複数の関数を反復的に作用させることでフラクタルを描画する
アルゴリズムの総称である.

IFSの描画アルゴリズムの一つとして,カオスゲームがある.
カオスゲームでは平面上の点をランダムにとり,その点にランダムに選んだ関数を作用させることを
繰り返すことで得られた点を描画する.
図\ref{fig:fern}はカオスゲームでレンダリングされたシダ植物である.
シェルピンスキーのギャスケットを含むその他のフラクタルもこの方法でレン
ダリングできるものがある.

IFSを拡張したフラクタルアートの一種に\emph{フラクタルフレーム}\textit{(Fractal
Flame)}~\cite{draves2003fractal}がある.
フラクタルフレームではIFSに用いる関数にメビウス変換などの非線形変換を用
い,美しくレンダリングされるようにカラーリングにも工夫が加えられている.
図\ref{fig:pseudoKleinian}はフラクタルフレームレンダラの一つである,
Apophysis 7Xを用いてレンダリングされたフラクタルフレームである.

IFSに関する書籍として\textit{Fractals
Everywhere}~\cite{BarnsleyMathematics201207}がある.
また,並列計算による高速化に関する論文に
~\cite{2010_fractal_flames}や\cite{Green:2005:GIF:1187112.1187128}がある.

\begin{figure}[htbp]
  \begin{minipage}{0.49\hsize}
   \center
   \includegraphics[width=3in, height=3in, keepaspectratio]{../img/fractal/fern.pdf}
   \caption{Fern}
   \label{fig:fern}
 \end{minipage}
 \begin{minipage}{0.49\hsize}
  \center
  \includegraphics[width=3in, height=3in, keepaspectratio]{../img/fractal/fractalFlame.pdf}
  \caption{Fractal Flame}
  \label{fig:fractalFlame}
 \end{minipage}
\end{figure}

\clearpage

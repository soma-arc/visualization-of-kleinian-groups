\documentclass[dvipdfmx]{article}
\usepackage{amsmath, amsthm, amssymb}
\usepackage{ascmac,here,txfonts,txfonts}
\usepackage[margin=2.5cm,noheadfoot]{geometry}
\usepackage{color}
\usepackage{graphicx}
\usepackage{hyperref}
\setlength{\parindent}{0.3in}

\title{ Visualization of Kleinian Groups }
\author{ Kento Nakamura\\
Meiji University\\
}

\date{}

\begin{document}

\maketitle

\begin{abstract}
クライン群(Kleinian Groups)はトポロジーや双曲幾何学に属する数学の1分野である.クライン群は様々な方法で可視化することで,フラクタルと呼ばれる美しい構造を見ることができる.この本では主にクライン群の可視化手法とその高速化についてまとめた.また,それに付随して,テセレーション(Tessellation)と呼ばれる敷き詰め模様とその他のフラクタル図形のレンダリング手法についてもまとめた.

1章ではいくつかのフラクタル図形とそのレンダリング手法についてまとめる.ここで述べる手法は後でクライン群を効率よく描画するための基礎となる.
2章では筆者が主に学習,実装したクライン群について記述する.
3章では敷き詰め模様のレンダリングのレンダリング手法をまとめる.
\end{abstract}

\tableofcontents

\section{Fractal}

\emph{フラクタル}は数学者ブノワ・マンデルブロが提唱した幾何学の概念である.
自己相似性をもつ図形はどこを拡大しても全体と同じ形を見ることができる.
図\ref{fig:gasket}は\emph{シェルピンスキーのギャスケット}と呼ばれるフラクタル図形である.
三角形を一定の法則に基づいて分割していくことで,無限に続いていく三角形が得られる.
その形はどこを拡大しても同様の法則性を持った図形が得られる.
また,自然界の多くの場面でもフラクタル構造をみることができる. 
海岸線や, シダ植物などにもフラクタル構造をみることができる. 
フラクタルを考えることは自然科学研究の新たなアプローチとしても発展した. 

自己相似性を持つ図形はフラクタルという名がつけられる以前から研究されていた.
また,アートとしてもフラクタルを連想するような作品が存在していたが,
コンピュータが普及することによって,様々なフラクタル図形を描画することができるようになり,アートとしても大きく発展した.
この章ではコンピュータで計算,描画されるフラクタルを紹介する.

アートやコンピュータ文化との関連は切り離せない. これらのフラクタルは多くの場合数学的な研究対象にならないものの.
Fractal Forum\footnote{Fractal Forum:  \url{http://www.fractalforums.com/}}では大きなコミュニティが形成されており,多くのフラクタルファンが日夜議論を重ねている.ここから生れたフラクタル図形や描画手法も多い. 

また,\emph{デモシーン}({\it Demoscene})という文化もフラクタルレンダリングとも関連が深い.デモシーンは\emph{デモ}({\it Demo})と呼ばれるリアルタイムにCGグラフィクスや音楽を生成するプログラムを見せ合うコンピューターサブカルチャーの一つである.
美しいコンピュータグラフィクスを効率よく,さらに小さなサイズのプログラムで描画する手法が考えられており,数式で複雑な形状を描画できるようなフラクタルと相性が良い.デモシーンに関しては,Demoscene - The Art of the Algorithm\footnote{Demoscene - The Art of the Algorithms: \url{https://www.youtube.com/watch?v=iRkZcTg1JWU}}が詳しい.
著名なデモシーナーのInigo Quilez(iq)は自身のウェブページ\footnote{Inigo Quilez's webpage: \url{http://iquilezles.org/index.html}}に様々な記事をまとめている. 

\begin{figure}[htbp]
 \begin{center}
      \includegraphics[width=3in, height=3in, keepaspectratio]{../img/fractal/gasket.pdf}
    \caption{Sierpinski gasket}
    \label{fig:gasket}
 \end{center}
\end{figure}

\subsection{About Rendering}

この節では,近年よく使われるシェーダを用いたレンダリング手法について触れる. 

最近のデモ作品でよく用いられる作品制作手法の1つに{\it OpenGL Shading Language(GLSL)}や{\it High Level Shading Language(HLSL)}の\emph{フラグメントシェーダ}(Fragment Shader)を用いるものがある.フラグメントシェーダは本来,ポリゴンに陰影をつけるために使用されるが,スクリーンスペースに矩形のポリゴンをレンダリングすることで各ピクセルごとの処理を記述しレンダリングすることができる.GPUは並列演算性能に優れていることに加え,各ピクセルの演算は並列に行われるため高速である. 

シェーダを用いたレンダリングでは, 全てのオブジェクトがプログラムで表現される.
例えば,  通常三次元物体を表現する場合には, メッシュで表現される. 
頂点や法線の情報を読み出して描画する.
それに対して, アルゴリズムで形状を生成することになる.  
小さなプログラムサイズでリアルタイムにレンダリングするという目的に合致する. 

例えば, フラグメントシェーダで円を描くことを考えてみる. 
原点中心,半径$r$の円の距離関数は以下のようになる.
\begin{eqnarray*}
 f(z) = | length(z) - r |
\end{eqnarray*}
この関数は円周からの距離を返す.この距離から一定以下の距離をもつ時にピクセルを塗りつぶす処理を書くと円を描くことができる.複数の図形を描きたい時は,複数の距離関数を評価し,最も小さな距離を求めればよい.
また,距離関数は陰関数とその微分から近似的に導出することができる.与えられた点$x$に対して, 陰関数$f(x)$が表す零点集合への最短距離を返す距離関数は以下のように近似的に求めることができる. 
\begin{eqnarray*}
 DistanceFunction(x) \approx \frac{|f(x)|}{|\bigtriangledown f(x)|}
\end{eqnarray*}
この手法はDistance Estimationと呼ばれ,複雑な形状を高速かつ綺麗に描画することに役立つ.この形は基本形であるが, より精度の高い近似式を得るための工夫が施される場合がある. 詳しい導出はInigo Quilez氏の記述が詳しい\footnote{distance estimation: \url{http://iquilezles.org/www/articles/distance/distance.htm}}.

シェーダで三次元空間を描画する場合には\emph{レイトレーシング}({\it Ray tracing})が用いられる.レイトレーシングは視点からスクリーンへ\emph{レイ}({\it Ray})を飛ばし,その挙動をシミュレーションすることで物体を描画する手法である.レイと物体の交差点はレイの方向と位置から代数的に計算することができる.
また,\emph{レイマーチング}({\it Ray marching, Sphere tracing}とも呼ばれる)\cite{sphereTracing}という手法を用いるこfとで,距離関数を用いて任意の三次元形状との交差点を近似的に計算することができる.物体の形状を代数的に計算することが難しいフラクタル形状のレンダリングを行う際は,Distance Estimationにより近似的に物体との距離を得てレイマーチングを用いることで,高速に描画することができる.

GLSLによるレンダリングや,レイマーチングに関しては,doxas氏のwgld\footnote{wgld: \url{https://wgld.org/d/glsl/}}によくまとまっている.シェーダを用いたレンダリングはglslsandbox\footnote{glslsandbox: \url{http://glslsandbox.com/}}やShadertoy\footnote{Shadertoy: \url{https://www.shadertoy.com/}}といったウェブサービスで手軽に試すことができる.

GPUの計算性能をより汎用的な計算に用いるためのプラットフォームとしてCUDAやOpenCLが登場している.これらのプラットフォームを用いることでシェーダではできない複雑な処理を行うことができるようになり,GPUは機械学習などの分野にも活躍の場を広げた.このようにGPUによる演算を画像処理以外の汎用的な用途に用いる技術はGPGPUと呼ばれている.

\subsection{Escape-time Fractals}
{\it Escape-time Fractals}は複素平面上の点に対して,特定の漸化式を計算し,その点の軌道の収束,発散を見ることで描画されるフラクタルである.
Escape-time Fractalsで代表的なものが\emph{マンデルブロ集合}({\it Mandelbrot set})である. 
マンデルブロ集合の漸化式は以下のようになる.
\begin{eqnarray*}
 \begin{cases}
  z_{n+1} = z^2_{n} + c \\ z_0 = 0
 \end{cases}
\end{eqnarray*}
$z_n$が無限大に発散しないものをマンデルブロ集合と呼ぶ.図\ref{fig:mandelbrot}は黒い部分がマンデルブロ集合である.発散と判定されるまでの計算の回数によって色をつけた.
マンデルブロ集合はその単純な式から驚くほど豊富なバリエーションの図を見ることができる.
その他のEscape-time fractalには,\emph{ジュリア集合}({\it Julia set})(図\ref{fig:julia})や\emph{ファトゥー集合}({\it Fatou set})等がある.

各点における漸化式の計算はお互いに干渉しないので,そのままシェーダでの実装が可能であるが,先に述べたDistance Estimationを用いることで,エイリアシングノイズ等を避けてより精細に描画することができる.これには{\it Green Function}と呼ばれる陰関数を用いる.導出はiq氏の記事が詳しい\footnote{distance rendering for fractals: \url{http://iquilezles.org/www/articles/distancefractals/distancefractals.htm}}.Green Functionに関しては,マンデルブロ集合に関する詳細な議論となってしまうので,こちらの論文\cite{mandelbrot}を参考にされたい.

\begin{figure}[htbp]
 \begin{minipage}{0.49\hsize}
  \begin{center}
   \includegraphics[width=3in, height=3in, keepaspectratio]{../img/fractal/mandelbrot.pdf}
   \caption{Mandelbrot set}
   \label{fig:mandelbrot}
  \end{center}
 \end{minipage}
 \begin{minipage}{0.49\hsize}
     \includegraphics[width=3in, height=3in, keepaspectratio]{../img/fractal/julia.pdf}
   \caption{Julia set}
   \label{fig:julia}
 \end{minipage}
\end{figure}

\subsection{Distance Estimated 3D Fractals}

マンデルブロ集合のような複雑な3次元形状を持つフラクタルが求められていた.マンデルブロ集合を3次元に拡張する試みはさまざまに行われていたものの,あまり大きな成果はあげられていなかった.
例えば,ジュリア集合を4次元に拡張し,Distance Estimationを用いて描く方法が1989年には発表されている\cite{4djulia}.
しかし,その形状は単純でフラクタルコミュニティの満足のいくものではなかった.
2004年にKeenan Crane氏がGPUで実装した\footnote{Ray Tracing Quaternion Julia Sets on the GPU: \url{https://www.cs.cmu.edu/~kmcrane/Projects/QuaternionJulia/}}.

2003年に発表された阿原・荒木\cite{sphairahedra}による4次元クライン群の一種であるQuasi Fuchsian 3D Fractalsはある程度の複雑性を持った3次元フラクタルであり,コミュニティに大きな影響を与えたと言われている.このフラクタルに関しては後の章で触れる.その後,2009年にマンデルブロ集合をうまく三次元に拡張したMandelbulbの出現を皮切りに,次々と複雑な三次元フラクタルの式が発見された.これらもDistance Estimationとレイマーチングを用いることで効率よくレンダリングすることができる.これらのフラクタルはDistance Estimated 3D Fractalsと呼ばれることが多い.これらのフラクタルの歴史と実装についてはMikael Hvidtfeldt Christensen氏のSyntopiaの一連のブログポスト\footnote{Syntopia, Distance Estimated 3D Fractals: \url{http://blog.hvidtfeldts.net/index.php/2011/06/distance-estimated-3d-fractals-part-i/}}によくまとめられている.

これらフラクタルをレンダリングするためのソフトウェアはMandelbulb 3D\footnote{Mandelbulb 3D: \url{http://mandelbulb.com/2014/mandelbulb-3d-mb3d-fractal-rendering-software/}}やMandelbulber\footnote{Mandelbulber: \url{http://www.mandelbulber.com/}}が有名である.
Fragmentarium\footnote{Fragmentarium: \url{http://syntopia.github.io/Fragmentarium/}}はシェーダベースのグラフィクスを開発するための環境である.フラクタルのサンプルコードも豊富で学習に役立つ.
Fractal Lab\footnote{Fractal Lab: \url{http://hirnsohle.de/test/fractalLab/}}はブラウザ上でGLSLを用いてフラクタルをレンダリングすることができるWebアプリケーションである.この章で使われている図のいくつかはFractal Labを用いてレンダリングした.

\subsubsection{Mandelbulb}

マンデルブロ集合を三次元に拡張するにあたって一番の問題は漸化式の意味付けであった.
マンデルブロ集合では複素平面上で計算を行なうため, $z^2$は回転と拡縮, $c$の加算は点の平行移動を表わし, それによって軌道を見ることができる. 
三次元空間の点において, 同様の意味を表わす代数体系のようなものを考える必要があった. 
様々なアプローチが考案されたが, 最終的にD.White(twinbee)氏とP.Nylander氏が開発したMandelbulbが広く知られるようになった. 

White氏は球面座標上で漸化式を計算するアプローチを提案\footnote{\url{http://www.fractalforums.com/3d-fractal-generation/true-3d-mandlebrot-type-fractal/}}した.
図\ref{fig:mandelbulb2}は$z_{n+1} = z_n^2 + c$を用いてレンダリングした結果であるが,あまり複雑な形状は現れなかった.
しかし,その後Nylander氏が式を高次の積を扱えるように拡張した.
図\ref{fig:mandelbulb8}は$z_{n+1} = z_n^8 + c $という式でレンダリングされた形状であり,Mandelbulbとして広く知られている.White氏のWebページ\footnote{Skytopia: \url{http://www.skytopia.com/project/fractal/mandelbulb.html}}では開発の詳しい経緯がまとめられている.

Mandelbulbは三次元空間の点に対して漸化式を計算することで形状を得ることもできるが, 通常はマンデルブロ集合と同様に距離関数を導出し, レイマーチングを用いてレンダリングする.

\begin{figure}[h!tbp]
 \begin{subfigure}{0.49\hsize}
   \begin{center}
    \includegraphics[width=3in, height=3in, keepaspectratio]{../img/fractal/mandelbulb2.pdf}
    \caption{$z_{n + 1} = z_n^2 + c$}
    \label{fig:mandelbulb2}
   \end{center}
 \end{subfigure}
 \hspace*{\fill}
 \begin{subfigure}{0.49\hsize}
   \begin{center}
    \includegraphics[width=3in, height=3in, keepaspectratio]{../img/fractal/mandelbulb8.pdf}
    \caption{$z_{n+1} = z_n^8 + c$}
    \label{fig:mandelbulb8}
   \end{center}
 \end{subfigure}
 \caption{Mandelbulb rendered by Fractal Lab}
\end{figure}

\subsubsection{Mandelbox}
その後,2010年にTom Lowe氏によってMandelbox\footnote{Mandelbox, \url{https://sites.google.com/site/mandelbox/what-is-a-mandelbox}}が開発された.
Mandelboxも三次元空間におけるEscape-time fractalである. 
漸化式は$z_{n+1} = scale * spherefold(minR, boxfold(z_n)) + c$となる.
ここで, boxfoldは$x=\pm1, y=\pm1, z=\pm1$を頂点にもつ立方体の各面に対して, 点が外側にある場合にその面に関する反転を行なう操作である.
また, spherefoldは以下の式で定義される.
\begin{eqnarray*}
 spherefold(r, z) = \begin{cases}
                  \frac{z}{r^2} & 0 \le |z| < r \\
                  \frac{z}{|z|^2} & r \le |z|
                 \end{cases}
\end{eqnarray*}
基本は原点中心, 半径1の球に関する反転である. 
球の反転は, 中心を無限遠点へ変換するが, Escape-timeアルゴリズムでは中心付近のほとんどの点が発散と判定されてしまう. 
そこでspherefoldでは, 中心付近の点の反転をあらかじめ決めておいた最小の半径における反転で制限をかけている.

Distance Estimationでは漸化式の微分としてヤコビアンの積を用いる. 
式の各操作のたびに, ヤコビアンを累積していき, 
最終的な点のノルムをヤコビアンの積で割ることで距離を近似的に求めることができる.

\begin{figure}[h!tbp]
 \begin{subfigure}{0.49\hsize}
   \begin{center}
    \includegraphics[width=3in, height=3in, keepaspectratio]{../img/fractal/mandelbox.pdf}
    \caption{}
    \label{fig:mandelbox1}
   \end{center}
 \end{subfigure}
 \hspace*{\fill}
 \begin{subfigure}{0.49\hsize}
   \begin{center}
    \includegraphics[width=3in, height=3in, keepaspectratio]{../img/fractal/mandelbox2.pdf}
    \caption{}
    \label{fig:mandelbox2}
   \end{center}
 \end{subfigure}
 \caption{Mandelbox rendered by Fractal Lab}
 \label{fig:mandelbox}
\end{figure}


\subsubsection{Pseudo-Kleinian}

Mandelboxの登場以降は漸化式にMandelbulbやMandelboxを含む複数の式を用いた{\it Hybrid System}によって.今も様々なフラクタル作品が作られている.
興味深いHybrid Systemの1つに{\it pseudo-kleinian}がある.これは球の反転に近い操作であるSpherefoldと平行移動と同じような働きをするboxfoldの組み合わせによってクライン群のような形状が得られる式である.図はfragmentariumのサンプルスクリプトを用いてレンダリングされた.

\subsection{Coloring by Orbit Trap}

Escape-timeアルゴリズムを用いたフラクタルのレンダリングでは,主に{\it Orbit Trap}というカラーリングの手法が使われる.trapと呼ばれる点や直線などの図形を空間上に置いておき,trapと軌道上の点の位置関係から色を付ける.図\ref{fig:mandelbox}のMandelboxもOrbit Trapを用いて色が着けられている.
図\ref{fig:mandelbrotOrbit}はマンデルブロ集合の計算によって$z_n$が桃色の領域に入った点を白く塗った.
桃色の領域に到達するまでの軌道を描いたといえる. この方法を応用すると,図\ref{fig:mandelbrotBitmap}のように画像をtrapに用いることで,点の軌道上に画像を張り付けることも可能である.この手法は特に{\it Bitmap Orbit Trap}とも呼ばれる.

\begin{figure}[htbp]
 \begin{minipage}{0.49\hsize}
  \begin{center}
   \includegraphics[width=3in, height=3in, keepaspectratio]{../img/fractal/mandelbrot-orbit.pdf}
    \caption{Mandelbrot set}
    \label{fig:mandelbrotOrbit}
  \end{center}
 \end{minipage}
 \begin{minipage}{0.49\hsize}
  \begin{center}
   \includegraphics[width=3in, height=3in, keepaspectratio]{../img/fractal/mandelbrot-bitmap.pdf}
   \caption{Mandelbrot set}
   \label{fig:mandelbrotBitmap}
  \end{center}
 \end{minipage}
\end{figure}

\subsection{Other Fractals}

Iterated Function system(IFS)は点に様々な関数を適用して,その集積点をレンダリングするアルゴリズムである.シェルピンスキーのギャスケットなどもIFSでレンダリングすることができる.
IFSに関する書籍としてはFractals Everywhere\cite{fractalsEverywhere}が有名である.並列計算による高速化に関して以下の論文にまとめられている\cite{highPerformanceIFS}\cite{GPUIFS}.
IFSの一種にフラクタルフレーム(Fractal Frame)\cite{fractalFrame}と呼ばれるものがある.フラクタルフレームはIFSに用いる関数にメビウス変換などの非線形変換を用い,美しくレンダリングされるようにカラーリングが工夫されている.

他にも様々なアルゴリズムが存在するが,今回は後述するクライン群のレンダリングには関連してこないので触れない.マンデルブロ自身の著作\cite{fractal}では種々のフラクタルについて網羅的にまとめられている.


%#!uplatex main.tex

\section{Kleinian Groups}

クライン群について学習する際には書籍『インドラの真珠』\cite{komori-indra201303}
(原著:\cite{MumfordSeriesWright200204})を読むことを勧める.
インドラの真珠は数学者以外にもクライン群の魅力を伝えるために書かれた.
中には研究者レベルの高度な内容も扱う部分もあるが,高校レベルの数学と簡単なプロ
グラムを組む能力があれば,自分で図をレンダリングしながら読み進めることができる.
この章では,クライン群に関する用語を確認した後にレンダリング手法と『イン
ドラの真珠』では言及のないクライン群の話題をまとめる.

\subsection{Conjugation}

クライン群を考えるうえで, 我々は様々な変換を扱う.
変換に関する操作において重要となるのが, \emph{共役}(\textit{Conjugation})
である.
共役を用いることで, ある変換をその性質を変えずに異なる座標において作
用させることができ, 複雑な変換をよりわかりやすい座標における変換としてみ
ることができる.

例えば, 図に示された矩形の中心における回転行列を求めたいとする.
原点を中心とした回転行列はよく知られているため, これを$T$とおく.
矩形の中心までの平行移動を表わす変換を$S$とすると, 求めたい行列$\hat{T}$
は$\hat{T} = STS^{-1}$で求めることができる.
これは, 矩形を一度原点まで平行移動で移し, 原点中心の回転を作用させ, 元の
場所まで戻すという操作に等しい.
ここで, 変換$\hat{T}$を$T$の共役, $S$を共役変換と呼ぶ.
$\hat{T}$はこの操作を行なった後でも回転を表わす変換であることに変わりはなく,
$\hat{T}$は回転と共役であるという.

\subsection{Group}

クライン群の\emph{群}(\textit{Group})という語は変換群のことを指す.
ここでは変換の集まりだと考えるとわかりやすい.
例えば,ある変換$f(z)$と$g(z)$,そしてそれらの逆変換の4つの変換を考える
とき,これらの組み合わせを合成することで得られる変換の集合を群とよぶ.
また,ここで群を構成した4つの変換を\emph{生成元}(\textit{Generator})とよ
ぶ.
この群は2つの変換から生成されるので\emph{二元生成群}というようによばれることも
ある.
クライン群は\emph{メビウス変換}(M\"obius Transformations)を生成元にもつ
群のなかで離散性という性質を持つものをいう.

ここから,それぞれの変換を小文字のアルファベットで表し,その逆変換を大文字で綴ることにする.
$f(z)$を$a$,$g(z)$を$b$とおくと逆変換は$A$,$B$となる.
変換の合成はアルファベットを並べて書くことで表す.
例えば合成変換$f(f(g(f(p))))$は$aaba$という語で表される.
また,上付文字はその変換の無限列とする.
例えば,$\overline{a}$は,$aaaaa...$というような変換$a$の無限列となる.
この無限列に対応する合成変換はどの点に作用させてもその点を変換$a$の
\emph{固定点}に移す.
固定点とは,$f(z) = z$となるような,変換を使って動かすことのできない点である.
 \begin{align*}
  Generators =
   \begin{cases}
    a \colon f(z) \\
    A \colon f^{-1}(z) \\
    b \colon g(z) \\
    B \colon g^{-1}(z)
   \end{cases}
  \quad
  Group =
   \begin{cases}
    aab\\
    abbbaB \\
    ABBAbba \\
    baBBAbaaa \\
    ...
   \end{cases}
 \end{align*}

\subsection{M\"obius Transformations}

二次元のメビウス変換は複素平面上に, ひとつの無限遠点を付け加えた拡張複素
平面$\hat{\mathbb{C}}$上で定義される.
これはリーマン球面とも呼ばれ, 拡張複素平面全体を球面として扱うことができ
る.
メビウス変換は一次分数変換とも呼ばれ,以下のように表わされる.
\begin{align*}
 a, b, c, d\in \hat{\mathbb{C}} \quad f(z) = \frac{az + b}{cz + d}
\end{align*}
メビウス変換は,等角性や円円対応といった性質をもつ.
我々になじみの深い平行移動,拡縮や回転といった変換もメビウス変換に含まれる.

メビウス変換は,以下のように係数を2×2の複素数行列として扱うことで,
代数として操作することができる.
例えば合成変換を行列の積として表すことができる.
\begin{align*}
  m = \left(
 \begin{array}{ccc}
  a & b \\
  c & d
 \end{array}
 \right)
\end{align*}

メビウス変換はおおまかに\emph{放物型}(\textit{Parabolic}), \emph{斜航
型}(\textit{Loxodromic}), \emph{楕円型}(\textit{Elliptic})の三種類に分
類される.
斜航型変換と楕円型変換は2つ,放物型変換は1つの固定点をもつ.
斜航型変換は単純な回転を除いた複素数による拡縮と共役である.
また斜航型変換の中でも,正の実数による拡縮と共役の
変換を\emph{双曲型}(\textit{Hyperbolic})と呼ぶ場合もある.
放物型は平行移動と共役で, 楕円型変換は回転と共役である.

クライン群の描画において,メビウス変換の分類は重要である.
なぜならば,楕円型変換を含む群の多くはクライン群にはなりえない.
また,放物型変換は固定点への収束が遅いという特徴をもつ.
これは後述する極限集合の描画の際に考慮することになる.

\subsection{Circle Inversion}

円は複素平面に作用するメビウス変換を考えるうえで重要な図形である.
先に述べたように,メビウス変換は円を円に移す円円対応という性質
をもつ.
ここでは円に関する反転(\textit{Circle Inversion})という操作をみる.
これは円弧を軸とみた鏡映変換であり,無限遠点は円の中心と入れ替えられる.

点aを中心とする半径rの円に関する反転を$I(z)$とすると式は以下のようになる.
\begin{align*}
I(z) =
 \begin{cases}
  \frac{r^2}{\overline{z - a}} + a \quad &for \quad z \in \mathbb{C} - \{a\} \\
  a \quad &for \quad z = \infty\\
  \infty \quad &for \quad z = a
 \end{cases}
\end{align*}
また,半径が無限大の円は直線とすることで,その円に関する反転は直線
に関する反転と考えることができる.

図\ref{fig:circleInversion}は黒い円に関する反転によって移される図形を可
視化したものである.
緑色の矩形を観察すると,変換の前後で大きく形を歪められているが, その角は直
角が保たれている.
直線は無限遠点が円の中心へと移るため, 反転円の中心を通る円となる.
また, 桃色の円は操作の前後で円が保たれている.

円に関する反転は複素平面の向き付けを逆にしてしまうため, メビウス変換では
ないが, 2つの反転を合成することで双曲型のメビウス変換となる.
偶数個の円の反転を組み合わせることで, その他の種類のメビウス変換を構成することが
できる.
円に関する反転はメビウス変換を構成する変換の最小単位ともいえる.

また, 円を球に拡張することで球に関する反転を考えることができる.
これを用いて三次元空間に作用するメビウス変換を構成することもできる.

\begin{figure}[htbp]
 \begin{center}
      \includegraphics[width=3in, height=3in, keepaspectratio]{../img/klein/circleInversion.pdf}
    \caption{Circle Inversion}
    \label{fig:circleInversion}
 \end{center}
\end{figure}

\subsection{Stereographic Projection}

拡張複素平面はリーマン球面という名の通り, 球面で表すことで全
体をよく見ることができる.
複素平面を球面へと写像するためには, \emph{立体射影}({\it
stereographic projection})と呼ばれる方法がよく使われる.
地図の制作のため, 地球のような球面を平面に投影する方法が様々に考えられて
きたが, それらの手法の中でも立体射影は角度と円を保つため, メビウス変換と相性が良い.

立体射影は以下のように行える.
ある球面上の点を$P$とおくと, $P$を立体射影で移した複素平面上の点は球の北極
から$P$へと引いた直線と複素平面との交点となる.
図\ref{fig:stereoProject}は立体射影を図示したものである.
赤の円は単位円,青の直線はx軸を表している.

\begin{figure}[htbp]
 \center
 \includegraphics[width=3in, height=3in, keepaspectratio]{../img/klein/stereoProject.pdf}
 \caption{Stereographic Projection}
 \label{fig:stereoProject}
\end{figure}


Saul Schleimer,Henry Segermanはメビウス変換を利用して全天球画像(上下左
右360度パノラマ写真)を編集することを考案した\cite{bridges2016-15}.
彼らは\emph{ドロステ効果}(\textit{Droste-effect})とよばれる効果をもつ再
帰的な画像をメビウス変換を用いてつくる手法を提案している
\footnote{Spherical Droste video:
\url{https://www.youtube.com/watch?v=qvh-EAipIUk}}.

全天球画像は,その名の通り球面に張り付けられた画像であるので,メビウス変
換で角度を保ったまま編集を加えることができる.
また全天球画像は正距円筒図法(Equirectangular map)で扱われることも多い.
この図法は縦軸に緯度,横軸に経度をとり,球面をそのまま写像するものである.
メビウス変換で球面の画像を編集したものを正距円筒図法で写像すると興味深い
効果をみることができる.

図\ref{fig:sphericalRotation},\ref{fig:sphericalTranslation},
\ref{fig:sphericalScaling}は図\ref{fig:sphericalStandard}の格子模様
に対して,簡単なメビウス変換を作用させたものである.
複素平面,リーマン球面,その正距円筒図,そして全天球画像の一部を注視した
図を示した.

\begin{figure}[htbp]
 \begin{minipage}{0.5\hsize}
  \center
  \includegraphics[width=3in, height=3in, keepaspectratio]{../img/klein/spherical.pdf}
  \subcaption{Standard}
  \label{fig:sphericalStandard}
 \end{minipage}
 \begin{minipage}{0.5\hsize}
  \center
  \includegraphics[width=3in, height=3in, keepaspectratio]{../img/klein/sphericalRotation.pdf}
  \subcaption{Rotation}
  \label{fig:sphericalRotation}
 \end{minipage}
 \begin{minipage}{0.5\hsize}
  \center
  \includegraphics[width=3in, height=3in, keepaspectratio]{../img/klein/sphericalTranslation.pdf}
  \subcaption{Translation}
  \label{fig:sphericalTranslation}
 \end{minipage}
 \begin{minipage}{0.5\hsize}
  \center
  \includegraphics[width=3in, height=3in, keepaspectratio]{../img/klein/sphericalScaling.pdf}
  \subcaption{Scaling}
  \label{fig:sphericalScaling}
 \end{minipage}
 \caption{Spherical images}
 \label{fig:spherical}
\end{figure}

\subsection{Vizualization with Graph Traversal Approach}

群を構成する変換の全組み合わせは,変換の語で構成される木構造で表現するこ
とができる.
図\ref{fig:wordTree}にその木構造のひとつを示した.
このような木構造はケーリーグラフ(Cayley graph)と呼ばれる.
この図では変換$a, b$とそれらの逆変換である4種類の変換の合成の組合せを表
わしている.
例えば,$a$というラベルがふられたノードでは,外側へ三本のエッジが
張られ,$ab, aa, aB$という合成変換を表すノードへとつながっている.
また,$aA$は逆変換を合成することになるので,このグラフ上では逆戻りを表し
ている.

このグラフを探索することで,群を構成する合成変換を列挙していくことができ
る.この節では,ケーリーグラフを探索することで得られる合成変換を用いて群の作
用を可視化する方法をみる.

群を構成するためには複数のメビウス変換が必要であるが,メビウス変換は4つ
の複素数で構成されているため,自由度が高く,そのまま群を考えることは難し
い.
そのため,群の生成元を得るための「レシピ」がよく用いられる.

例えば, 『インドラの真珠』では「おばあちゃんのレシピ」とよぶレシピが紹介
されている.
このレシピは, 二つの複素数$t_a, t_b$をパラメータにとり,以下のような操作
で2つの生成元を得ることができる.
\begin{enumerate}
 \item 二つの複素数,$t_z, t_b$を選ぶ.
 \item  二次方程式
        $x^2 - t_a t_b x + t_a^2 + t_b^2 = 0 \text{の一方の解}x\text{を
        選び, }t_{ab}= x \text{とする. }$
 \item $z_0$を次のように定義する.
       \begin{align*}
        z_0 = \frac{(t_{ab} -2)t_b}{t_b t_{ab} - 2 t_a + 2it_{ab}}
       \end{align*}
 \item 生成元の行列は次のようになる.
        \begin{align*}
       &a = \left(
      \begin{array}{ccc}
       \frac{t_a}{2} & \frac{t_a t_{ab} - 2 t_b + 4i}{(2 t_{ab} + 4)z_0} \\
       \frac{(t_a t_{ab} - 2 t_b -4i)z_0}{2 t_{ab} - 4} & \frac{t_a}{2}
      \end{array}
     \right)\\
 &b = \left(
      \begin{array}{ccc}
       \frac{t_b - 2i}{2} & \frac{t_b}{2} \\
       \frac{t_b}{2} & \frac{t_b + 2i}{2}
      \end{array}
     \right)
        \end{align*}
\end{enumerate}
これによって8つの複素数のパラメータをもつ二元生成群を2つの複素数から生
成することができ,考えやすくなる.
また,「おばあちゃんのレシピ」は群を可視化した際に綺麗な図がレンダリング
されるように共役によって調整されている.

\begin{figure}[htbp]
 \center
 \includegraphics[width=3in, height=3in, keepaspectratio]{../img/klein/wordTree.pdf}
 \caption{Word Tree}
 \label{fig:wordTree}
\end{figure}

\subsubsection{The Orbit of Transformations}

まず, 円の反転で構成される群を可視化してみる.
円の反転は変換自体を円という図形で表現できるため, その作用を理解しやすい.
図\ref{fig:schottky}には4つの大きな円の内側にたくさんの円が入れ子状に描
かれている.
これは反転円の\emph{軌道}とよばれる.また,円列の極限を\emph{極限集合}とよぶ.

\begin{enumerate}
 \item 図\ref{fig:level0}に示した4つの大きな円を反転円とよぶ.
       これらの円に関する反転を生成元とする群の軌道を描くことを考える.
 \item それぞれの反転円に関する反転は\ref{fig:level0inv}のように,自分以外の反
       転円を自分の内側に移す.
 \item この操作を4つの反転円でおこなうと, それぞれの円の内側には3つの円が移
       され,図\ref{fig:level1}のように合せて12個の小円ができる.
 \item 次に,新たにできた小円に対してもその小円が属している反転円以外の反転を作
       用させる図\ref{fig:level1inv}
 \item 小円の下に新たな小円ができる.図\ref{fig:level2}
 \item このことを繰り返すと,最終的に図\ref{fig:levelMax}を得ることができる.
\end{enumerate}

\begin{figure}[htbp]
 \begin{minipage}{0.33\hsize}
  \begin{center}
   \includegraphics[width=2in, height=2in, keepaspectratio]{../img/klein/orbit/level0.pdf}
   \subcaption{orbit}
   \label{fig:level0}
  \end{center}
 \end{minipage}
 \begin{minipage}{0.33\hsize}
  \begin{center}
   \includegraphics[width=2in, height=2in, keepaspectratio]{../img/klein/orbit/level0inv.pdf}
   \subcaption{orbit}
   \label{fig:level0inv}
  \end{center}
 \end{minipage}
 \begin{minipage}{0.33\hsize}
  \begin{center}
   \includegraphics[width=2in, height=2in, keepaspectratio]{../img/klein/orbit/level1.pdf}
   \subcaption{orbit}
   \label{fig:level1}
  \end{center}
 \end{minipage}
 \begin{minipage}{0.33\hsize}
  \begin{center}
   \includegraphics[width=2in, height=2in, keepaspectratio]{../img/klein/orbit/level1inv.pdf}
   \subcaption{orbit}
   \label{fig:level1inv}
  \end{center}
 \end{minipage}
 \begin{minipage}{0.33\hsize}
  \begin{center}
   \includegraphics[width=2in, height=2in, keepaspectratio]{../img/klein/orbit/level2.pdf}
   \subcaption{orbit}
   \label{fig:level2}
  \end{center}
 \end{minipage}
 \begin{minipage}{0.33\hsize}
  \begin{center}
   \includegraphics[width=2in, height=2in, keepaspectratio]{../img/klein/orbit/levelMax.pdf}
   \subcaption{level max}
   \label{fig:levelMax}
  \end{center}
 \end{minipage}
 \caption{Process}
\end{figure}

図\ref{fig:schottky}では反転円の軌道を描いが, 同様にして異なる図形の
軌道を描くこともできる.
例えば, 図\ref{fig:orbit}は中央に置いた猫の画像を4つ反転円による反転で移
した軌道を描いた.

図\ref{fig:apr}はP.Nylander氏による作品\footnote{bugman123.com Fractals:
\url{http://bugman123.com/Fractals/index.html}}を参考に描画した.
中央右側の蝶をおばあちゃんのレシピで得られた群の生成元で移した.
虹色の曲線は極限集合である.
どちらの図も図形の軌道は最終的に極限集合へと収束していくことがわかる.

このアルゴリズムは変換で構成されるケーリーグラフを幅優先探索で探索するこ
とに等しい.
例えば, 図\ref{fig:wordTree}の木を時計回りに探索すると$a, b, A, B,
aB, aa, ab, ba, bb, ...$の順に合成された変換を得ることができる.
あらかじめ決められた深さまで調べたら, 得られたすべての合成変換を大本とな
る図形(図\ref{fig:orbit}においては中央の猫)に対して作用させることで, 軌
道を計算することができる.
このようにして描かれた図形の軌道を軌道をみることで,おおまかに生成元の作
用を観察することができる.

\begin{figure}[htbp]
 \begin{minipage}{0.49\hsize}
  \begin{center}
   \includegraphics[width=3in, height=3in, keepaspectratio]{../img/klein/schottkyCircles.pdf}
   \caption{Schottky Circles}
   \label{fig:schottky}
  \end{center}
 \end{minipage}
 \begin{minipage}{0.49\hsize}
  \begin{center}
   \includegraphics[width=3in, height=3in, keepaspectratio]{../img/klein/circleOrbit.pdf}
   \caption{orbit}
   \label{fig:orbit}
  \end{center}
 \end{minipage}
\end{figure}

\subsubsection{Rendering the Limit Set}

軌道の元となる図の位置によって得られる図が変わることや,幅優先探索
は計算量が大きいことから,より複雑なメビウス変換群を可視化する際には, 極
限集合を直接描く方法が使われる.

極限集合はケーリーグラフにおいて, 無限の深さのノードの合成変換に対応している.
しかし, 現実にそれを求めることは不可能なので, 群の生成元の固定点をある程
度の長さをもった変換の語で移すことで求める.

例えば, 変換$a$の固定点は任意の点に$a$を無限回作用させた点, つまり
$\overline{a}$を作用させた点と考えることができ,その点は極限集合に含まれる.
そのため,合成変換$abbAAb\overline{a}$で表される点は$a$の固定点を合成変換
$abbAAb$で移すことで求めることができ,これも極限集合に含まれる.
この時, $abbAAb$を接頭語と呼ぶ.

このように,ケーリーグラフを探索して接頭語を求め,それを生成元の固定点に
作用させることで極限集合を得ることができる.
接頭語を求める際には,深さ優先探索が用いられる.
図\ref{fig:wordTree}の語の木を時計回りに長さ3までの語を探索すると,$ abA, abb, aba, aab ...$という順で接頭語が得られる.
これらの語に, 逆変換をかけないように, 生成元の固定点を与えると極限点を求めることができる.

例えば$aBA$を用いて$\overline{A}, \overline{b}, \overline{B}$の3つの固定点を変換すると互いに近い3つの極限点が得られる.
$aab\overline{a}$と$aaa\overline{a}$のように隣同士の接頭語を持つ極限点や
$aaa\overline{a}$と$aaa\overline{b}$のような同じ接頭語を持つ極限点同士は
近い位置にあることが証明されている.
そのため, 得られた点同士を順番に結ぶと図\ref{fig:limitset}のようなフラクタル
構造を持つ曲線を得ることができる.
また,得られた点同士の距離によって,より深い語を探索するかどうかを判断し,
そこで計算を打ち切ることで探索を高速化することができる.

\begin{figure}[htbp]
 \begin{minipage}{0.5\hsize}
  \center
  \includegraphics[ height=2.5in, keepaspectratio]{../img/klein/limitset/limit1.pdf}
  \subcaption{}
  \label{fig:limit1}
 \end{minipage}
 \begin{minipage}{0.5\hsize}
  \center
  \includegraphics[ height=2.5in, keepaspectratio]{../img/klein/limitset/limit2.pdf}
  \subcaption{}
  \label{fig:limit2}
 \end{minipage}
 \begin{minipage}{0.5\hsize}
  \center
  \includegraphics[ height=2.5in, keepaspectratio]{../img/klein/limitset/limit3.pdf}
  \subcaption{}
  \label{fig:limit3}
 \end{minipage}
 \begin{minipage}{0.5\hsize}
  \center
  \includegraphics[ height=2.5in, keepaspectratio]{../img/klein/limitset/limit4.pdf}
  \subcaption{}
  \label{fig:limit4}
 \end{minipage}
 \caption{The limit set of Kleinian groups}
 \label{fig:limitset}
\end{figure}

ただし, 綺麗に描画するには様々な工夫が必要となる.
まず, 放物型変換は固定点への収束に時間がかかるため, 放物型変換の固定点付近
をうまく描画するためには,グラフを深く探索しなければならない.
図\ref{fig:limit1}の右端をよく見ると直線で結ばれていることがわかる.
ここには放物型変換の固定点が存在しており, 収束が遅いため, この近辺の極限
点を得るためには, より長い接頭語が必要となる.
『インドラの真珠』では放物型の生成元を特別視して扱う「特殊語アルゴリズム」と
よばれるアルゴリズムでこの問題に対処する.

また, ある変換とその逆変換を合成してしまうと恒等変換になってしまうことは
自明であるが,特定の変換の組み合わせが恒等変換を生み出してしまうことがある.
例えば,語$a$を$f(z) = z + 1$,語$b$を$g(z) = z + i$とするとき,
abABという語が表す合成変換は恒等変換となってしまう.
このような語を$\dot{も}\dot{た}\dot{な}\dot{い}$群を\emph{自由群}と呼ぶ.
こういった語が存在すると,ケーリーグラフの探索の効率が悪くなってしまうた
め,効率よく描画するためにはこういった語を取り除く必要がある.
このことに対処するためには有限オートマトンが使われる.
詳しくは『インドラの真珠』を参照されたい.
変換群に関する有限オートマトンを取り扱った書籍には『Word Processing In
Groups』\cite{Epstein:1992:wordProcessing}がある.

先に述べたように,メビウス変換群の離散部分群がクライン群となる.
そのため群の中には描画すると極限集合が収束せず,図\ref{fig:non-discrete}
のように乱れるものがある.
このような群は\emph{非離散群}と呼ばれ,クライン群ではない.
メビウス変換群のなかで,どのようなパラメータを持つ群が離散的であるか,す
なわちクライン群となるかを調べることがクライン群の研究テーマの一つとなる.
離散,非離散を判定するために有用な定理として\emph{ポアンカレの多面体定理}
や\emph{ヨルゲンセン不等式定理}とよばれるものが存在する.

離散と非離散のパラメータ領域を描画した図は\emph{スライス}と呼ばれ,
スライスの離散と非離散の境界もしばしばフラクタル形状となる.
また,その境界付近のパラメータによる群を描画するとしばしば極限集合が興味
深い形になる.
山下靖のウェブページ\footnote{Discreteness Locus:
\url{http://vivaldi.ics.nara-wu.ac.jp/~yamasita/Slice/}}では様々な
スライスの画像を見ることができる.

\begin{figure}[htbp]
 \begin{minipage}{0.5\hsize}
  \center
  \includegraphics[ height=2.5in, keepaspectratio]{../img/klein/limitset/non-discrete1.pdf}
  \subcaption{}
 \end{minipage}
 \begin{minipage}{0.5\hsize}
  \center
  \includegraphics[ height=2.5in, keepaspectratio]{../img/klein/limitset/non-discrete2.pdf}
  \subcaption{}
 \end{minipage}
 \caption{The limit set of non-discrete groups}
 \label{fig:non-discrete}
\end{figure}

\subsubsection{Faults of Graph Traversal Approach}

ケーリーグラフを探索する方法にはいくつかの欠点がある.
まず, 生成元の数を増やしたり,探索を深くしたりするごとに指数オーダーで計
算量が増えてしまうことが挙げられる.
さらに木構造の探索はそもそも並列化に向かないため, 高速化が難しい.
また, 部分的に拡大したい場合に無駄な部分を計算しないといった処理にも手間
がかかる.
このような欠点を避けるため,木構造の探索に頼らない方法がいくつか考案され
ている.

しかしながら, 最新のOpenCLやCUDAといった並列計算プラットフォームでは
Dynamic Parallelismという機能を用いることで,木構造探索の並列計算を効率
よく行うことが可能である.
ただし, 最新のハードウェアに依存した機能のため, どの計算機環境でも使える
とはいえない.

\subsection{Iterated Inversion System}

ケーリーグラフの探索に依らないアルゴリズムを考案するにあたり,
筆者らは, 1章でみたシェーダを用いたフラクタルのレンダリングアルゴリズム
に着想を参考にした.
そして,円や球の反転で構成される群の軌道を高速に描画するためのアル
ゴリズム, \textit{Iterated Inversion System (IIS)}\cite{bridges2016-367}
を開発した.
円や球の座標と半径を直接計算するこれまでのアプローチに対して,このアルゴ
リズムは任意の点が属する円や球の深さを特定する.
これはスクリーンスペースのピクセルそれぞれに対して独立に計算することがで
きるので,シェーダによる並列計算,描画を行うことができる.

\subsubsection{The Algorithm}

アルゴリズムは非常にシンプルである.
図\ref{fig:iis-orb}のような4つの反転円の軌道を描くことを考える.
ここで,すべての円の外側である黒い領域を\emph{基本領
域}(\textit{fundamental domain})とよぶ.

IISは平面状の全ての点に対して計算を行う.
ある点がいずれかの反転円に属している時にその反転円に関する反転を作用させ
る.
操作を反転後の点が基本領域に移されるまで繰り返す.
最終的に行なった反転の回数が, その点が属している円盤の深さとなる.
図\ref{fig:iis-orb}では青い点を反転を繰り返して移した軌道を描いている.
この点は二回の反転によって基本領域に到達したので, 下から二番目の円に属し
ている.

実際にこのアルゴリズムを用いてすべての点を基本領域へと移動させるためには,
無限回の反転が必要である. そのため, あらかじめ最大の反転回数を決めておく.
それぞれの計算は最大の反転回数と,円の数の多項式時間で描画することができる.
また,画面に入っている部分のみを計算するので,無駄な部分を計算する必要が
ない.

\begin{figure}[htbp]
 \center
 \includegraphics[ height=1.5in, keepaspectratio]{../img/klein/orb.pdf}
 \caption{Orbit}
 \label{fig:iis-orb}
\end{figure}

このアルゴリズムをより一般化した擬似コードをアルゴリズム\ref{alg:iis2d}に示す.
後に我々は単純な円の反転以外の生成元を導入する.
そこで,基本領域上にある点に対しては恒等変換を, その他の点に対しては反転,
もしくは反転の合成を作用させる写像$G$を考える.

 \begin{algorithm}
  \caption{Iterated Inversion System (IIS)}
  \label{alg:iis2d}
  \begin{algorithmic}
   \REQUIRE count $= 0$ and coordinates $=$ position determined by
   pixel
   \FOR{$i=0$ to MAX\_INVERSION}
   \STATE inFundamentalDomain $\leftarrow$ \TRUE
   \FOR{ each map $G$ in Maps }
   \IF{$G$ is available to coordinates}
   \STATE coordinates $\leftarrow$ $G(\text{coordinates})$
   \STATE INCREMENT count
   \STATE inFundamentalDomain $\leftarrow$ \FALSE
   \ENDIF
   \ENDFOR
   \IF {inFundamentalDomain}
   \STATE BREAK for
   \ENDIF
   \ENDFOR
   \STATE RETURN count
  \end{algorithmic}
 \end{algorithm}

\subsubsection{3D Extension}

このアルゴリズムは三次元空間における球の反転に対しても用いることができる.
空間上の点に対しても点が属している球の深さを計算することができるが,球に
よる反転の軌道は球の内側に入り込むため,二次元の場合と同様に描画するには
手間がかかる.
そこで我々は軌道の元となる球を用意し, その球を球の反転の組み合わせで
移した軌道を描画する.
図\ref{fig:3baseSphereGen}に描かれている3つの緑の球をすべての灰色の球の
反転の組み合わせで移した軌道が図\ref{fig:3baseSphereOrb}のようになる.
ここで,灰色の球を\emph{反転球},緑色の球を\emph{基本球}とよぶ.

\begin{figure}[htbp]
 \begin{minipage}{0.5\hsize}
  \center
  \includegraphics[ height=1.7in, keepaspectratio]{../img/klein/3baseGen.pdf}
  \subcaption{Generator}
  \label{fig:3baseSphereGen}
 \end{minipage}
 \begin{minipage}{0.5\hsize}
  \center
  \includegraphics[ height=1.7in, keepaspectratio]{../img/klein/3baseOrb.pdf}
  \subcaption{Orbit}
  \label{fig:3baseSphereOrb}
 \end{minipage}
 \caption{The orbit of 3 base spheres}
 \label{fig:3baseSphere}
\end{figure}

これはレイマーチングとDistance Estimationを用いることで高速に描画するこ
とができる.レイマーチングに用いる距離関数を導出するため,ここから図
\ref{fig:simple3diis}に示されるような,6つの反転球と1つの基本球による
軌道を考える.
簡単のために, この図のXY平面での断面を図\ref{fig:xySlice}に示した.
橙色の円列はショットキー球の軌道, 水色の円が基本球の軌道である.
緑色の点$P2$はレイの先端であり,これに最も近い球は$S2$である.
黄色の点$P1$は$P2$の属する反転球で$P2$を反転した像である.
また,$S1$は$S2$をその反転球で移した像であり,基本球でもある.

距離関数は$P2$から$S2$までの距離を返す必要がある.
しかし, $S2$の位置と半径は分からない.
そこで我々は球の反転のヤコビアンを用いて, 点$P1$と基本球である$S1$から求
める距離を逆算する.
求める距離を$d$,$S1$の半径と中心をそれぞれ$S1r, S1c$とすると,次の式で
近似することができる.
\begin{align*}
 d \approx (distance(P1, S1c) - S1r) / Jacobian
\end{align*}
これは1章でみたDistance Estimationと同様のアプローチである.

また,反転球の中心を$S$その半径を$R$,反転をさせる前の点を$P$とおくと
球の反転のヤコビアンは次のように計算できる.
\begin{align*}
 Jacobian = R^2 / distance(P, S)^2 
\end{align*}
半径が無限大の球の反転のヤコビアンは1であることに注意する.

実際の距離関数では,レイの先端にIIS作用させる過程で球の反転を行う毎にそ
のヤコビアンをかけ合わせていく.
最終的に基本領域上に移された点から基本球への距離をヤコビアンの積で割ること
で近似された距離を求めることができる.
ただし,球から点が離れすぎてしまうと誤差が大きくなってしまうことに注意が
必要である.

また,もう一つ考慮すべき事がある.
レイの先端が図\ref{fig:xySlice}における極限集合より外側にあるとき, その
点は球の反転によって反転球の外側に移されてしまう.
そのため距離関数は実際の最短距離よりも大きな距離を返してしまい,
レイはオブジェクトを突きぬけてしまう.
このことは,図\ref{fig:artifact}に描かれるような乱れを生み出す.
軌道の前面が正しくレンダリングされていないことがわかる.
この不具合を避けるため, 算出された距離を縮小する方法をとる.
レイマーチングのステップ数は増えてしまうが,正しく描画することができる.
距離の縮小率は球の大きさによって実験的に決めた.

レイマーチングで描画したことにより得られる副次的な効果として,
基本球の大きさを反転球と重なりあうように大きくすることで, 反転球の軌道の
形状を変形することができる.
図\ref{fig:limitSetOnSphere}は基本球の半径を球面が極限集合と被るような大
きさにしたときのものである.
球の上に極限集合をみることができる..
図はそこから基本球の大きさをさらに大きくした図である.数学的な意味を読み
取ることはできないが,興味深い形状である.

ここでみた処理を一般化した距離関数の擬似コードをアルゴリズム
\ref{alg:iis3d}に示した.

\begin{figure}[htbp]
  \begin{minipage}[]{0.49\hsize}
   \center
   \includegraphics[ height=3in, keepaspectratio]{../img/klein/simpleGen.pdf}
   \subcaption{Generator}
   \label{fig:simpleGen3d}
  \end{minipage}
  \begin{minipage}[]{0.49\hsize}
   \center
   \includegraphics[ height=3in, keepaspectratio]{../img/klein/simpleOrbit.pdf}
   \subcaption{Orbit}
   \label{fig:simpleOrb3d}
  \end{minipage}
  \caption{The orbit of the green sphere}
  \label{fig:simple3diis}
\end{figure}

\begin{figure}[htbp]
 \begin{minipage}{0.5\hsize}
  \center
  \includegraphics[ height=1.7in, keepaspectratio]{../img/klein/xySlice.pdf}
  \caption{XY-slice image of Figure \ref{fig:simple3diis}}
  \label{fig:xySlice}
 \end{minipage}
 \begin{minipage}{0.5\hsize}
  \center
  \includegraphics[ height=1.7in, keepaspectratio]{../img/klein/artifact.pdf}
  \caption{The Artifact}
  \label{fig:artifact}
 \end{minipage}
\end{figure}

\begin{figure}[htbp]
  \center
  \includegraphics[ height=1.5in, keepaspectratio]{../img/klein/3dLimitSet.pdf}
  \caption{The Limit set on the sphere}
  \label{fig:limitSetOnSphere}
\end{figure}

\begin{algorithm}
 \caption{Distance function}
 \label{alg:iis3d}
 \begin{algorithmic}
  \REQUIRE count $= 0$, $d$ = MAX\_DISTANCE, $dr = 1.0$, and coordinates $=$ tipping
  point of the ray
  \FOR{$i=0$ to MAX\_INVERSION}
  \STATE inFundamentalDomain $\leftarrow$ \TRUE
  \FOR{ each Map $G$ in Maps}
  \IF{$G$ is available to coordinates}
  \STATE coordinates $\leftarrow$ $G(\text{coordinates})$
  \STATE $dr \leftarrow dr * $ (Jacobian of $G(\text{coordinates})$)
  \STATE INCREMENT count
  \STATE inFundamentalDomain $\leftarrow$ \FALSE
  \ENDIF
  \ENDFOR
  \IF {isInFundamentalDomain}
  \STATE BREAK for
  \ENDIF
  \ENDFOR
  \FOR{ each BaseSphere $S$ in BaseSpheres}
  \STATE $d \leftarrow$ min($d$, scalingFactor * (distance(coordinates, $S$.center) $-$
  $S$.radius) $/$ (absolute value of $dr$))
  \ENDFOR
  \RETURN $d$
 \end{algorithmic}
\end{algorithm}

\subsection{Geometrical Representation of M\"obius Transformations}

多くの場合メビウス変換は行列を用いて代数的に扱う.
しかし,行列表現でメビウス変換を操作することは直観が得にくい.
行列による変換の表現から変換の作用を推測することは非常に難しい.

また,三次元空間に作用するメビウス変換を考えることで,高次元のクライン群
を構成することができる.
高次元のクライン群の極限集合は三次元空間でのねじれをもつ非常に興味深い形
状をもつ.
しかし,このような群を構成するメビウス変換は,
四元数行列で表現されるため,さらに複雑な代数計算が必要となる.

クライン群の生成元のレシピを考案することは非常に難しい研究テーマである.
クライン群の極限集合の形状はアートとしてみても非常に魅力的であるが,美し
い極限集合を生成するオリジナルのレシピを作ることは一介のアーティストにで
きることではない.

しかしながら,図\ref{fig:schottky}や図\ref{fig:simple3diis}のような円や
球の反転による軌道の形状には,円や球といった形から幾何学的な直観が働くため,
生成元と可視化される図形の間の関係を容易に理解することができる.
そこで筆者はすべてのメビウス変換を円や球の反転で構成し,その軌道を描くこ
とを考えた.
群の軌道を描く処理は計算量が大きいものであったが,IISを用いることで複雑
な生成元をもつ群も高速に描くことができるようになった.
このように,群の生成元を直観的に構成し,リアルタイムに可視化することは研
究者だけでなく,フラクタルアーティストにも有益なことである.

この節では,二次元と三次元のメビウス変換を円と球の反転の合成で構成し,そ
の軌道をIISで描画するための手法をまとめる.この内容は\textit{Bridges
2017}に投稿中である.
また,筆者は円や球の反転で構成される群をインタラクティブに構成するウェブアプリ
ケーション,{\it Schottky Link} \footnote{Schottky Link:
\url{https://schottky.jp}}を開発している.

\subsubsection{2D Generators}

複素平面に作用するメビウス変換は円の反転を合成することで構成することがで
きる.

\noindent\textbf{Simple Inversion.}

先に述べたように, 円に関する反転は複素平面の向きを変えるので, メビウス変
換ではない.
二つの反転円がペアとなってはじめてメビウス変換となるため,偶数個の円が必
要である.
しかし, ここでは簡単のために一つの円の反転もメビウス変換の元として扱う.

二つの円をペアにする変換は特に\emph{ショットキー変換}(\textit{Schottky
transformation})とよばれる.
ショットキー変換は片方の円の外側をもう片方の円の内側へと移す変換であり,
これは円の反転と同一視することができる.
また,円のペアによるショットキー変換で構成された群を\emph{古典的ショット
キー群}(\textit{classical Schottky groups})とよぶ.

\noindent\textbf{Inversion of a Circle with Infinite Radius.}

無限の半径をもつ円はその円弧を直線として表すことができるので,その反転は
直線に関する反転として扱うことができる.
図\ref{fig:infCircle}では右側の反転円が円の内側である直線の左側に反転で
移されていることがわかる.

\noindent\textbf{Rotation.}

二つの交差する直線の反転の組み合わせは回転を表わす. 回転角は二直線のなす
角の二倍である. 図\ref{fig:rotation}において,二直線は45度で交わっている
ので90度回転を表している. また, 軌道の円同士がお互いに
重なりあわないようにするためには, 回転角は有理角である必要がある.

\noindent\textbf{Parallel Translation.}

平行な二直線による反転の組は直線の垂直方向への平行移動を表す.
図\ref{fig:translation2d}では二つの無限の半径をもつ円が向かい合っている.
これらの円の反転が繰り返されることにより,中央の4つの反転円はx軸方向に
平行移動される.
平行移動の固定点は無限遠点であり,これは放物型変換である.

\begin{figure}[h!tbp]
 \begin{minipage}[t]{0.3\hsize}
  \center
   \includegraphics[width=2in, height=2in, keepaspectratio]{../img/klein/2diis/infCircle.pdf}
  \caption{Inversion of the circle with infinite radius}
  \label{fig:infCircle}
 \end{minipage}
 \hspace*{\fill}
 \begin{minipage}[t]{0.3\hsize}
  \center
  \includegraphics[width=2in, height=2in, keepaspectratio]{../img/klein/2diis/rotation.pdf}
  \caption{Rotation}
  \label{fig:rotation}
 \end{minipage}
 \hspace*{\fill}
 \begin{minipage}[t]{0.3\hsize}
  \center
  \includegraphics[width=2in, height=2in, keepaspectratio]{../img/klein/2diis/translation.pdf}
  \caption{Parallel translation}
  \label{fig:translation2d}
 \end{minipage}
\end{figure}

\noindent\textbf{Composition of Two Circles.}

二つの同心円による反転を合成することで,実数倍による拡縮
を表すことができる.これは双曲型変換となる.
図\ref{fig:scaling2d}における赤の円を$C1$,緑の円を$C2$, そして青の円を
$C1'$とよぶ.ここで$C1'$は$C1$を$C2$による反転で移した像である.
図\ref{fig:scaling2d}では, 白い縁をもつ二つの円の軌道を描いている.
固定点は円の中心と無限遠点である.

IISに用いる写像$G$は以下のように定義される.
また,接頭辞$I$は反転を表わす.
例えば, $I_{C1}$は$C1$による反転を表わす.
\begin{align*}
 G =
  \begin{cases}
   I_{C2} \circ I_{C1} & (\text{The point is inside of } C1) \\
   I_{C1} \circ I_{C2} & (\text{The point is outside of } C1')
  \end{cases}
\end{align*}
$G$を点に繰り返し作用させることでその点を基本領域へと移動させることができる.
この種類の生成元の基本領域は図\ref{fig:scaling2d}における青と緑の領域で
ある.つまり$C1'$の内側かつ$C1$の外側の領域である.
完全な円の軌道をIISを用いて描くには, 反転円をこの領域内に配置す
る必要がある.

次に$C1$を少し動かすと, 図\ref{fig:hyperbolic2d}のような軌道が得られる.
無限遠点の固定点は有限の点に移り,有限の固定点は, $C1$の内部で移動する.
二つの円が交差しない限り, これも双曲型変換である.

$C1$と$C2$が接触する時, 図\ref{fig:parabolic2d}のように,固定点はその交点で重なりあい, 放物型変換となる.

\noindent\textbf{Loxodromic.}

上記の二つの円による反転の合成変換は実数による拡縮と共役であった.
この生成元にさらに二つの反転を加えることで軌道に捻りを加えることができる.
図\ref{fig:loxodromic2d}における$C1$と$C2$の中心を通る白い直線を$L$,
黄色の円を$C3$, そして水色の点を$P$とよぶ.
$P$は制御点であり, この点に$C1$の反転を作用させて得られた点を$P'$,$C2$の
反転を作用させて得られた点を$P''$とする.
$C3$は$P, P', P''$の三点から定義される.
$L$と$C3$の反転の合成は双曲型変換の軌道に捻りを加える回転を表す.

写像$G$は以下のようになる.
\begin{align*}
G =
\begin{cases}
 (I_{C2} \circ I_{C1}) \circ (I_{C3} \circ I_L) & (\text{The point is inside of } C1) \\
 (I_L \circ I_{C3}) \circ (I_{C1} \circ I_{C2}) & (\text{The point is outside of }C1')
\end{cases}
\end{align*}

また, 二つの固定点からこの種類の生成元を得ることもできる.

\begin{figure}[htbp]
 \begin{minipage}[]{0.65\hsize}
  \begin{minipage}[]{0.22\hsize}
   \center
   \includegraphics[width=1.5in, height=1.5in, keepaspectratio]{../img/klein/2diis/scalingEdged.pdf}
   \subcaption{Scaling}
   \label{fig:scaling2d}
  \end{minipage}
 \hspace*{\fill}
  \begin{minipage}[]{0.22\hsize}
   \center
   \includegraphics[width=1.5in, height=1.5in, keepaspectratio]{../img/klein/2diis/hyperbolicEdged.pdf}
   \subcaption{Hyperbolic}
   \label{fig:hyperbolic2d}
  \end{minipage}
 \hspace*{\fill}
  \begin{minipage}[]{0.22\hsize}
   \center
   \includegraphics[width=1.5in, height=1.5in, keepaspectratio]{../img/klein/2diis/parabolicEdged.pdf}
   \subcaption{Parabolic}
   \label{fig:parabolic2d}
  \end{minipage}
  \caption{Composition of two circles}
 \end{minipage}
 \hspace*{\fill}
 \begin{minipage}[]{0.22\hsize}
  \center
  \includegraphics[width=1.5in, height=1.5in, keepaspectratio]{../img/klein/2diis/loxodromicEdged.pdf}
  \caption{Loxodromic transformation}
  \label{fig:loxodromic2d}
 \end{minipage}
\end{figure}

\subsubsection{3D Generators}

球に関する反転の合成で三次元空間に作用するメビウス変換を得る.
ほとんどの三次元の生成元は円を球に拡張することで自然に導くことが
できる.

\noindent\textbf{Simple Inversion.}

二次元の場合と同様に球の反転単体はメビウス変換ではなく, 二つの球のペアに
よる反転がメビウス変換である.

\noindent\textbf{Inversion of a Sphere with Infinite Radius.}

半径が無限大の球の面は平面とみることができるので,その反転は平面による反
転として表すことができる.
図\ref{fig:infSphereGen}では青い板が半径が無限の球を表現している.
図\ref{fig:infSphereOrb}における軌道の右側はこの板によって左側へと
移されているため,軌道の左右に対称性をみることができる.

\noindent\textbf{Rotation.}

二次元の場合と同様にして,互いに交わる二つの平面の反転によって
回転を表すことができる.
回転軸は二枚の平面の交線で,なす角の二倍が回転角となる.
図\ref{fig:rotation3d}では180度の回転を示した.

\noindent\textbf{Parallel Translation.}

図\ref{fig:translation3d}のように二枚の平行な平面による反転の組は平面の
法線方向への平行移動を表わす.
これは無限遠点に固定点をもつ放物型変換である.
基本領域は二枚の平面の間になる.

\noindent\textbf{Compound Parabolic.}

更に, 三次元では平行移動の軌道に捻りを加えることができる.
図\ref{fig:compParabolic}では軌道に平行移動が作用されるたびに回転
が加えられていることがわかる.
この生成元は\emph{複合放物型変換}(\textit{Compound Parabolic})とよばれ,
高次元特有のものである.

\begin{figure}[h!tbp]
 \begin{minipage}[]{0.49\hsize}
  \begin{minipage}[]{0.24\hsize}
   \center
   \includegraphics[width=1.5in, height=1.5in, keepaspectratio]{../img/klein/3diis/infSphereGen.pdf}
   \subcaption{Generator}
   \label{fig:infSphereGen}
  \end{minipage}
  \hspace*{\fill}
  \begin{minipage}[]{0.24\hsize}
   \center
   \includegraphics[width=1.5in, height=1.5in, keepaspectratio]{../img/klein/3diis/infSphereOrbit.pdf}
   \subcaption{Orbit}
   \label{fig:infSphereOrb}
  \end{minipage}
  \hspace*{\fill}
  \caption{Inversion of the Sphere with infinite radius}
  \label{fig:infSphere}
 \end{minipage}
 \begin{minipage}[]{0.49\hsize}
  \begin{minipage}[]{0.24\hsize}
   \center
   \includegraphics[width=1.5in, height=1.5in, keepaspectratio]{../img/klein/3diis/rotationGen.pdf}
   \subcaption{Generator}
   \label{fig:rotationGen}
  \end{minipage}
  \hspace*{\fill}
  \begin{minipage}[]{0.24\hsize}
   \center
   \includegraphics[width=1.5in, height=1.5in, keepaspectratio]{../img/klein/3diis/rotationOrb.pdf}
   \subcaption{Orbit}
   \label{fig:rotationOrb}
  \end{minipage}
  \hspace*{\fill}
  \caption{Rotation}
  \label{fig:rotation3d}
 \end{minipage}
\end{figure}

\begin{figure}[h!tbp]
 \begin{minipage}{0.49\hsize}
  \begin{minipage}{0.24\hsize}
   \center
   \includegraphics[width=1.5in, height=1.5in, keepaspectratio]{../img/klein/3diis/translationGen.pdf}
   \subcaption{Generator}
   \label{fig:translationGen}
  \end{minipage}
  \hspace*{\fill}
  \begin{minipage}{0.24\hsize}
   \center
   \includegraphics[width=1.5in, height=1.5in, keepaspectratio]{../img/klein/3diis/translationOrbit.pdf}
   \subcaption{Orbit}
   \label{fig:translationOrb}
  \end{minipage}
  \hspace*{\fill}
  \caption{Parallel translation}
  \label{fig:translation3d}
 \end{minipage}
 \begin{minipage}{0.49\hsize}
  \begin{minipage}{0.24\hsize}
   \center
   \includegraphics[width=1.5in, height=1.5in, keepaspectratio]{../img/klein/3diis/compParabolicGen.pdf}
   \subcaption{Generator}
   \label{fig:compParabolicGen}
  \end{minipage}
  \hspace*{\fill}
  \begin{minipage}{0.24\hsize}
   \center
   \includegraphics[width=1.5in, height=1.5in, keepaspectratio]{../img/klein/3diis/compParabolicOrb.pdf}
   \subcaption{Orbit}
   \label{fig:compParabolicOrb}
  \end{minipage}
  \hspace*{\fill}
  \caption{Compound parabolic transformation}
  \label{fig:compParabolic}
 \end{minipage}
\end{figure}

\noindent\textbf{Composition of Two Spheres}

二次元の場合と同様に二つの球の反転を組み合わせた生成元を定義することができる.
図\ref{fig:loxoGen3d}に示される赤い球を$S1$, 黄緑の球を$S2$,
青い球を$S1'$とよぶとき,写像$G$は以下のようになる.
\begin{align*}
G =
\begin{cases}
 I_{S2} \circ I_{S1} & (\text{The point is inside of } S1) \\
 I_{S1} \circ I_{S2} & (\text{The point is outside of }S1')
\end{cases}
\end{align*}
$S1$と$S2$に接触がない場合,この生成元は斜航型変換となる.
軌道は図\ref{fig:loxoOrb3d}に示した.
この生成元に6つの反転球を加えた軌道が\ref{fig:loxoOrbSch3d}のようになる.

$S1$と$S2$が一点で接触するとき, この生成元は放物型変換となる.
図\ref{fig:parabolicGen3d}において球同士の接点が固定点となる.
図\ref{fig:parabolicOrb3d}に描かれる球の軌道が固定点に収束していくことが
わかる.

\noindent\textbf{Compound Loxodromic}
最後に, $S1$と$S2$に直交する二つの球を追加する.
図\ref{fig:compLoxoGen}に示される桃色の球を$S3$, 黄色の球を$S4$,
3つの制御点を$P,Q1,Q2$とよぶ.
$P$に$S1$の反転を作用させて得られた点を$P'$,$S2$の
反転を作用させて得られた点を$P''$とする.
そして,写像$G$は以下のようになる.
\begin{align*}
S3 = sphere(P, P', P'', Q1) \quad
S4 = sphere(P, P', P'', Q2) \\
G =
\begin{cases}
 (I_{S4} \circ I_{S3}) \circ (I_{S1} \circ I_{S2}) & (\text{The point is inside of } S1) \\
 (I_{S2} \circ I_{S1}) \circ (I_{S3} \circ I_{S4}) & (\text{The point is
 outside of } S1')\\
\end{cases}
\end{align*}
$S3$と$S4$の反転の合成は回転を表す.
図\ref{fig:compLoxoOrb}では, 二次元における斜航型変換のような三次元空間
における捻じれた軌道をみることができる.
この生成元は\emph{複合斜航型変換}(Compound Loxodromic)とよばれ,高次元特
有のメビウス変換である.

\begin{figure}[h!tbp]
 \begin{minipage}{0.24\hsize}
  \center
  \includegraphics[width=1.5in, height=1.5in, keepaspectratio]{../img/klein/3diis/loxoGenSimple.pdf}
  \subcaption{Generator}
  \label{fig:loxoGen3d}
 \end{minipage}
 \hspace*{\fill}
 \begin{minipage}{0.24\hsize}
  \center
   \includegraphics[width=1.5in, height=1.5in, keepaspectratio]{../img/klein/3diis/loxoOrbSimple.pdf}
   \subcaption{Orbit}
   \label{fig:loxoOrb3d}
 \end{minipage}
 \hspace*{\fill}
 \begin{minipage}{0.24\hsize}
  \center
  \includegraphics[width=1.5in, height=1.5in, keepaspectratio]{../img/klein/3diis/loxoOrbSch.pdf}
  \subcaption{Orbit}
  \label{fig:loxoOrbSch3d}
 \end{minipage}
 \caption{Loxodromic transformation}
 \label{fig:loxodromic3d}
\end{figure}

\begin{figure}[h!tbp]
 \begin{minipage}{0.49\hsize}
  \begin{minipage}{0.24\hsize}
   \center
   \includegraphics[width=1.5in, height=1.5in, keepaspectratio]{../img/klein/3diis/parabolicGen.pdf}
   \subcaption{Generator}
   \label{fig:parabolicGen3d}
  \end{minipage}
  \hspace*{\fill}
  \begin{minipage}{0.24\hsize}
   \center
   \includegraphics[width=1.5in, height=1.5in, keepaspectratio]{../img/klein/3diis/parabolicOrb.pdf}
   \subcaption{Orbit}
   \label{fig:parabolicOrb3d}
  \end{minipage}
  \hspace*{\fill}
  \caption{Parabolic transformation}
  \label{fig:parabolic3d}
 \end{minipage}
 \begin{minipage}{0.49\hsize}
  \begin{minipage}{0.24\hsize}
   \center
   \includegraphics[width=1.5in, height=1.5in, keepaspectratio]{../img/klein/3diis/compLoxoGen.pdf}
   \subcaption{Generator}
   \label{fig:compLoxoGen}
  \end{minipage}
 \hspace*{\fill}
 \begin{minipage}{0.24\hsize}
  \center
  \includegraphics[width=1.5in, height=1.5in, keepaspectratio]{../img/klein/3diis/compLoxoOrb.pdf}
  \subcaption{Orbit}
   \label{fig:compLoxoOrb}
 \end{minipage}
 \hspace*{\fill}
  \caption{Compound loxodromic transformation}
 \label{fig:compLoxo}
 \end{minipage}
\end{figure}

\subsubsection{Optimization}

写像$G$を用いて点を基本領域へ移すためには$G$を繰り返し点に作用させる必要
がある.
しかし,いくつかの写像は適切な共役をとることによって点を一度の操作で基本
領域に移動させることができる.
このような最適化は特に固定点への収束が遅い放物型変換の軌道を描くこ
とに役立つ.
この節では, 二次元の生成元の最適化についてまとめる.
もちろん,三次元の生成元に関しても以下で述べる方法と同様のアプローチで最
適化することができる.

\noindent\textbf{Parallel Translation.}

まずは図\ref{fig:translation2d}と同様の平行な二直線の組による平行移動を
考える.
繰り返し二直線に関する反転を作用させる代わりに, 剰余を用いて点を移すこと
ができる.

まず始めに, 平行移動と回転の合成によって,
平行な二直線の一方をX軸に垂直に, 片方の直線がY軸に沿うようにする変換$T$を考
える.この変換が共役変換となる.
共役変換で移された直線は図\ref{fig:translationMod}に示した.
$w$を二直線の距離, $i$を平行移動によって移された半径無限の円の指数,
$P$を写像に与えられた点に$T$を作用させた点,
$P'$を写像によって移された点, $d$と$d'$をY軸からの距離とする.
$d$を$w$で割ることにより, 剰余$d'$と商$i$を得る.
最後に$d'$を用いて$P'$を計算し, これを$T^{-1}$で元の座標に戻す.
以上の操作を一度行うことで,平面上の全ての点を基本領域に移すことができる.

\noindent\textbf{Parabolic.}

次に図\ref{fig:parabolic2d}のような放物型変換を考える.
まず共役変換$T$をこの放物型変換の固定点を中心とする円の反転とする.
$T$を$C1,C2,C1'$に対して適用すると, 固定点は無限遠点に移るので, 三本の平行な直
線を得ることができる.
それぞれの直線を$TC1, TC2, TC1'$とよぶ.
図\ref{fig:parabolicMod}にこれらの直線を示した.
赤の直線$TC1$と青の直線$TC1'$が平行移動を表す.

最適化された写像の操作過程は以下のようになる.
まず, $T$を与えられた点$P$に作用させ, $T(P)$を得る.
そして, $T(P)$を前述の平行移動と同様の操作で移し,点$P'$を得る.
最後にもう一度$T(= T^{-1})$を$P'$に作用させて元の座標に移し,基本領域上
の点を得る.

\noindent\textbf{Loxodromic.}

最後に図\ref{fig:hyperbolic2d}のような双曲型変換を考える.
共役変換$T$をこの変換の片方の固定点を中心にもつ円に関する反転とする.
$T$を$C1, C2, C1'$に作用させて得られた円をそれぞれ$TC1, TC2, TC1'$とよぶ.
これらはもう一方の固定点に$T$を作用させた点を中心にもつ同心円となる.

これらの円を図\ref{fig:hyperbolicMod}に示した.
図において, $P$を写像に与えられた点, $P'$を写像された点, $r$と$r'$を
$TC1$と$TC1'$の半径, $d$を$T(P)$と同心円の中心からの距離,$d'$を$P'$と同
心円の中心からの距離とする.
$k$を倍数, $q$を指数係数, $i$を円列の指数とする.
これらは以下のように計算される.
 \begin{align*}
  k = \frac{r'}{r} \quad
  q = \log_{k} \frac{d}{r} \quad
  d' = r \cdot k^{fracionalPart(q)} \quad
  i = floor(q)
 \end{align*}
我々は$d'$を用いて$T(P)$を基本領域へと移した点である$P'$を計算することが
できる.また,$i$から通常の合成変換を何度作用させると基本領域へと移せるか
がわかる.

ここで, 双曲型変換の生成元に直線と円の反転が加えられた形である,斜航型変
換であるとき,$T$を図\ref{fig:loxodromic2d}における$L$と$C3$にも作用させ
ることでこの変換も最適化することができる.
$T$によって得られた$L$と$C3$の像を$TL$と$TC3$とよぶ.
$L$と$C3$は二つの固定点を通るので,$TL$と$TC3$は同心円の中心で交わる二直
線となり,これらの反転の組み合わせが同心円の中心を回転中心とする回転を表
す.

$T$によって移された生成元を図\ref{fig:loxodromicMod}に示した.
白の直線が$TL$,黄色の直線が$TC3$である.
斜航型変換では,$P'$を得た後で$P'$に回転を与えて$P''$を得る.
$TL$と$TC3$のなす角を$\theta$, 回転角を$\varphi$とすると, $\varphi$は
$\varphi = 2 \theta i$で計算することができる.

上記の処理をまとめると以下のようになる.
まず, $T$を与えられた点$P$に適用し, $T(P)$を得る.
$TC1$,$TC1'$から$d'$を算出し,$P'$を計算する.
もしも$TL$,$TC3$があれば, 同心円の中心を回転中心として$P'$を$\varphi$回転
させ, $P''$を得る.
最後にもう一度$T$を$P'$もしくは$P''$に作用させ,元の座標に戻すことで基本
領域上の点を得る.

\begin{figure}[h!tbp]
 \begin{minipage}[t]{0.24\hsize}
  \center
  \includegraphics[width=1.5in, height=1.5in, keepaspectratio]{../img/klein/2diis/translationMod.pdf}
  \caption{Parallel translation}
  \label{fig:translationMod}
 \end{minipage}
 \hspace*{\fill}
 \begin{minipage}[t]{0.24\hsize}
  \center
  \includegraphics[width=1.5in, height=1.5in, keepaspectratio]{../img/klein/2diis/parabolicMod.pdf}
  \caption{Inverted parabolic generator}
  \label{fig:parabolicMod}
 \end{minipage}
 \hspace*{\fill}
 \begin{minipage}[t]{0.24\hsize}
  \center
  \includegraphics[width=1.5in, height=1.5in, keepaspectratio]{../img/klein/2diis/hyperbolicMod.pdf}
  \caption{Inverted hyperbolic generator}
  \label{fig:hyperbolicMod}
 \end{minipage}
 \hspace*{\fill}
 \begin{minipage}[t]{0.24\hsize}
  \center
  \includegraphics[width=1.5in, height=1.5in, keepaspectratio]{../img/klein/2diis/loxodromicModRotation.pdf}
  \caption{Inverted loxodromic generator}
  \label{fig:loxodromicMod}
  \hspace*{\fill}
 \end{minipage}
\end{figure}

\subsection{Other Topics}

ここまでは『インドラの真珠』における話題と筆者による成果をまとめた.
この節では, クライン群の研究の中でも興味深い図像を見ることができる先行研
究とその他の可視化手法を簡単に紹介する.

\subsubsection{Quasi Fuchsian 3-Dimensional Fractals}

阿原・荒木は擬フックス群を三次元に拡張した4次元クライン群を考案した
\cite{ahara2003sphairahedral}\cite{ahara2003sphaira}.
これはQuasi Fuchsian 3-Dimensional Fractalsとよばれ,
フラクタルコミュニティに大きな影響を与えたといわれている.

図\ref{fig:schottky}において, 中央の円盤に囲まれた基本領域を\emph{円辺形}と
よぶ.
擬フックス群は円辺形を囲む円の組から生成元を得ることができる.
そこで円辺形を三次元に拡張した\emph{球面体}を定義することで三次元の擬フッ
クス群を構成することができる.
図\ref{fig:sphairahedra}は6つの球に囲まれた球面体のひとつである.
側面は3つの半径無限大の球に囲まれている.
この球面体をそれを囲む球の反転で移していくことで極限集合を得ることができ
る.
図\ref{fig:quasiFuchsian}に極限集合のひとつを示した.
阿原・荒木による動画\footnote{Quasi-fuchsian fractals:
\url{https://www.youtube.com/watch?v=3lcO9zRCv-4}}ではこのフラクタルの構
成過程をみることができる.

また, レイマーチングとDistance Estimationによる高速描画のアルゴリズムが
2012年にKnightyによって開発された
\footnote{Another 3D Kleinian:
 \url{http://www.fractalforums.com/ifs-iterated-function-systems/another-3d-kleinian/}}.

荒木はこのフラクタル図形の物質化についてまとめている\cite{araki2006materializing}.
数学的な事項については蔭山\cite{kageyama2016masterSphaira}がまとめている.

\begin{figure}[htbp]
 \begin{minipage}{0.49\hsize}
  \begin{center}
   \includegraphics[width=3in, height=3in, keepaspectratio]{../img/klein/sphairahedra.pdf}
   \caption{Sphairahedra}
   \label{fig:sphairahedra}
  \end{center}
 \end{minipage}
 \begin{minipage}{0.49\hsize}
  \begin{center}
   \includegraphics[width=3in, height=3in, keepaspectratio]{../img/klein/quasi-fuchsian.pdf}
   \caption{quasi-fuchsian}
   \label{fig:quasiFuchsian}
  \end{center}
 \end{minipage}
\end{figure}

\subsubsection{Other 4-Dimensional Kleinian Groups}

『インドラの真珠』でみるクライン群は複素平面に作用するメビウス変換から構成されている.
現在このようなクライン群は, ほぼ研究され尽されてしまった.
しかし, 先に述べたように, 三次元空間に作用するメビウス変換を四元数を用い
て定義することができる.
これは$Sp^k(1, 1)$とよばれ,2x2の四元数行列で表現される.
四元数で構成されるメビウス変換とその分類は佐久川による
\cite{sakugawa2007master}や\cite{sakugawa2009accidental}にまとまっている.
これらの論文において,佐久川は複合放物型変換を含む群のレシピを考案
している.
図\ref{fig:sakugawa}はそのレシピで生成された群の極限集合を描画したものである.

\begin{figure}[h!tbp]
 \begin{minipage}{0.49\hsize}
  \center
  \includegraphics[width=3in, height=3in, keepaspectratio]{../img/klein/sakugawa1.pdf}
  \subcaption{}
 \end{minipage}
 \hspace*{\fill}
 \begin{minipage}{0.49\hsize}
  \center
  \includegraphics[width=3in, height=3in, keepaspectratio]{../img/klein/sakugawa2.pdf}
  \subcaption{}
 \end{minipage}
 \begin{minipage}{0.49\hsize}
  \center
  \includegraphics[width=3in, height=3in, keepaspectratio]{../img/klein/sakugawa3.pdf}
  \subcaption{}
 \end{minipage}
 \hspace*{\fill}
 \begin{minipage}{0.49\hsize}
  \center
  \includegraphics[width=3in, height=3in, keepaspectratio]{../img/klein/sakugawa4.pdf}
  \subcaption{}
 \end{minipage}
 \caption{The limit set of the 4-Dimensional Kleinian Groups}
 \label{fig:sakugawa}
\end{figure}

また,荒木・糸はマスキット群を高次元に拡張することで四次元クライン群を求
めている.\cite{araki2008extension}
\footnote{4-dimensional Kleinian punctured torus groups:
\url{http://www.math.nagoya-u.ac.jp/~itoken/3d-maskit/3d-maskit.html}}.
この論文中では幾何学的性質をうまく使うことで,四元数の計算を避けている.
佐久川はこの群の四元数表示を導出した\cite{sakugawa2010limit}.

三浦は3つ以上の生成元による4次元クライン群をモジュラー群から構成するこ
とを試みた\cite{miura2015master}.
筆者はこの論文における数値実験と可視化を行った.

\subsubsection{Once Punctured Torus Groups}

クライン群の部分群である一点穴開きトーラス群({\it Once Punctured Torus Groups})は
和田による{\it OPTi}\footnote{OPTi:
\url{http://delta-mat.ist.osaka-u.ac.jp/OPTi/index.html}}という描画ソフ
トウェアによって大きく研究が進んだ.
描画や離散性判定のためのアルゴリズムが開発者の和田によりまとめられている
\cite{wada2003optiDrawingLimit}\cite{wada2006optiDiscreteness}.
この群の極限集合は図\ref{fig:opt}のようになる.

 \begin{figure}[htbp]
  \begin{minipage}{0.5\hsize}
   \center
   \includegraphics[width=3in, height=3in,
   keepaspectratio]{../img/klein/opt1N.pdf}
   \subcaption{}
  \end{minipage}
  \begin{minipage}{0.5\hsize}
   \center
   \includegraphics[width=3in, height=3in,
   keepaspectratio]{../img/klein/opt2N.pdf}
   \subcaption{}
  \end{minipage}
  \caption{The Limit Set of the once punctured torus groups}
  \label{fig:opt}
 \end{figure}

\subsubsection{Other Rendering Methods}

ここまでに述べたクライン群の可視化手法とは別のアルゴリズムも存在する.

Aaron MontagはIterated Inversion Systemを拡張したHyperbolic Iterated
Function Systemによる極限集合の描画を提案している\cite{Montag2014hyperbolicIFS}
\footnote{Kleinian Groups WebGL:
 \url{https://www-m10.ma.tum.de/bin/view/Lehrstuhl/AaronMontagKleinian}}.

また, Jos Ley, Knightyはマスキット群におけるEscape-timeアルゴリズムの式
とDistance Estimationを考案した
\footnote{An escape time algorithm for Kleinian group limit set:\\ \quad
\quad \url{http://www.fractalforums.com/3d-fractal-generation/an-escape-tim-algorithm-for-kleinian-group-limit-sets/}}.
荒木・糸\cite{araki2008extension}が考えた群と同様に, 一方に平行移動を含
む2つの放物型変換による群を考えている.
この方法を用いると, 二次元だけでなく,三次元の極限集合も高速に描画することができる.
詳細なアルゴリズムに関しては, Jos Leyがまとめている
\footnote{Mathematical Imagery, An escape-time algorithm for a family of
Kleinian groups:\\ \quad \quad \url{http://www.josleys.com/article_show.php?id=221}}.

\subsection{Further Readings}

この章の最後に,クライン群の数学的な背景を学習するための書籍をいくつか挙げておく.
クライン群は双曲幾何学の一分野である.
双曲幾何学の入門書には『双曲幾何学への招待』
\cite{taniguchi_okumura199610invitation}や
『双曲幾何』\cite{mitani200409hyperbolicGeometry}がある.
クライン群のより数学的内容に踏み込んだ書籍には『双曲多様体とクライン群』
\cite{taniguchi_matsuzaki_hyperbolicManifold}や『Outer Circles』
\cite{Marden200705outerCircles}がある.
また, レクチャーノート集に『Kleinian Group and Hyperbolic 3-Manifolds』
\cite{Y._V._C.200311}や
『Spaces of Kleinian Groups』\cite{Yair_Makoto_Caroline200606}がある.

\clearpage


\section{Tessellation}

敷き詰めはテセレーション(Tessellation),タイリング(Tiling)とも呼ばれる.例えば図は黒の基本領域を周囲の円盤の反転で移した敷き詰め模様と考えることもできる.タイリングは幾何学の重要の道具であると同時に,装飾やアートとしても非常に人気のある分野である.
この章では,シェーダを用いてタイリングを描く方法とその他の参考文献について少しであるがまとめる.

\subsection{About Rendering}

図のような正方形のタイリングを考える.
図のように,基本となる正方形を縦横全4種類の変換の組み合わせで平面全体を正方形で埋め尽くすことができる.これは前章で紹介した,変換の木構造の幅優先探索で正方形を移動して描画することができる.
GLSLを用いる場合はその逆を行う.すなわち,各ピクセルを敷き詰めを構成する4種の変換で基本タイルへと点を動かす.
そうして,基本タイルへ動かした後で操作の回数や基本タイル状の色を参照し描画する.
このアルゴリズムをまとめると以下のような手順になる
\begin{enumerate}
 \item 基本タイルを見つける
 \item 基本タイルを敷き詰めるための変換を見つける
 \item 各ピクセルを基本タイルに入るまで変換し続ける
 \item 変換の回数や基本タイル上の色を使って色を付ける
\end{enumerate}

\begin{figure}[h!tbp]
   \begin{subfigure}{0.3\textwidth}
   \begin{center}
    \includegraphics[width=2in, height=2in, keepaspectratio]{../img/tessellation/rectTile.pdf}
    \caption{rectangular tiling}
    \label{fig:rectTile}
   \end{center}
  \end{subfigure}
 \hspace*{\fill}
   \begin{subfigure}{0.3\textwidth}
   \begin{center}
    \includegraphics[width=2in, height=2in, keepaspectratio]{../img/tessellation/tile.pdf}
    \caption{Tiling}
    \label{fig:tile}
   \end{center}
  \end{subfigure}
 \hspace*{\fill}
 \begin{subfigure}{0.3\textwidth}
  \begin{center}
   \includegraphics[width=2in, height=2in, keepaspectratio]{../img/tessellation/tileMove.pdf}
   \caption{Move}
   \label{fig:tileMove}
  \end{center}
 \end{subfigure}
 \hspace*{\fill}
\caption{Tile}
\end{figure}

ユークリッド平面状での敷き詰め模様は,数学的に分類がされている.『装飾パターンの法則』\cite{tessellationDesign}では装飾に用いられる敷き詰め模様の法則を解説するとともにデザインへの応用についてもまとめられている.
\cite{hyperbolization}はユークリッド平面上での敷き詰めを双曲平面での敷き詰めに変換する方法についてまとめられている.また,著者が書いたmorenaments\footnote{morenaments: http://www.morenaments.de/}は各敷き詰めパターンの挙動をみることができる.

\subsection{Hyperbolic Tessellation}
双曲タイリングは双曲平面上での
双曲幾何学では,ポアンカレの円盤モデルと呼ばれる.円盤の端がユークリッド平面における無限遠点となり,円弧がこの平面における直線だと考えられる.また,三角形の和が180度以下になるなどの興味深い現象もみられる.
この平面における敷き詰めが双曲タイリング(Hyperbolic Tessellation)と呼ばれ,画家のエッシャーはこの図に影響されて有名な『天使と悪魔』や『Circle Limit』といった作品を生み出した.

図\ref{fig:poincare}の半球の内側にある円盤がポアンカレの円盤モデルの円盤である.直線を半径が無限の円盤の円周だとみなすと,あらゆるタイルが円弧で構成されていることがわかる.
円盤の内側と外側は円に関する反転で移りあう.
幾何学的には内側も外側も同じものとして扱う.

円盤は原点中心,半径1の円とする.中央の最も大きなタイルはx軸,y軸,そして2つの円弧で囲まれた領域である.
これらの弧に関する反転を用いて,先ほどのアルゴリズムを用いると図\ref{fig:outer}のような敷き詰め模様を描くことができる.
円盤の内側に属する点は内側の基本タイルに,円盤の外側に属する点は外側の基本タイルに移るため,最終的な点の位置で処理を切り分けることにより,両者の色を区別することができる.
双曲タイリングは,木構造の探索によって計算しようとすると,計算量が多くなるため,円盤の端まで描画することは難しいが,前述のアルゴリズムを用いると端までリアルタイムで描画することができる.

図\ref{fig:outer}では基本タイルとして,3つの角が$\frac{\pi}{2}$,残りの角が$\frac{\pi}{3}$である4辺形を用いたが,基本タイルでも敷き詰めることができるわけではない.
敷き詰め条件はポアンカレの張り合わせ定理として知られている.双曲三角形のタイリング条件は以下のようになる.
\begin{eqnarray*}
\text{3内角を}\frac{\pi}{p},\frac{\pi}{q},\frac{\pi}{r}\text{とする時,}
 \frac{1}{p} + \frac{1}{q} + \frac{1}{r} < 1 \text{を満たす.}
\end{eqnarray*}
双曲三角形を2つ張り合わせることで双曲4辺形を得ることができる.

Vladimir Bulatov氏は円盤を変形させる方法を考案した\cite{bending}.Bulatov氏は双曲平面や双曲空間におけるタイリングや,その変形についてまとめている\footnote{http://bulatov.org/math/index.html}.
筆者は図\ref{fig:deformed}のような双曲タイリングとその変形を行うWebアプリケーションを制作している.
\footnote{Hyperbolic Tessellator: https://soma-arc.net/HyperbolicTessellator/}.


\begin{figure}[h!tbp]
 \begin{minipage}{0.49\hsize}
   \begin{center}
    \includegraphics[width=3in, height=3in, keepaspectratio]{../img/tessellation/poincare.pdf}
    \caption{Poincare disk}
    \label{fig:poincare}
   \end{center}
 \end{minipage}
 \hspace*{\fill}
 \begin{minipage}{0.49\hsize}
   \begin{center}
    \includegraphics[width=3in, height=3in, keepaspectratio]{../img/tessellation/outer.pdf}
    \caption{Hyperbolic Tessellation}
    \label{fig:outer}
   \end{center}
 \end{minipage}
\end{figure}


\begin{figure}[htbp]
 \begin{center}
      \includegraphics[width=3in, height=3in, keepaspectratio]{../img/tessellation/deformedTessellation.pdf}
    \caption{deformed}
    \label{fig:deformed}
 \end{center}
\end{figure}


\end{document}
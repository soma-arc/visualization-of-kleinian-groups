%#!uplatex main.tex

\section{Kleinian Groups}

クライン群について学習する際には書籍『インドラの真珠』\cite{komori-indra201303}
(原著:\cite{MumfordSeriesWright200204})を読むことを勧める.
インドラの真珠は数学者以外にもクライン群の魅力を伝えるために書かれた.
中には研究者レベルの高度な内容も扱う部分もあるが,高校レベルの数学と簡単なプロ
グラムを組む能力があれば,自分で図をレンダリングしながら読み進めることができる.
この章では,クライン群に関する用語を確認した後にレンダリング手法と『イン
ドラの真珠』では言及のないクライン群の話題をまとめる.

\subsection{Conjugation}

クライン群を考えるうえで, 我々は様々な変換を扱う.
変換に関する操作において重要となるのが, \emph{共役}(\textit{Conjugation})
である.
共役を用いることで, ある変換をその性質を変えずに異なる座標において作
用させることができ, 複雑な変換をよりわかりやすい座標における変換としてみ
ることができる.

例えば, 図に示された矩形の中心における回転行列を求めたいとする.
原点を中心とした回転行列はよく知られているため, これを$T$とおく.
矩形の中心までの平行移動を表わす変換を$S$とすると, 求めたい行列$\hat{T}$
は$\hat{T} = STS^{-1}$で求めることができる.
これは, 矩形を一度原点まで平行移動で移し, 原点中心の回転を作用させ, 元の
場所まで戻すという操作に等しい.
ここで, 変換$\hat{T}$を$T$の共役, $S$を共役変換と呼ぶ.
$\hat{T}$はこの操作を行なった後でも回転を表わす変換であることに変わりはなく,
$\hat{T}$は回転と共役であるという.

\subsection{Group}

クライン群の\emph{群}(\textit{Group})という語は変換群のことを指す.
ここでは変換の集まりだと考えるとわかりやすい.
例えば,ある変換$f(z)$と$g(z)$,そしてそれらの逆変換の4つの変換を考える
とき,これらの組み合わせを合成することで得られる変換の集合を群とよぶ.
また,ここで群を構成した4つの変換を\emph{生成元}(\textit{Generator})とよ
ぶ.
この群は2つの変換から生成されるので\emph{二元生成群}というようによばれることも
ある.
クライン群は\emph{メビウス変換}(M\"obius Transformations)を生成元にもつ
群のなかで離散性という性質を持つものをいう.

ここから,それぞれの変換を小文字のアルファベットで表し,その逆変換を大文字で綴ることにする.
$f(z)$を$a$,$g(z)$を$b$とおくと逆変換は$A$,$B$となる.
変換の合成はアルファベットを並べて書くことで表す.
例えば合成変換$f(f(g(f(p))))$は$aaba$という語で表される.
また,上付文字はその変換の無限列とする.
例えば,$\overline{a}$は,$aaaaa...$というような変換$a$の無限列となる.
この無限列に対応する合成変換はどの点に作用させてもその点を変換$a$の
\emph{固定点}に移す.
固定点とは,$f(z) = z$となるような,変換を使って動かすことのできない点である.
 \begin{align*}
  Generators =
   \begin{cases}
    a \colon f(z) \\
    A \colon f^{-1}(z) \\
    b \colon g(z) \\
    B \colon g^{-1}(z)
   \end{cases}
  \quad
  Group =
   \begin{cases}
    aab\\
    abbbaB \\
    ABBAbba \\
    baBBAbaaa \\
    ...
   \end{cases}
 \end{align*}

\subsection{M\"obius Transformations}

二次元のメビウス変換は複素平面上に, ひとつの無限遠点を付け加えた拡張複素
平面$\hat{\mathbb{C}}$上で定義される.
これはリーマン球面とも呼ばれ, 拡張複素平面全体を球面として扱うことができ
る.
メビウス変換は一次分数変換とも呼ばれ,以下のように表わされる.
\begin{align*}
 a, b, c, d\in \hat{\mathbb{C}} \quad f(z) = \frac{az + b}{cz + d}
\end{align*}
メビウス変換は,等角性や円円対応といった性質をもつ.
我々になじみの深い平行移動,拡縮や回転といった変換もメビウス変換に含まれる.

メビウス変換は,以下のように係数を2×2の複素数行列として扱うことで,
代数として操作することができる.
例えば合成変換を行列の積として表すことができる.
\begin{align*}
  m = \left(
 \begin{array}{ccc}
  a & b \\
  c & d
 \end{array}
 \right)
\end{align*}

メビウス変換はおおまかに\emph{放物型}(\textit{Parabolic}), \emph{斜航
型}(\textit{Loxodromic}), \emph{楕円型}(\textit{Elliptic})の三種類に分
類される.
斜航型変換と楕円型変換は2つ,放物型変換は1つの固定点をもつ.
斜航型変換は単純な回転を除いた複素数による拡縮と共役である.
また斜航型変換の中でも,正の実数による拡縮と共役の
変換を\emph{双曲型}(\textit{Hyperbolic})と呼ぶ場合もある.
放物型は平行移動と共役で, 楕円型変換は回転と共役である.

クライン群の描画において,メビウス変換の分類は重要である.
なぜならば,楕円型変換を含む群の多くはクライン群にはなりえない.
また,放物型変換は固定点への収束が遅いという特徴をもつ.
これは後述する極限集合の描画の際に考慮することになる.

\subsection{Circle Inversion}

円は複素平面に作用するメビウス変換を考えるうえで重要な図形である.
先に述べたように,メビウス変換は円を円に移す円円対応という性質
をもつ.
ここでは円に関する反転(\textit{Circle Inversion})という操作をみる.
これは円弧を軸とみた鏡映変換であり,無限遠点は円の中心と入れ替えられる.

点aを中心とする半径rの円に関する反転を$I(z)$とすると式は以下のようになる.
\begin{align*}
I(z) =
 \begin{cases}
  \frac{r^2}{\overline{z - a}} + a \quad &for \quad z \in \mathbb{C} - \{a\} \\
  a \quad &for \quad z = \infty\\
  \infty \quad &for \quad z = a
 \end{cases}
\end{align*}
また,半径が無限大の円は直線とすることで,その円に関する反転は直線
に関する反転と考えることができる.

図\ref{fig:circleInversion}は黒い円に関する反転によって移される図形を可
視化したものである.
緑色の矩形を観察すると,変換の前後で大きく形を歪められているが, その角は直
角が保たれている.
直線は無限遠点が円の中心へと移るため, 反転円の中心を通る円となる.
また, 桃色の円は操作の前後で円が保たれている.

円に関する反転は複素平面の向き付けを逆にしてしまうため, メビウス変換では
ないが, 2つの反転を合成することで双曲型のメビウス変換となる.
偶数個の円の反転を組み合わせることで, その他の種類のメビウス変換を構成することが
できる.
円に関する反転はメビウス変換を構成する変換の最小単位ともいえる.

また, 円を球に拡張することで球に関する反転を考えることができる.
これを用いて三次元空間に作用するメビウス変換を構成することもできる.

\begin{figure}[htbp]
 \begin{center}
      \includegraphics[width=3in, height=3in, keepaspectratio]{../img/klein/circleInversion.pdf}
    \caption{Circle Inversion}
    \label{fig:circleInversion}
 \end{center}
\end{figure}

\subsection{Stereographic Projection}

拡張複素平面はリーマン球面という名の通り, 球面で表すことで全
体をよく見ることができる.
複素平面を球面へと写像するためには, \emph{立体射影}({\it
stereographic projection})と呼ばれる方法がよく使われる.
地図の制作のため, 地球のような球面を平面に投影する方法が様々に考えられて
きたが, それらの手法の中でも立体射影は角度と円を保つため, メビウス変換と相性が良い.

立体射影は以下のように行える.
ある球面上の点を$P$とおくと, $P$を立体射影で移した複素平面上の点は球の北極
から$P$へと引いた直線と複素平面との交点となる.
図\ref{fig:stereoProject}は立体射影を図示したものである.
赤の円は単位円,青の直線はx軸を表している.

\begin{figure}[htbp]
 \center
 \includegraphics[width=3in, height=3in, keepaspectratio]{../img/klein/stereoProject.pdf}
 \caption{Stereographic Projection}
 \label{fig:stereoProject}
\end{figure}


Saul Schleimer,Henry Segermanはメビウス変換を利用して全天球画像(上下左
右360度パノラマ写真)を編集することを考案した\cite{bridges2016-15}.
彼らは\emph{ドロステ効果}(\textit{Droste-effect})とよばれる効果をもつ再
帰的な画像をメビウス変換を用いてつくる手法を提案している
\footnote{Spherical Droste video:
\url{https://www.youtube.com/watch?v=qvh-EAipIUk}}.

全天球画像は,その名の通り球面に張り付けられた画像であるので,メビウス変
換で角度を保ったまま編集を加えることができる.
また全天球画像は正距円筒図法(Equirectangular map)で扱われることも多い.
この図法は縦軸に緯度,横軸に経度をとり,球面をそのまま写像するものである.
メビウス変換で球面の画像を編集したものを正距円筒図法で写像すると興味深い
効果をみることができる.

図\ref{fig:sphericalRotation},\ref{fig:sphericalTranslation},
\ref{fig:sphericalScaling}は図\ref{fig:sphericalStandard}の格子模様
に対して,簡単なメビウス変換を作用させたものである.
複素平面,リーマン球面,その正距円筒図,そして全天球画像の一部を注視した
図を示した.

\begin{figure}[htbp]
 \begin{minipage}{0.5\hsize}
  \center
  \includegraphics[width=3in, height=3in, keepaspectratio]{../img/klein/spherical.pdf}
  \subcaption{Standard}
  \label{fig:sphericalStandard}
 \end{minipage}
 \begin{minipage}{0.5\hsize}
  \center
  \includegraphics[width=3in, height=3in, keepaspectratio]{../img/klein/sphericalRotation.pdf}
  \subcaption{Rotation}
  \label{fig:sphericalRotation}
 \end{minipage}
 \begin{minipage}{0.5\hsize}
  \center
  \includegraphics[width=3in, height=3in, keepaspectratio]{../img/klein/sphericalTranslation.pdf}
  \subcaption{Translation}
  \label{fig:sphericalTranslation}
 \end{minipage}
 \begin{minipage}{0.5\hsize}
  \center
  \includegraphics[width=3in, height=3in, keepaspectratio]{../img/klein/sphericalScaling.pdf}
  \subcaption{Scaling}
  \label{fig:sphericalScaling}
 \end{minipage}
 \caption{Spherical images}
 \label{fig:spherical}
\end{figure}

\subsection{Vizualization with Graph Traversal Approach}

群を構成する変換の全組み合わせは,変換の語で構成される木構造で表現するこ
とができる.
図\ref{fig:wordTree}にその木構造のひとつを示した.
このような木構造はケーリーグラフ(Cayley graph)と呼ばれる.
この図では変換$a, b$とそれらの逆変換である4種類の変換の合成の組合せを表
わしている.
例えば,$a$というラベルがふられたノードでは,外側へ三本のエッジが
張られ,$ab, aa, aB$という合成変換を表すノードへとつながっている.
また,$aA$は逆変換を合成することになるので,このグラフ上では逆戻りを表し
ている.

このグラフを探索することで,群を構成する合成変換を列挙していくことができ
る.この節では,ケーリーグラフを探索することで得られる合成変換を用いて群の作
用を可視化する方法をみる.

群を構成するためには複数のメビウス変換が必要であるが,メビウス変換は4つ
の複素数で構成されているため,自由度が高く,そのまま群を考えることは難し
い.
そのため,群の生成元を得るための「レシピ」がよく用いられる.

例えば, 『インドラの真珠』では「おばあちゃんのレシピ」とよぶレシピが紹介
されている.
このレシピは, 二つの複素数$t_a, t_b$をパラメータにとり,以下のような操作
で2つの生成元を得ることができる.
\begin{enumerate}
 \item 二つの複素数,$t_z, t_b$を選ぶ.
 \item  二次方程式
        $x^2 - t_a t_b x + t_a^2 + t_b^2 = 0 \text{の一方の解}x\text{を
        選び, }t_{ab}= x \text{とする. }$
 \item $z_0$を次のように定義する.
       \begin{align*}
        z_0 = \frac{(t_{ab} -2)t_b}{t_b t_{ab} - 2 t_a + 2it_{ab}}
       \end{align*}
 \item 生成元の行列は次のようになる.
        \begin{align*}
       &a = \left(
      \begin{array}{ccc}
       \frac{t_a}{2} & \frac{t_a t_{ab} - 2 t_b + 4i}{(2 t_{ab} + 4)z_0} \\
       \frac{(t_a t_{ab} - 2 t_b -4i)z_0}{2 t_{ab} - 4} & \frac{t_a}{2}
      \end{array}
     \right)\\
 &b = \left(
      \begin{array}{ccc}
       \frac{t_b - 2i}{2} & \frac{t_b}{2} \\
       \frac{t_b}{2} & \frac{t_b + 2i}{2}
      \end{array}
     \right)
        \end{align*}
\end{enumerate}
これによって8つの複素数のパラメータをもつ二元生成群を2つの複素数から生
成することができ,考えやすくなる.
また,「おばあちゃんのレシピ」は群を可視化した際に綺麗な図がレンダリング
されるように共役によって調整されている.

\begin{figure}[htbp]
 \center
 \includegraphics[width=3in, height=3in, keepaspectratio]{../img/klein/wordTree.pdf}
 \caption{Word Tree}
 \label{fig:wordTree}
\end{figure}

\subsubsection{The Orbit of Transformations}

まず, 円の反転で構成される群を可視化してみる.
円の反転は変換自体を円という図形で表現できるため, その作用を理解しやすい.
図\ref{fig:schottky}には4つの大きな円の内側にたくさんの円が入れ子状に描
かれている.
これは反転円の\emph{軌道}とよばれる.また,円列の極限を\emph{極限集合}とよぶ.

\begin{enumerate}
 \item 図\ref{fig:level0}に示した4つの大きな円を反転円とよぶ.
       これらの円に関する反転を生成元とする群の軌道を描くことを考える.
 \item それぞれの反転円に関する反転は\ref{fig:level0inv}のように,自分以外の反
       転円を自分の内側に移す.
 \item この操作を4つの反転円でおこなうと, それぞれの円の内側には3つの円が移
       され,図\ref{fig:level1}のように合せて12個の小円ができる.
 \item 次に,新たにできた小円に対してもその小円が属している反転円以外の反転を作
       用させる図\ref{fig:level1inv}
 \item 小円の下に新たな小円ができる.図\ref{fig:level2}
 \item このことを繰り返すと,最終的に図\ref{fig:levelMax}を得ることができる.
\end{enumerate}

\begin{figure}[htbp]
 \begin{minipage}{0.33\hsize}
  \begin{center}
   \includegraphics[width=2in, height=2in, keepaspectratio]{../img/klein/orbit/level0.pdf}
   \subcaption{orbit}
   \label{fig:level0}
  \end{center}
 \end{minipage}
 \begin{minipage}{0.33\hsize}
  \begin{center}
   \includegraphics[width=2in, height=2in, keepaspectratio]{../img/klein/orbit/level0inv.pdf}
   \subcaption{orbit}
   \label{fig:level0inv}
  \end{center}
 \end{minipage}
 \begin{minipage}{0.33\hsize}
  \begin{center}
   \includegraphics[width=2in, height=2in, keepaspectratio]{../img/klein/orbit/level1.pdf}
   \subcaption{orbit}
   \label{fig:level1}
  \end{center}
 \end{minipage}
 \begin{minipage}{0.33\hsize}
  \begin{center}
   \includegraphics[width=2in, height=2in, keepaspectratio]{../img/klein/orbit/level1inv.pdf}
   \subcaption{orbit}
   \label{fig:level1inv}
  \end{center}
 \end{minipage}
 \begin{minipage}{0.33\hsize}
  \begin{center}
   \includegraphics[width=2in, height=2in, keepaspectratio]{../img/klein/orbit/level2.pdf}
   \subcaption{orbit}
   \label{fig:level2}
  \end{center}
 \end{minipage}
 \begin{minipage}{0.33\hsize}
  \begin{center}
   \includegraphics[width=2in, height=2in, keepaspectratio]{../img/klein/orbit/levelMax.pdf}
   \subcaption{level max}
   \label{fig:levelMax}
  \end{center}
 \end{minipage}
 \caption{Process}
\end{figure}

図\ref{fig:schottky}では反転円の軌道を描いが, 同様にして異なる図形の
軌道を描くこともできる.
例えば, 図\ref{fig:orbit}は中央に置いた猫の画像を4つ反転円による反転で移
した軌道を描いた.

図\ref{fig:apr}はP.Nylander氏による作品\footnote{bugman123.com Fractals:
\url{http://bugman123.com/Fractals/index.html}}を参考に描画した.
中央右側の蝶をおばあちゃんのレシピで得られた群の生成元で移した.
虹色の曲線は極限集合である.
どちらの図も図形の軌道は最終的に極限集合へと収束していくことがわかる.

このアルゴリズムは変換で構成されるケーリーグラフを幅優先探索で探索するこ
とに等しい.
例えば, 図\ref{fig:wordTree}の木を時計回りに探索すると$a, b, A, B,
aB, aa, ab, ba, bb, ...$の順に合成された変換を得ることができる.
あらかじめ決められた深さまで調べたら, 得られたすべての合成変換を大本とな
る図形(図\ref{fig:orbit}においては中央の猫)に対して作用させることで, 軌
道を計算することができる.
このようにして描かれた図形の軌道を軌道をみることで,おおまかに生成元の作
用を観察することができる.

\begin{figure}[htbp]
 \begin{minipage}{0.49\hsize}
  \begin{center}
   \includegraphics[width=3in, height=3in, keepaspectratio]{../img/klein/schottkyCircles.pdf}
   \caption{Schottky Circles}
   \label{fig:schottky}
  \end{center}
 \end{minipage}
 \begin{minipage}{0.49\hsize}
  \begin{center}
   \includegraphics[width=3in, height=3in, keepaspectratio]{../img/klein/circleOrbit.pdf}
   \caption{orbit}
   \label{fig:orbit}
  \end{center}
 \end{minipage}
\end{figure}

\subsubsection{Rendering the Limit Set}

軌道の元となる図の位置によって得られる図が変わることや,幅優先探索
は計算量が大きいことから,より複雑なメビウス変換群を可視化する際には, 極
限集合を直接描く方法が使われる.

極限集合はケーリーグラフにおいて, 無限の深さのノードの合成変換に対応している.
しかし, 現実にそれを求めることは不可能なので, 群の生成元の固定点をある程
度の長さをもった変換の語で移すことで求める.

例えば, 変換$a$の固定点は任意の点に$a$を無限回作用させた点, つまり
$\overline{a}$を作用させた点と考えることができ,その点は極限集合に含まれる.
そのため,合成変換$abbAAb\overline{a}$で表される点は$a$の固定点を合成変換
$abbAAb$で移すことで求めることができ,これも極限集合に含まれる.
この時, $abbAAb$を接頭語と呼ぶ.

このように,ケーリーグラフを探索して接頭語を求め,それを生成元の固定点に
作用させることで極限集合を得ることができる.
接頭語を求める際には,深さ優先探索が用いられる.
図\ref{fig:wordTree}の語の木を時計回りに長さ3までの語を探索すると,$ abA, abb, aba, aab ...$という順で接頭語が得られる.
これらの語に, 逆変換をかけないように, 生成元の固定点を与えると極限点を求めることができる.

例えば$aBA$を用いて$\overline{A}, \overline{b}, \overline{B}$の3つの固定点を変換すると互いに近い3つの極限点が得られる.
$aab\overline{a}$と$aaa\overline{a}$のように隣同士の接頭語を持つ極限点や
$aaa\overline{a}$と$aaa\overline{b}$のような同じ接頭語を持つ極限点同士は
近い位置にあることが証明されている.
そのため, 得られた点同士を順番に結ぶと図\ref{fig:limitset}のようなフラクタル
構造を持つ曲線を得ることができる.
また,得られた点同士の距離によって,より深い語を探索するかどうかを判断し,
そこで計算を打ち切ることで探索を高速化することができる.

ただし, 綺麗に描画するには様々な工夫が必要となる.
まず, 放物型変換は固定点への収束に時間がかかるため, 放物型変換の固定点付近
をうまく描画するためには,グラフを深く探索しなければならない.
図\ref{fig:limit1}の右端をよく見ると直線で結ばれていることがわかる.
ここには放物型変換の固定点が存在しており, 収束が遅いため, この近辺の極限
点を得るためには, より長い接頭語が必要となる.
『インドラの真珠』では放物型の生成元を特別視して扱う「特殊語アルゴリズム」と
よばれるアルゴリズムでこの問題に対処する.

また, ある変換とその逆変換を合成してしまうと恒等変換になってしまうことは
自明であるが,特定の変換の組み合わせが恒等変換を生み出してしまうことがある.
例えば,語$a$を$f(z) = z + 1$,語$b$を$g(z) = z + i$とするとき,
abABという語が表す合成変換は恒等変換となってしまう.
このような語を$\dot{も}\dot{た}\dot{な}\dot{い}$群を\emph{自由群}と呼ぶ.
こういった語が存在すると,ケーリーグラフの探索の効率が悪くなってしまうた
め,効率よく描画するためにはこういった語を取り除く必要がある.
このことに対処するためには有限オートマトンが使われる.
詳しくは『インドラの真珠』を参照されたい.
変換群に関する有限オートマトンを取り扱った書籍には『Word Processing In
Groups』\cite{Epstein:1992:wordProcessing}がある.

先に述べたように,メビウス変換群の離散部分群がクライン群となる.
そのため群の中には描画すると極限集合が収束せず,図\ref{fig:non-discrete}
のように乱れるものがある.
このような群は\emph{非離散群}と呼ばれ,クライン群ではない.
メビウス変換群のなかで,どのようなパラメータを持つ群が離散的であるか,す
なわちクライン群となるかを調べることがクライン群の研究テーマの一つとなる.
離散,非離散を判定するために有用な定理として\emph{ポアンカレの多面体定理}
や\emph{ヨルゲンセン不等式定理}とよばれるものが存在する.

離散と非離散のパラメータ領域を描画した図は\emph{スライス}と呼ばれ,
スライスの離散と非離散の境界もしばしばフラクタル形状となる.
また,その境界付近のパラメータによる群を描画するとしばしば極限集合が興味
深い形になる.
山下靖のウェブページ\footnote{Discreteness Locus:
\url{http://vivaldi.ics.nara-wu.ac.jp/~yamasita/Slice/}}では様々な
スライスの画像を見ることができる.

\begin{figure}[htbp]
 \begin{minipage}{0.5\hsize}
  \center
  \includegraphics[ height=2.5in, keepaspectratio]{../img/klein/limitset/limit1.pdf}
  \subcaption{}
  \label{fig:limit1}
 \end{minipage}
 \begin{minipage}{0.5\hsize}
  \center
  \includegraphics[ height=2.5in, keepaspectratio]{../img/klein/limitset/limit2.pdf}
  \subcaption{}
  \label{fig:limit2}
 \end{minipage}
 \begin{minipage}{0.5\hsize}
  \center
  \includegraphics[ height=2.5in, keepaspectratio]{../img/klein/limitset/limit3.pdf}
  \subcaption{}
  \label{fig:limit3}
 \end{minipage}
 \begin{minipage}{0.5\hsize}
  \center
  \includegraphics[ height=2.5in, keepaspectratio]{../img/klein/limitset/limit4.pdf}
  \subcaption{}
  \label{fig:limit4}
 \end{minipage}
 \caption{The limit set of Kleinian groups}
 \label{fig:limitset}
\end{figure}

\begin{figure}[htbp]
 \begin{minipage}{0.5\hsize}
  \center
  \includegraphics[ height=2.5in, keepaspectratio]{../img/klein/limitset/non-discrete1.pdf}
  \subcaption{}
 \end{minipage}
 \begin{minipage}{0.5\hsize}
  \center
  \includegraphics[ height=2.5in, keepaspectratio]{../img/klein/limitset/non-discrete2.pdf}
  \subcaption{}
 \end{minipage}
 \caption{The limit set of non-discrete groups}
 \label{fig:non-discrete}
\end{figure}

\subsubsection{Faults of Graph Traversal Approach}

ケーリーグラフを探索する方法にはいくつかの欠点がある.
まず, 生成元の数を増やしたり,探索を深くしたりするごとに指数オーダーで計
算量が増えてしまうことが挙げられる.
さらに木構造の探索はそもそも並列化に向かないため, 高速化が難しい.
また, 部分的に拡大したい場合に無駄な部分を計算しないといった処理にも手間
がかかる.
このような欠点を避けるため,木構造の探索に頼らない方法がいくつか考案され
ている.

しかしながら, 最新のOpenCLやCUDAといった並列計算プラットフォームでは
Dynamic Parallelismという機能を用いることで,木構造探索の並列計算を効率
よく行うことが可能である.
ただし, 最新のハードウェアに依存した機能のため, どの計算機環境でも使える
とはいえない.

\subsection{Iterated Inversion System}


筆者らは, 1章でみたシェーダを用いたフラクタルのレンダリングアルゴリズム
に着想を得て, 円や球の反転で構成される群の軌道を高速に描画するためのアル
ゴリズム, \textit{Iterated Inversion System (IIS)}\cite{bridges2016-367}を開発した.
円や球の座標と半径を直接計算するこれまでのアプローチに対して,このアルゴ
リズムは任意の点が属する円や球の深さを特定する.
これはスクリーンスペースのピクセルそれぞれに対して独立に計算することがで
きるので,シェーダによる並列計算,描画を行うことができる.

アルゴリズム自体は非常にシンプルである.
先にみた4つの反転円による群を考える.
このとき,黒い領域を基本領域とよぶ.
ある点がいずれかの反転円に属している時にその反転円に関する反転を行なうという操作を
反転後の点がすべての円の外側である基本領域に移されるまで繰り返す.
最終的に, 行なった反転の回数が, その点が属している円盤の深さとなる.
図\ref{fig:iis-orb}では青い点の軌道を描いている.
この点は二回反転が行なわれているので, 下から二番目の円に属している.
実際にこのアルゴリズムを用いてすべての点を基本領域へと移動させるためには,
無限回の反転が必要である. そのため, あらかじめ最大の反転回数を決めておく.
それぞれの計算はピクセルごとに行なうことができるため, 最大の反転回数と,円の数の
多項式時間で描画することができる.

このアルゴリズムをより一般化した擬似コードをアルゴリズム\ref{alg:iis2d}に示す.
後に我々は単純な円の反転以外の生成元を導入する.
そこで,基本領域にある点に対しては恒等写像で, その他の点に対しては反転の合成を
適用する写像$G$を考え,これを反転の代わりに用いることとする.

 \begin{algorithm}
  \caption{Iterated Inversion System (IIS)}
  \label{alg:iis2d}
  \begin{algorithmic}
   \REQUIRE count $= 0$ and coordinates $=$ position determined by
   pixel
   \FOR{$i=0$ to MAX\_INVERSION}
   \STATE inFundamentalDomain $\leftarrow$ \TRUE
   \FOR{ each map $G$ in Maps }
   \IF{$G$ is available to coordinates}
   \STATE coordinates $\leftarrow$ $G(\text{coordinates})$
   \STATE INCREMENT count
   \STATE inFundamentalDomain $\leftarrow$ \FALSE
   \ENDIF
   \ENDFOR
   \IF {inFundamentalDomain}
   \STATE BREAK for
   \ENDIF
   \ENDFOR
   \STATE RETURN count
  \end{algorithmic}
 \end{algorithm}

\begin{figure}[htbp]
 \center
 \includegraphics[ height=1.5in, keepaspectratio]{../img/klein/orb.pdf}
 \caption{Orbit}
 \label{fig:iis-orb}
\end{figure}

このアルゴリズムは三次元空間における球の反転に対しても用いることができる.
しかし,反転球の軌道は球の内側に入り込むため,二次元の場合と同様に描画するには手間がかかる.
そこで我々は軌道の元となる球を用意し, その球を反転球の反転の組み合わせで移した軌道を描画する.
これはレイマーチングとDistance Estimationを用いることで高速に描画することができる.

図\ref{fig:simpleGen3d}における灰色の球を\emph{反転球},緑色の球を\emph{基本球}とよぶ.
基本球をすべての反転球の反転の組合わせで移すと図\ref{fig:simpleOrb3d}のよう
な形状が得られる.
基本球の数を増やすことも可能である. 図\ref{fig:3baseSphere}は3つの基本球
の軌道を描画した.

レイマーチングに用いる距離関数を求めるため,
ここで簡単のために, 図\ref{fig:iis-orb}のXY平面での断面を考えることにする.
これを図\ref{fig:xySlice}に示した.
橙色の円列はショットキー球の軌道, 水色の円が基本球の軌道である.
緑色の点$P2$をレイの先端であるとすると最も近い球は$S2$となる.
しかし, $S2$の位置と半径は分からない.
そこで我々は変換のヤコビアンを用いて, 基本球である$S1$の位置と半径から距離を逆算する.

反転球の中心を$S$その半径を$R$,半径を適用する前の点を$P$とおくと
球の反転のヤコビアンは以下のように計算できる.
$ Jacobian = R^2 / distance(P, S)^2 $
また, 半径が無限大の球の反転のヤコビアンは1であることに注意する.

距離関数では,レイの先端にIISを作用させ,
球の反転のヤコビアンを反転をおこなう毎にかけあわせていく.
最終的に基本領域上の点と基本球との距離をヤコビアンの積で割ることで近似された距離を求めることができる.

この近似は荒いため, もう一つ考慮すべき事がある.
レイの先端が極限集合より外側にあるとき, その点は球の反転によって外側に移さ
れるため, 距離関数が正しく機能せず,大きな距離を返してしまい,
レイは予期せずしてオブジェクトを突きぬけてしまう.
よってオブジェクトの前面が正しくレンダリングされずに,
図\ref{fig:artifact}のような乱れを生みだす.

このような不具合を避けるため, 算出された距離を縮小する方法をとった.
距離の縮小率は球の大きさによって実験的に決められる.

基本球の大きさを変更することで, 形状を変形することができる.
図\ref{fig:limitSetOnSphere}は基本球の大きさを極限集合と同じ大きさにしたものである.
球の上に極限集合が描画されていることがわかる.

一般化した擬似コードはアルゴリズム\ref{alg:iis3d}に示した.

\begin{figure}[htbp]
 \begin{minipage}[b]{0.5\hsize}
  \begin{minipage}[]{0.49\hsize}
   \center
   \includegraphics[ height=1.5in, keepaspectratio]{../img/klein/simpleGen.pdf}
   \subcaption{Generator}
   \label{fig:simpleGen3d}
  \end{minipage}
  \begin{minipage}[]{0.49\hsize}
   \center
   \includegraphics[ height=1.5in, keepaspectratio]{../img/klein/simpleOrbit.pdf}
   \subcaption{Orbit}
   \label{fig:simpleOrb3d}
  \end{minipage}
  \caption{The orbit of the green sphere}
  \label{fig:iis-orb}
 \end{minipage}
 \begin{minipage}[b]{0.5\hsize}
  \center
  \includegraphics[ height=1.5in, keepaspectratio]{../img/klein/xySlice.pdf}
  \caption{XY-slice image of Figure \ref{fig:iis-orb}}
  \label{fig:xySlice}
 \end{minipage}
\end{figure}

\begin{figure}[htbp]
 \begin{minipage}{0.5\hsize}
  \center
  \includegraphics[ height=1.5in, keepaspectratio]{../img/klein/artifact.pdf}
  \caption{The Artifact}
  \label{fig:artifact}
 \end{minipage}
 \begin{minipage}{0.5\hsize}
  \center
  \includegraphics[ height=1.5in, keepaspectratio]{../img/klein/3dLimitSet.pdf}
  \caption{The Limit set on the sphere}
  \label{fig:limitSetOnSphere}
 \end{minipage}
\end{figure}

\begin{figure}[htbp]
 \begin{minipage}{0.5\hsize}
  \center
  \includegraphics[ height=1.5in, keepaspectratio]{../img/klein/3baseGen.pdf}
  \subcaption{Generator}
  \label{fig:}
 \end{minipage}
 \begin{minipage}{0.5\hsize}
  \center
  \includegraphics[ height=1.5in, keepaspectratio]{../img/klein/3baseOrb.pdf}
  \subcaption{Orbit}
  \label{fig:}
 \end{minipage}
 \caption{The orbit of 3 base spheres}
 \label{fig:3baseSphere}
\end{figure}

\begin{algorithm}
 \caption{Distance function}
 \label{alg:iis3d}
 \begin{algorithmic}
  \REQUIRE count $= 0$, $d$ = MAX\_DISTANCE, $dr = 1.0$, and coordinates $=$ tipping
  point of the ray
  \FOR{$i=0$ to MAX\_INVERSION}
  \STATE inFundamentalDomain $\leftarrow$ \TRUE
  \FOR{ each Map $G$ in Maps}
  \IF{$G$ is available to coordinates}
  \STATE coordinates $\leftarrow$ $G(\text{coordinates})$
  \STATE $dr \leftarrow dr * $ (Jacobian of $G(\text{coordinates})$)
  \STATE INCREMENT count
  \STATE inFundamentalDomain $\leftarrow$ \FALSE
  \ENDIF
  \ENDFOR
  \IF {isInFundamentalDomain}
  \STATE BREAK for
  \ENDIF
  \ENDFOR
  \FOR{ each BaseSphere $S$ in BaseSpheres}
  \STATE $d \leftarrow$ min($d$, scalingFactor * (distance(coordinates, $S$.center) $-$
  $S$.radius) $/$ (absolute value of $dr$))
  \ENDFOR
  \RETURN $d$
 \end{algorithmic}
\end{algorithm}

\subsection{Geometrical Representation of M\"obius Transformations}

木構造を用いて群を可視化する際には,変換を行列で考えたが,行列表現でメビ
ウス変換を操作することは直観が得にくい.
例えば,行列による変換の表現から変換の作用を推測することは非常に難しい.

また,三次元空間に作用するメビウス変換を考えることで,高次元のクライン群を作ることができる.
図は四次元クライン群の極限集合を可視化したものである.
このように,三次元上での捩じれをもつ非常に興味深い形状が得られる.
しかし,このような群を構成するメビウス変換は,
四元数行列で表現されるため,さらに複雑な代数計算が必要となる.

そこで筆者はすべてのメビウス変換を円や球の反転で構成し,その軌道を描くことを考えた.
円や球を用いることで,幾何学的な直観が働き,生成元と可視化される図形の間の関係性を
容易に理解することができる.
それに加え,IISを用いることで複雑な生成元をもつ群もリアルタイムに可視化
することができる.
このことは,研究者や学習者だけでなく,フラクタルアーティストにも有益なも
のになる.
筆者は円や球の反転で構成される群をインタラクティブに構成するウェブアプリケーション,
{\it Schottky Link} \footnote{Schottky Link:
\url{https://schottky.jp}}を開発している.

\subsubsection{2D Generators}

\noindent\textbf{Simple Inversion}
先に述べたように, 円に関する反転は複素平面の向きを変えるので, メビウス変
換ではない.
ふたつの反転円がペアとなってはじめてメビウス変換となる.
しかし, ここでは簡単のためにメビウス変換の元として扱う.

\noindent\textbf{Inversion of a Circle with Infinite Radius}
無限の半径をもつ円に関する反転は直線に関する反転で表される.
図\ref{fig:infCircle}では右側の反転円が左側の半径無限の円の内側に反転で移されていることがわ
かる.

\noindent\textbf{Rotation}
二つの交差する直線の反転の組み合わせは回転を表わす. 回転角は二直線のなす
角の二倍である. 図\ref{fig:rotation}では90度回転を示した. また, 軌道となる円同士がお互いに
重なりあわないようにするためには, 回転角は有理角である必要がある.

\noindent\textbf{Parallel Translation}
平行な二直線による反転の組は平行移動を表す.
図\ref{fig:translation2d}では二つの無限の半径をもつ円が向かい合っている.
これらの円の反転によって,中央の4つの反転円はx軸方向に移されている.
平行移動の固定点は無限遠点であり,これは放物型変換である.

\noindent\textbf{Composition of Two Circles}
二つの同心円をもちいることで, 双曲型変換である単純な拡縮を表わすことがで
きる.
図\ref{fig:scaling2d}における赤い円を$C1$,緑の円を$C2$, そして青い円を$C1'$とよぶ.
$C1'$は$C1$を$C2$による反転で移した像である.
図\ref{fig:scaling2d}では, 白い縁をもつ二つの円の軌道を描いている.
固定点は円の中心と無限遠点である.
写像$G$は以下のように定義される.
また,ここで接頭辞$I$は反転を表わす.
例えば, $I_{C1}$は$C1$による反転を表わす.
\begin{align*}
 G =
  \begin{cases}
   I_{C2} \circ I_{C1} & (\text{The point is inside of } C1) \\
   I_{C1} \circ I_{C2} & (\text{The point is outside of } C1')
  \end{cases}
\end{align*}
$G$を点に繰り返し作用させることで基本領域へと移動させることができる.
この種類の生成元の基本領域は青と緑の領域である.
完全な円の軌道をIISを用いて描くには, 反転円をこの領域内に配置す
る必要がある.

次に$C1$を少し動かすと, 図\ref{fig:hyperbolic2d}のような軌道が得られる.
無限遠点の固定点は有限の点に移り,有限の固定点は, $C1$の内部で移動する.
二つの円が交差しない限り, これも双曲型変換である.

$C1$と$C2$が接触する時, 図\ref{fig:parabolic2d}のように,固定点はその交点で重なりあい, 放物型変換となる.

\noindent\textbf{Loxodromic}
上記の生成元の軌道は実数による拡縮と共役であった.
この生成元にさらに二つの反転を加えることで軌道に捻りを加えることができる.
図\ref{fig:loxodromic2d}における$C1$と$C2$の中心を通る白い直線を$L$,
黄色の円を$C3$, そして水色の点を$P$とよぶ.
$P$は制御点であり, $C3$は$P$とその$C1$と$C2$による反転の像である$P'$と
$P''$の三点から定義される.
$L$と$C3$の反転の組み合わせは回転を表し,これが軌道に捻りを加える.

写像$G$は以下のようになる.
\begin{align*}
G =
\begin{cases}
 (I_{C2} \circ I_{C1}) \circ (I_{C3} \circ I_L) & (\text{The point is inside of } C1) \\
 (I_L \circ I_{C3}) \circ (I_{C1} \circ I_{C2}) & (\text{The point is outside of }C1')
\end{cases}
\end{align*}
また, 二つの固定点からこの種類の生成元を得ることもできる.

\begin{figure}[h!tbp]
 \begin{minipage}[t]{0.3\hsize}
  \center
   \includegraphics[width=2in, height=2in, keepaspectratio]{../img/klein/2diis/infCircle.pdf}
  \caption{Inversion of the circle with infinite radius}
  \label{fig:infCircle}
 \end{minipage}
 \hspace*{\fill}
 \begin{minipage}[t]{0.3\hsize}
  \center
  \includegraphics[width=2in, height=2in, keepaspectratio]{../img/klein/2diis/rotation.pdf}
  \caption{Rotation}
  \label{fig:rotation}
 \end{minipage}
 \hspace*{\fill}
 \begin{minipage}[t]{0.3\hsize}
  \center
  \includegraphics[width=2in, height=2in, keepaspectratio]{../img/klein/2diis/translation.pdf}
  \caption{Parallel translation}
  \label{fig:translation2d}
 \end{minipage}
\end{figure}

\begin{figure}[h!tbp]
 \begin{minipage}[]{0.65\hsize}
  \begin{minipage}[]{0.22\hsize}
   \center
   \includegraphics[width=1.5in, height=1.5in, keepaspectratio]{../img/klein/2diis/scalingEdged.pdf}
   \subcaption{Scaling}
   \label{fig:scaling2d}
  \end{minipage}
 \hspace*{\fill}
  \begin{minipage}[]{0.22\hsize}
   \center
   \includegraphics[width=1.5in, height=1.5in, keepaspectratio]{../img/klein/2diis/hyperbolicEdged.pdf}
   \subcaption{Hyperbolic}
   \label{fig:hyperbolic2d}
  \end{minipage}
 \hspace*{\fill}
  \begin{minipage}[]{0.22\hsize}
   \center
   \includegraphics[width=1.5in, height=1.5in, keepaspectratio]{../img/klein/2diis/parabolicEdged.pdf}
   \subcaption{Parabolic}
   \label{fig:parabolic2d}
  \end{minipage}
  \caption{Composition of two circles}
 \end{minipage}
 \hspace*{\fill}
 \begin{minipage}[]{0.22\hsize}
  \center
  \includegraphics[width=1.5in, height=1.5in, keepaspectratio]{../img/klein/2diis/loxodromicEdged.pdf}
  \caption{Loxodromic transformation}
  \label{fig:loxodromic2d}
 \end{minipage}
\end{figure}

\subsubsection{3D Generators}
ほとんどの三次元の生成元は二次元の生成元から自然に導くことができる.

\noindent\textbf{Simple Inversion.}
二次元の場合と同様に球の反転単体はメビウス変換ではなく, 二つの球のペアに
よる反転がメビウス変換である.
しかし, ここではこれもメビウス変換として扱う.

l\noindent\textbf{Inversion of a Sphere with Infinite Radius}
半径が無限大の球による反転は平面による反転と表すことができる.
図\ref{fig:infSphereGen}では青い板が半径が無限の球を表現している.
図\ref{fig:infSphereOrb}における軌道の右側はこの板によって左側へと
移されることによって左右に対称性をみることができる.

\noindent\textbf{Rotation}
二次元の場合と同様にして,互いに交わる二つの平面の反転によって
回転を表すことができる.
回転軸は二枚の平面の交線で,なす角の二倍が回転角となる.
図\ref{fig:rotation3d}では180度の回転を示した.

\noindent\textbf{Parallel Translation}
二つの平行な平面の組は平面の法線方向への平行移動を表わす.
図\ref{fig:translation3d}のような生成元は無限遠点に固定点
をもつ放物型変換である.

\noindent\textbf{Compound Parabolic}
更に, 三次元では平行移動の軌道に捻りを加えることができる.
図\ref{fig:compParabolic}では軌道に平行移動の元が適用されるたびに回転
が加えられていることがわかる.
この生成元は\emph{複合放物型変換}(\textit{Compound Parabolic})とよばれる.

\noindent\textbf{Composition of Two Spheres}

二次元の場合と同様に二つの球の反転を組み合わせた生成元を定義することができる.
図\ref{fig:loxoGen3d}に示される赤い球を$S1$, 黄緑の球を$S2$,
青い球を$S1'$とよぶとき,写像$G$は二次元と同様に以下のようになる.
\begin{align*}
G =
\begin{cases}
 I_{S2} \circ I_{S1} & (\text{The point is inside of } S1) \\
 I_{S1} \circ I_{S2} & (\text{The point is outside of }S1')
\end{cases}
\end{align*}
$S1$と$S2$に接触がない場合,この生成元は斜航型変換となる.
軌道は図\ref{fig:loxoOrb3d}に示した.
$S1$と$S2$が一点で接触し時, 放物型変換となる.
その接点が固定点となる.
図\ref{fig:parabolic3d}において球の軌道は固定点ので接触していることがわかる.

\noindent\textbf{Compound Loxodromic}
最後に, $S1$と$S2$に直交する二つの球を追加する.
図\ref{fig:compLoxoGen}に示される桃色の球を$S3$, 黄色の球を$S4$,
3つの制御点を$P$,$Q1$,$Q2$とよぶ.
球は4つの点により決定される.
$P'$と$P''$を$P$の$S1$と$S2$の反転による像とすると,
$G$は以下のようになる.
\begin{align*}
S3 = Sphere(P, P', P'', Q1) \quad
S4 = Sphere(P, P', P'', Q2) \\
G =
\begin{cases}
 (I_{S4} \circ I_{S3}) \circ (I_{S1} \circ I_{S2}) & (\text{The point is inside of } S1) \\
 (I_{S2} \circ I_{S1}) \circ (I_{S3} \circ I_{S4}) & (\text{The point is
 outside of } S1')\\
\end{cases}
\end{align*}
$S3$と$S4$の反転の組み合わせは回転を表す.
図\ref{fig:compLoxoOrb}では, 二次元における斜航型変換のような捻じれた
軌道をみることができる.
この生成元を\emph{複合斜航型変換}(Compound Loxodromic)とよぶ.


\begin{figure}[h!tbp]
 \begin{minipage}[t]{0.49\hsize}
  \begin{subfigure}{0.24\textwidth}
   \begin{center}
    \includegraphics[width=1.5in, height=1.5in, keepaspectratio]{../img/klein/3diis/infSphereGen.pdf}
    \caption{Generator}
    \label{fig:infSphereGen}
   \end{center}
  \end{subfigure}
  \hspace*{\fill}
  \begin{subfigure}{0.24\textwidth}
   \begin{center}
    \includegraphics[width=1.5in, height=1.5in, keepaspectratio]{../img/klein/3diis/infSphereOrbit.pdf}
    \caption{Orbit}
    \label{fig:infSphereOrb}
   \end{center}
  \end{subfigure}
  \hspace*{\fill}
  \caption{Inversion of the Sphere with infinite radius}
  \label{fig:infSphere}

 \end{minipage}
 \hspace*{\fill}
 \begin{minipage}[t]{0.49\hsize}

  \begin{subfigure}{0.24\textwidth}
   \begin{center}
    \includegraphics[width=1.5in, height=1.5in, keepaspectratio]{../img/klein/3diis/rotationGen.pdf}
    \caption{Generator}
    \label{fig:rotationGen}
   \end{center}
  \end{subfigure}
  \hspace*{\fill}
  \begin{subfigure}{0.24\textwidth}
   \begin{center}
    \includegraphics[width=1.5in, height=1.5in, keepaspectratio]{../img/klein/3diis/rotationOrb.pdf}
    \caption{Orbit}
    \label{fig:rotationOrb}
   \end{center}
  \end{subfigure}
  \hspace*{\fill}
  \caption{Rotation}
  \label{fig:rotation3d}
 \end{minipage}
\end{figure}

\begin{figure}[h!tbp]
 \begin{minipage}{0.49\hsize}
  \begin{subfigure}{0.24\textwidth}
   \begin{center}
    \includegraphics[width=1.5in, height=1.5in, keepaspectratio]{../img/klein/3diis/translationGen.pdf}
    \caption{Generator}
    \label{fig:translationGen}
   \end{center}
  \end{subfigure}
 \hspace*{\fill}
 \begin{subfigure}{0.24\textwidth}
  \begin{center}
   \includegraphics[width=1.5in, height=1.5in, keepaspectratio]{../img/klein/3diis/translationOrbit.pdf}
   \caption{Orbit}
    \label{fig:translationOrb}
  \end{center}
 \end{subfigure}
 \hspace*{\fill}
\caption{Parallel translation}
 \label{fig:translation3d}

 \end{minipage}
 \begin{minipage}{0.49\hsize}
  \begin{subfigure}{0.24\textwidth}
   \begin{center}
    \includegraphics[width=1.5in, height=1.5in, keepaspectratio]{../img/klein/3diis/compParabolicGen.pdf}
    \caption{Generator}
    \label{fig:compParabolicGen}
   \end{center}
  \end{subfigure}
 \hspace*{\fill}
 \begin{subfigure}{0.24\textwidth}
  \begin{center}
   \includegraphics[width=1.5in, height=1.5in, keepaspectratio]{../img/klein/3diis/compParabolicOrb.pdf}
   \caption{Orbit}
   \label{fig:compParabolicOrb}
  \end{center}
 \end{subfigure}
 \hspace*{\fill}
\caption{Compound parabolic transformation}
 \label{fig:compParabolic}
 \end{minipage}
\end{figure}

\begin{figure}[h!tbp]

  \begin{subfigure}{0.24\textwidth}
   \begin{center}
    \includegraphics[width=1.5in, height=1.5in, keepaspectratio]{../img/klein/3diis/loxoGenSimple.pdf}
    \caption{Generator}
    \label{fig:loxoGen3d}
   \end{center}
  \end{subfigure}
 \hspace*{\fill}
  \begin{subfigure}{0.24\textwidth}
   \begin{center}
    \includegraphics[width=1.5in, height=1.5in, keepaspectratio]{../img/klein/3diis/loxoOrbSimple.pdf}
    \caption{Orbit}
    \label{fig:loxoOrb3d}
   \end{center}
  \end{subfigure}
 \hspace*{\fill}
 \begin{subfigure}{0.24\textwidth}
  \begin{center}
   \includegraphics[width=1.5in, height=1.5in, keepaspectratio]{../img/klein/3diis/loxoOrbSch.pdf}
   \caption{Orbit}
   \label{fig:loxoOrbSch3d}
  \end{center}
 \end{subfigure}

  \caption{Loxodromic transformation}
  \label{fig:loxodromic3d}

\end{figure}

\begin{figure}[h!tbp]
 \begin{minipage}{0.49\hsize}
  \begin{subfigure}{0.24\textwidth}
   \begin{center}
    \includegraphics[width=1.5in, height=1.5in, keepaspectratio]{../img/klein/3diis/parabolicGen.pdf}
    \caption{Generator}
    \label{fig:parabolicGen3d}
   \end{center}
  \end{subfigure}
 \hspace*{\fill}
 \begin{subfigure}{0.24\textwidth}
  \begin{center}
   \includegraphics[width=1.5in, height=1.5in, keepaspectratio]{../img/klein/3diis/parabolicOrb.pdf}
   \caption{Orbit}
   \label{fig:parabolicOrb3d}
  \end{center}
 \end{subfigure}
  \hspace*{\fill}
  \caption{Parabolic transformation}
  \label{fig:parabolic3d}
 \end{minipage}
 \begin{minipage}{0.49\hsize}
  \begin{subfigure}{0.24\textwidth}
   \begin{center}
    \includegraphics[width=1.5in, height=1.5in, keepaspectratio]{../img/klein/3diis/compLoxoGen.pdf}
    \caption{Generator}
    \label{fig:compLoxoGen}
   \end{center}
  \end{subfigure}
 \hspace*{\fill}
 \begin{subfigure}{0.24\textwidth}
  \begin{center}
   \includegraphics[width=1.5in, height=1.5in, keepaspectratio]{../img/klein/3diis/compLoxoOrb.pdf}
   \caption{Orbit}
   \label{fig:compLoxoOrb}
  \end{center}
 \end{subfigure}
  \hspace*{\fill}
  \caption{Compound loxodromic transformation}
  \label{fig:compLoxo}
 \end{minipage}
\end{figure}

\begin{figure}[h!tbp]
 \begin{minipage}[t]{0.24\hsize}
  \begin{center}
   \includegraphics[width=1.5in, height=1.5in, keepaspectratio]{../img/klein/2diis/translationMod.pdf}
  \end{center}
  \caption{Parallel translation}
  \label{fig:translationMod}
 \end{minipage}
 \hspace*{\fill}
 \begin{minipage}[t]{0.24\hsize}
  \begin{center}
   \includegraphics[width=1.5in, height=1.5in, keepaspectratio]{../img/klein/2diis/parabolicMod.pdf}
  \end{center}
  \caption{Inverted parabolic generator}
  \label{fig:parabolicMod}
 \end{minipage}
 \hspace*{\fill}
 \begin{minipage}[t]{0.24\hsize}
   \begin{center}
    \includegraphics[width=1.5in, height=1.5in, keepaspectratio]{../img/klein/2diis/hyperbolicMod.pdf}
   \end{center}
   \caption{Inverted hyperbolic generator}
   \label{fig:loxodromicMod}
 \end{minipage}
  \hspace*{\fill}
 \begin{minipage}[t]{0.24\hsize}
   \begin{center}
    \includegraphics[width=1.5in, height=1.5in, keepaspectratio]{../img/klein/2diis/loxodromicModRotation.pdf}
   \end{center}
   \caption{Inverted loxodromic generator}
   \label{fig:loxodromicRotationMod}
  \hspace*{\fill}
 \end{minipage}
\end{figure}

\subsubsection{Optimization}

いくつかの反転で構成された写像は適切な共役をとることによって, 最適化する
ことができる.
最適化された写像は点を一度の操作で基本領域に移動させることができる.
この節では, 二次元の生成元の最適化についてまとめる.

\noindent\textbf{Parallel Translation.}
まずは図\ref{translation2d}と同様の平行な二直線の組による平行移動を考える.
繰り返し反転を適用する代わりに, 剰余を用いて点を移すことができる.

まず始めに, 与えられた点を平行移動と回転を用いて,
平行な二直線がX軸に垂直に, 片方の直線がY軸に沿うようにする.
これらの変換が共役変換となる.
共役変換で移された直線は図\ref{translationMod}に示した.
$w$を二直線の距離, $i$を平行移動によって移された半径無限の円の指数,
$P$を写像に与えられた点,
$P'$を写像によって移された点, $d$と$d'$をY軸からの距離とする.
$d$を$w$で割ることにより, 剰余$d'$と商$i$を得る.
最後に$d'$を用いて$P'$を計算し, これを共役変換で元の座標に戻す.
以上の操作を一度行うことで,平面上の全ての点を基本領域に移すことができる.

\noindent\textbf{Parabolic.}
次に図のような放物型変換を考える.
まず共役変換$T$をこの変換の固定点を中心とする円の反転とする.
$T$を$C1,C2,C1'$に対して適用すると, 固定点は無限遠点に移るので, 三本の平行な直
線を得ることができる.
これらの直線を$TC1, TC2, TC1'$とよぶ.
図ではこれらの直線が示されている.
赤の直線$TC1$と青の直線$TC1'$が平行移動を表す.
写像の過程は以下のようになる.
まず, $T$を与えられた点$P$に適用し, $T(P)$を得る.
そして, $T(P)$を平行移動と同様にして移し, $P'$を得る.
最後にもう一度$T(= T^{-1})$を$P'$に適用する.

\noindent\textbf{Loxodromic}
最後に図のような双曲型変換を考える.
共役変換$T$を片方の固定点を中心にもつ円に関する反転とする.
$T$を$C1, C2, C1'$に適用した像を$TC1, TC2, TC1'$とよび, これらは, もう一
方の固定点の反転した像を中心にもつ同心円となる.
これは実数倍を表す.
図において, $P$を写像に与えられた点, $P'$を写像された点, $r$と$r'$を
$TC1$と$TC1'$の半径, $d$と$d'$を$TC1$からの距離, $k$を倍数, $q$を
$exponential ciefficient$,  $i$を円列の指数とする.
これらは以下のように計算される.
 \begin{math}
  k = \frac{r'}{r},
  q = \log_{k} \frac{d}{r},
  d' = r \cdot k^{fracionalPart(q)},
  i = floor(q).
 \end{math}
我々は$d'$を用いて$P'$を計算することができる.

また, 図のような双曲型変換であるとき, $T$を$L$と$C3$にも適用し, その像を
$TL$と$TC3$とよぶ.
図は複素数によるスケーリングを表している.
白い直線$TL$と黄色の直線$TC3$は同心円の中心で交わっている.
$P'$を得た後で回転を与えて$P''$を得る.
$TL$と$TC3$のなす角を$\theta$, 回転角を$\varphi$とすると, $\varphi$は
$\varphi = 2 \theta i$で計算することができる.

まず, $T$を与えられた点$P$に適用し, $T(P)$を得る.
そして, $d'$を用いて$P'$を計算する.
もしも交差する直線があれば, $P'$を$\varphi$回転させ, $P''$を得る.
最後にもう一度$T$を$P'$もしくは$P''$に適用する.

\subsection{Other Topics}

この節では, その他のレンダリング手法とクライン群の研究の中でも興味深い図
像を見ることができる先行研究を紹介する.

\subsubsection{Other Rendering Methods}
Aaron MontagはIterated Inversion Systemを拡張したHyperbolic Iterated
Function Systemによる極限集合の描画を提案している\cite{Montag2014hyperbolicIFS}
\footnote{Kleinian Groups WebGL:
 \url{https://www-m10.ma.tum.de/bin/view/Lehrstuhl/AaronMontagKleinian}}.

また, Jos Ley・Knightyはマスキット群におけるEscape-timeアルゴリズムの式
とDistance Estimationを考案した
\footnote{An escape time algorithm for Kleinian group limit set:\\ \quad
\quad \url{http://www.fractalforums.com/3d-fractal-generation/an-escape-tim-algorithm-for-kleinian-group-limit-sets/}}.
荒木・糸\cite{araki2008extension}の考えた群と同様に, 一方に平行移動を含む2つの放物型変換による群を考えている.
この方法を用いると, 二次元だけでなく,三次元の図像も高速に描画することができる.
詳細なアルゴリズムに関しては, Jos Leyがまとめている
\footnote{Mathematical Imagery, An escape-time algorithm for a family of
Kleinian groups:\\ \quad \quad \url{http://www.josleys.com/article_show.php?id=221}}.

\subsubsection{Quasi Fuchsian 3-Dimensional Fractals}

阿原・荒木は擬フックス群を三次元に拡張した4次元クライン群を考案した
\cite{ahara2003sphairahedral}\cite{ahara2003sphaira}.
これはQuasi Fuchsian 3-Dimensional Fractalsとよばれ,
前述のフラクタルコミュニティに大きな影響を与えたといわれている.
荒木はこのフラクタル図形の物質化についてまとめている\cite{araki2006materializing}.
数学的な事項については2016年に蔭山\cite{kageyama2016masterSphaira}がまとめている.

図\ref{fig:schottky}において, 中央の円盤に囲まれた黒い領域を\emph{円辺形}と
よぶ.
擬フックス群は円辺形を囲む円盤の組から生成元を得ることができる.
そこで円辺形を三次元に拡張した\emph{球面体}を用いることで三次元の擬フッ
クス群を構成することができる.
図\ref{fig:sphairahedra}は6つの球に囲まれた球面体のひとつである.
側面は3つの半径無限大の球に囲まれていると考える.
この球面体をそれを囲む球の反転で移していくことで図\ref{fig:quasiFuchsian}のような形状を得る.
阿原・荒木による動画\footnote{Quasi-fuchsian fractals:
\url{https://www.youtube.com/watch?v=3lcO9zRCv-4}}ではこのフラクタルの構成過程をみることができる.

また, Distance Estimationによる高速描画のアルゴリズムが2012年にKnightyによって開発された
\footnote{Another 3D Kleinian:
 \url{http://www.fractalforums.com/ifs-iterated-function-systems/another-3d-kleinian/}}.

\begin{figure}[htbp]
 \begin{minipage}{0.49\hsize}
  \begin{center}
   \includegraphics[width=3in, height=3in, keepaspectratio]{../img/klein/sphairahedra.pdf}
   \caption{Sphairahedra}
   \label{fig:sphairahedra}
  \end{center}
 \end{minipage}
 \begin{minipage}{0.49\hsize}
  \begin{center}
   \includegraphics[width=3in, height=3in, keepaspectratio]{../img/klein/quasi-fuchsian.pdf}
   \caption{quasi-fuchsian}
   \label{fig:quasiFuchsian}
  \end{center}
 \end{minipage}
\end{figure}

\subsubsection{Other 4-Dimensional Kleinian Groups}

インドラの真珠で紹介されるクライン群は複素平面に作用するメビウス変換から構成されている.
現在このようなクライン群は, ほぼ研究され尽されてしまった.
しかし, 先に述べたように, 三次元空間に作用するメビウス変換を四元数を用い
て定義することができる.
これは$sp^k(1, 1)$と呼ばれ,2x2の四元数行列で表現される.
三次元空間上で捩じりが加わるようなメビウス変換は複合放物型, 複合斜航型と分類
される.
四元数で構成されるメビウス変換とその分類は佐久川による
\cite{sakugawa2007master}や\cite{sakugawa2009accidental}にまとまっている.

図\ref{fig:sakugawa}は佐久川による複合放物型変換を含む群のレシピによる群
の極限集合を描画したものである.
放物型変換である平行移動に回転が加わることで複合放物型変換となる.

荒木・糸はマスキット群を高次元に拡張した\cite{araki2008extension}
\footnote{4-dimensional Kleinian punctured torus groups:
\url{http://www.math.nagoya-u.ac.jp/~itoken/3d-maskit/3d-maskit.html}}.
論文中では幾何学的性質をうまく使うことで,四元数の計算を避けている.
また, 佐久川はこの群の四元数表示を導出した\cite{sakugawa2010limit}.

三浦は3つ以上の生成元による4次元クライン群をモジュラー群から構成するこ
とを試みた\cite{miura2015master}.
筆者はこの論文における数値実験と可視化を行った.

\begin{figure}[h!tbp]
 \begin{minipage}{0.49\hsize}
  \center
  \includegraphics[width=3in, height=3in, keepaspectratio]{../img/klein/sakugawa1.pdf}
  \subcaption{}
 \end{minipage}
 \hspace*{\fill}
 \begin{minipage}{0.49\hsize}
  \center
  \includegraphics[width=3in, height=3in, keepaspectratio]{../img/klein/sakugawa2.pdf}
  \subcaption{}
 \end{minipage}
 \begin{minipage}{0.49\hsize}
  \center
  \includegraphics[width=3in, height=3in, keepaspectratio]{../img/klein/sakugawa3.pdf}
  \subcaption{}
 \end{minipage}
 \hspace*{\fill}
 \begin{minipage}{0.49\hsize}
  \center
  \includegraphics[width=3in, height=3in, keepaspectratio]{../img/klein/sakugawa4.pdf}
  \subcaption{}
 \end{minipage}
 \caption{The limit set of the 4-Dimensional Kleinian Groups}
 \label{fig:sakugawa}
\end{figure}

\subsubsection{Once Punctured Torus Groups}

クライン群の部分群である一点穴開きトーラス群({\it Once Punctured Torus Groups})は
和田による{\it OPTi}\footnote{OPTi:
\url{http://delta-mat.ist.osaka-u.ac.jp/OPTi/index.html}}という描画ソフ
トウェアによって大きく研究が進んだ.
描画や離散性判定のためのアルゴリズ
ムが開発者の和田によりまとめられている
\cite{wada2003optiDrawingLimit}\cite{wada2006optiDiscreteness}.
この群の極限集合は図\ref{fig:opt}のようになる.

 \begin{figure}[htbp]
  \begin{minipage}{0.5\hsize}
   \center
   \includegraphics[width=3in, height=3in,
   keepaspectratio]{../img/klein/opt1N.pdf}
   \subcaption{}
  \end{minipage}
  \begin{minipage}{0.5\hsize}
   \center
   \includegraphics[width=3in, height=3in,
   keepaspectratio]{../img/klein/opt2N.pdf}
   \subcaption{}
  \end{minipage}
  \caption{The Limit Set of the once punctured torus groups}
  \label{fig:opt}
 \end{figure}

\subsection{Further Readings}
この章の最後に,クライン群のより数学的な背景を学習するための書籍をいくつか挙げておく.
双曲幾何学の入門書には「双曲幾何学への招待」\cite{taniguchi_okumura199610invitation}や「双曲幾何」\cite{mitani200409hyperbolicGeometry}がある.
クライン群のより数学的内容に踏み込んだ書籍には『双曲多様体とクライン群』
\cite{taniguchi_matsuzaki_hyperbolicManifold}や『Outer Circles』\cite{Marden200705outerCircles}がある.
また, レクチャーノート集に『Kleinian Group and Hyperbolic 3-Manifolds』\cite{Y._V._C.200311}や
『Spaces of Kleinian Groups』\cite{Yair_Makoto_Caroline200606}がある.

\clearpage
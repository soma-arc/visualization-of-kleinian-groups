\section{Fractal}
フラクタルは数学者は数学者ブノワ・マンデルブロが提唱した幾何学の概念である.フラクタルとは,自己相似性という特徴を持つ.どこを拡大しても自分と同じ形を見ることができる.自然界にも多くの形を見ることができる.
コンピュータの普及によって,アートとしても大きく発展した.

\subsection{Mandelbrot Set}
マンデルブロ集合はマンデルブロの名からとられた最も有名といえるフラクタルである.
複素数平面上の点c全体に対して,以下の漸化式を計算する.
\begin{eqnarray*}
 \begin{cases}
  z_{n+1} = z^2_{n} + c \\ z_0 = 0
 \end{cases}
\end{eqnarray*}
$z_n$が無限大に発散しないものをマンデルブロ集合と呼ぶ.図では黒い部分が黒い部分がマンデルブロ集合である.先端部分が拡大しても見ることができることがわかる.
カラーリングには様々な方法がある.
漸化式を計算するプロセスは見方を変えると,複素数平面上の点を漸化式で動かしていき,その軌道を見ていると考えることができる.このことから,Escape-time fractalと呼ばれている.Escape-time fractalには,マンデルブロ集合のほかに,ジュリア集合やファタウ集合が存在している.

\subsection{Parallel Rendering}
これまではCPUによって計算されることが多かったが,グラフィクス性能の向上により,GPUを用いた計算を容易に行うことができるようになった.GPUは浮動小数点演算と並列演算が優れている・
ピクセルごとに計算することができるので,容易に並列計算を行うことができる.
最近ではGPUを用いた高速な並列計算を行うことができる.OpenGL Shading Language(GLSL)やHigh Level Shading Language(HLSL)のフラグメントシェーダを用いる方法である.フラグメントシェーダは本来,ポリゴンに陰影をつけるために使用されるが,スクリーンスペースに矩形のポリゴンをレンダリングすることで各ピクセルごとの処理を記述し計算を行うことができる.
その性質上,何か図を描くときには陰関数表示された関数から図を描く.ただし,より扱いやすくするために,ある点から図形までの最短距離を求める手法であるDistance Estimationを使うことが多い.これはInigo Quilez氏の記述が詳しい.

\section{Fractal}
フラクタルは数学者は数学者ブノワ・マンデルブロが提唱した幾何学の概念である.フラクタルとは,自己相似性という特徴を持つ.どこを拡大しても自分と同じ形を見ることができる.自然界にも多くの形を見ることができる.
自己相似性を持つ図形はフラクタルという名がつけられる以前から研究されていた.
コンピュータの普及によって,アートとしても大きく発展した.
Fractal Forumには大きなコミュニティが形成されており,多くのフラクタルファンが日夜議論を重ねている.
また,デモシーンという文化でも美しいコンピュータグラフィクスを効率よく描画する手法が考えられており,フラクタルレンダリングとも関連が深い.デモシーンに関しては,Demoscene - The Art of the Algorithmが詳しい.

\subsection{About Rendering}

その性質上,何か図を描くときには陰関数表示された関数から図を描く.ただし,より扱いやすくするために,最短距離を返す距離関数(Distance Function)を用いることが多い.例えば,円の距離関数はこの関数は以下のようになる.この関数は演習からの距離を返す.微分と陰関数から導出することができる.この手法はDistance Estimationと呼ばれる.詳しい導出はInigo Quilez氏の記述が詳しい.
距離関数は二次元平面だけでなく,三次元空間でも定義することができる.
三次元物体を描画する場合にはレイトレーシングが用いられる.視点からスクリーンへレイ(光線)を飛ばし,その挙動をシミュレーションすることで描画する手法である.レイの方向と位置から物体との交差点を代数的に計算する.
また,レイマーチングという手法を用いることで,距離関数を用いて任意の三次元形状との交差点を近似的に計算することができる.この方法は,物体の形状が代数的に計算することが難しいフラクタル形状のレンダリングに適している.
これらの方法の多くは,コンピュータグラフィクスの黎明期に研究されていたものであるが,コンピュータのグラフィクス性能が向上することで,GPUを用いた並列計算が用いられるようになった.GPUは浮動小数点演算性能と並列演算性能に優れている.
最近よく用いられるレンダリング手法の1つにOpenGL Shading Language(GLSL)やHigh Level Shading Language(HLSL)のフラグメントシェーダを用いるものがある.フラグメントシェーダは本来,ポリゴンに陰影をつけるために使用されるが,スクリーンスペースに矩形のポリゴンをレンダリングすることで各ピクセルごとの処理を記述し計算を行うことができる.
はQuaternion JuriaをGPUでレンダリングすることに成功した.iq氏は4kbという小さな容量の実行ファイルから動画のような作品を制作した.
また,GPUの計算性能をより汎用的な計算に用いるためのプラットフォームとしてCUDAやOpenCLが登場している.これらのプラットフォームを用いることでシェーダではできない複雑な処理を行うことができ,GPUは機械学習などの分野にも活躍の場を広げた.
シェーダを用いたレンダリングを試すことができるWebサイトにglslsandboxやshadertoyといったサイトがある.
GLSLによるレンダリングや,レイマーチングに関しては,doxas氏のwgldが詳しい.

\subsection{Escape-time Fractals}
マンデルブロ集合はマンデルブロ自身が発見したフラクタルであり,Escape-time fractalと呼ばれる種類のフラクタルである.
Escape-time Fractalsは複素平面上の点に対して,特定の漸化式を計算し,その点の軌道の収束,発散を見ることで描画されるフラクタルである.
マンデルブロ集合の漸化式は以下のようになる.
\begin{eqnarray*}
 \begin{cases}
  z_{n+1} = z^2_{n} + c \\ z_0 = 0
 \end{cases}
\end{eqnarray*}
$z_n$が無限大に発散しないものをマンデルブロ集合と呼ぶ.図では黒い部分がマンデルブロ集合である.先端部分が拡大しても見ることができることがわかる.マンデルブロ集合はその単純な式から驚くほど豊富なバリエーションの図を見ることができる.
カラーリングには様々な方法がある.図では発散と判定されるまでの計算の回数によって色をつけた.その他のEscape-time fractalには,ジュリア集合やファタウ集合等がある.
各点における漸化式の計算はお互いに干渉しないので,そのままシェーダでの実装が可能であるが,先に述べたDistance Estimationを用いることで,エイリアシングノイズ等を避けてより精細に描画することができる.Green Functionと呼ばれる陰関数を用いる.導出はiq氏の記事が詳しい.Green Functionに関しては,マンデルブロ集合に関する詳細な議論となってしまうので,こちらの論文を参考にされたい.Doaudy Hubard Potensialを用いることでカラーリングやテセレーションに応用することができる.

\subsection{Distance Estimated 3D Fractals}
メンガーのスポンジ等の形状は見られたが,マンデルブロ集合のような複雑な3次元形状を持つフラクタルが求められていた.マンデルブロ集合を3次元に拡張する試みはさまざまに行われていたものの,あまり大きな成果はあげられていなかった.しかし,年に発表された阿原・荒木による4次元クライン群の一種であるQuasi Fuchsian 3D Fractalsはある程度の複雑性を持った3次元フラクタルであり,コミュニティに大きな影響を与えたと言われている.このフラクタルに関しては後の章で触れる.その後,2009年にマンデルブロ集合をうまく拡張したmandelbulbの出現を皮切りに,次々と複雑なEscape-time fractalの式が発見された.これらもDistance Estimationとレイマーチングを用いることで効率よくレンダリングすることができる.これら一連のフラクタルに関してはSyntopiaの一連のポストにまとめられている.
注意が必要なのは,これらのフラクタルは純粋な数学の研究対象になりにくい.

\subsubsection{Mandelbulb}
mandelbulbはD.WhiteとP.Nylanderによって開発された.
White(twinbee)氏は球面座標を用いるアプローチを提案(http://www.fractalforums.com/3d-fractal-generation/true-3d-mandlebrot-type-fractal/)したがその後,Nylander氏はそれを高次の積に拡張した.
mandelbulbとして知られている式は$z^8 + c $であり,図のような形状をもつ.
その経緯はWhite氏のWebサイトにまとめられている.
レイマーチングを用いる際にはDistance Estimationを用いるが,これには通常のmandelbro集合で用いたものと同じアプローチを用いることができる.

\subsubsection{Mandelbox}
その後,2010年にTom Lowe氏によってMandelboxが開発された.
Mandelboxはball foldやbox foldといった操作を定義する.
sphere foldは円の中心付近に来た点が発散してしまわないように,制限をかけた円の反転である.式は以下のように定義され
https://sites.google.com/site/mandelbox/what-is-a-mandelbox
様々にあるが,ヤコビアンの積で割ることで距離を求めることができる.
その後は各漸化式の計算ごとに異なる式を使うHybrid Systemが考案された.今も様々なフラクタル作品が作られている.

\subsection{Coloring by Orbit trap}

ここでOrbit trapと呼ばれるカラーリングのアルゴリズムを紹介しておく.Escape-timeのような点に軌道が存在

\subsection{Other Fractals}

Iterated Function systemは点に様々な関数を適用して,その集積点をレンダリングするアルゴリズムである.
Fractals EverywhereはIFSによるフラクタルの研究を広めた書籍である.
アルゴリズムやカラーリングに関する事項はThe Fractal Frame Algorithmによくまとまっている.並列計算による高速化に関して以下の論文にまとめられている.

他にはL-Systemなどのアルゴリズムが存在するが,今回は後に記述するクライン群のレンダリングには関連してこないのでここでは触れない.


